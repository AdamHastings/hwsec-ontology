% Anonymous submission toggle:
% - \anonsubmissiontrue  => anonymous submission build
% - \anonsubmissionfalse => named draft / camera-ready build
\newif\ifanonsubmission
\anonsubmissiontrue
\ifanonsubmission
  \PassOptionsToClass{anonymous}{acmart}
\fi
\documentclass[sigconf]{acmart}

\usepackage{amsmath}
\usepackage{booktabs}

\setcopyright{none}
\settopmatter{printfolios=true}
\acmDOI{}
\acmISBN{}
\acmConference[NSPW '26]{New Security Paradigms Workshop}{2026}{TBD}
\acmBooktitle{Proceedings of the New Security Paradigms Workshop (NSPW '26)}

\title{Hardware Security is Security Economics:\\ A Unifying Synthesis and Ontology}

\author{Author Name}
\affiliation{%
  \institution{Affiliation}
  \city{City}
  \country{Country}
}
\email{email@example.com}

\begin{document}
\begin{abstract}
Every security mechanism hides a distributional question that its designers rarely ask
explicitly: \emph{who pays?} The performance tax from speculative execution mitigations
fell on cloud operators who had no role in the original design decision. The integration
burden from capability-based memory safety falls on software ecosystems that must
retrofit decades of code. The compliance overhead from trusted boot requirements falls
on OEMs navigating certification timelines they did not set. In each case, the technical
choice and its economic consequence are made by different actors, recorded in different
documents, and analyzed by different communities---if they are analyzed at all.

This paper argues that this disconnect is not accidental. It is the product of two
traditions that have been analyzing the same underlying problem with incompatible
vocabularies: systems engineering, which characterizes the technical tradeoff space with
precision but treats cost distribution as out of scope, and security economics, which
understands distribution and incentive misalignment clearly but treats the mechanism
as a black box. Neither tradition, alone, can explain why a given mechanism produces
the burden distribution it does---or what alternative designs would have implied a
different one. We propose a synthesis: a burden-allocation framework in which security
mechanisms are understood as instruments for distributing costs, risks, and constraints
across stakeholders and timescales. A lightweight shared vocabulary, grounded in two
case studies, makes the translation between engineering and economic concepts precise
enough to be useful in practice.
\end{abstract}

\maketitle

\begin{CCSXML}
<ccs2012>
<concept>
<concept_id>10002978.10003022.10003023</concept_id>
<concept_desc>Security and privacy~Embedded systems security</concept_desc>
<concept_significance>500</concept_significance>
</concept>
<concept>
<concept_id>10002978.10003022.10003036</concept_id>
<concept_desc>Security and privacy~Systems security</concept_desc>
<concept_significance>300</concept_significance>
</concept>
</ccs2012>
\end{CCSXML}
\ccsdesc[500]{Security and privacy~Embedded systems security}
\ccsdesc[300]{Security and privacy~Systems security}

\keywords{hardware security, ontology, security economics, computer architecture, burden-allocation instruments, risk transfer}

\section{Introduction}
Hardware security decisions are frequently framed as technical tradeoffs (for
example, performance versus protection), but this framing is incomplete.
Security costs are distributed unevenly across actors (chip vendors, OEMs,
cloud operators, users, and regulators) and across time (upfront design effort
versus contingent incident losses). This paper develops a structured ontology to
unify these costs and make tradeoffs explicit.

\subsection{Research Questions}
This paper is guided by four research questions:
\begin{itemize}
  \item \textbf{RQ1}: What cost types are induced by hardware-security mechanisms across the design, deployment, and incident lifecycle?
  \item \textbf{RQ2}: How should these costs be represented so that stakeholder, time horizon, and uncertainty are explicit?
  \item \textbf{RQ3}: Which relations are required to reason about risk transfer, externalities, and incentive misalignment?
  \item \textbf{RQ4}: How can the ontology support comparable analysis across mechanisms with different technical overhead and governance context?
\end{itemize}

\subsection{Contributions}
\begin{itemize}
  \item A lifecycle-oriented taxonomy of security costs in hardware, spanning physical resource, engineering process, compliance, strategic, and externality costs.
  \item A formal ontology design that links mechanisms, threats, stakeholders, and cost-bearing pathways with explicit temporal semantics.
  \item A multi-stakeholder analytical framing that combines direct implementation costs, expected incident loss, and cost externalization.
  \item A set of competency-question-driven evaluation goals for testing whether the ontology supports policy and architecture decision use cases.
\end{itemize}

%=============================================================
\section{Two Traditions, One Problem}
\label{sec:traditions}
%=============================================================

Hardware security mechanisms are evaluated twice, by two communities
that rarely compare notes. Systems engineers evaluate them as points in
a design space: what do they cost in silicon area, power draw, and
runtime performance, and what do those costs foreclose? Security
economists evaluate them as instruments of policy: who adopts them,
who pays for them, and who bears the risk when they fail? Both
evaluations are rigorous. Both are incomplete. And the gap between
them is precisely where the fog of war lives.

%-------------------------------------------------------------
\subsection{The Systems Engineering Lens: Everything is a Tradeoff}
\label{sec:engineering}
%-------------------------------------------------------------

Hardware and systems designers operate under a foundational constraint:
resources are finite and their allocation is zero-sum. Silicon area
committed to a security feature is area unavailable for compute cores,
cache capacity, or memory controllers. Power budgeted to always-on
integrity checking is power unavailable for performance headroom or
battery life. Engineering time spent on formal verification of a
security property is engineering time not spent on new functionality
or performance optimization. This is not a failure of imagination or
will. It is the structure of the problem.

The discipline that has grown up around this constraint is sometimes
called \emph{design space exploration}: the systematic characterization
of how design choices interact, which combinations of objectives are
simultaneously achievable, and where the Pareto frontier lies. In
hardware security, this means that every mechanism comes with a
profile---not a single cost, but a vector of costs and benefits
distributed across multiple dimensions and multiple points in the
product lifecycle.

Consider what a hardware architect actually tracks when evaluating a
security mechanism. At design time: how much die area does it consume,
and what did that area cost in terms of foregone alternatives? How does
it affect the power envelope, and under what workload conditions? Does
it add critical-path latency, and if so, how much clock frequency must
be sacrificed? What verification burden does it add---how many
engineer-months of formal analysis, simulation, and penetration testing
does it require before the design can tape out? At deployment time: does
it interact with specific workload classes in ways that create
performance cliffs? Is it always active, or does it engage only under
specific conditions? Can it be disabled by software, and if so, who
controls that switch and bears the consequences? At lifecycle time: does
it require firmware updates as the threat landscape evolves? Does it
impose certification obligations that must be renewed? When a bypass is
discovered, who owns the patch and who pays for its deployment?

This is a rich and precise analytical framework. It has produced a
substantial body of empirical work: measured overhead figures for
trusted execution environments~\cite{costan2016sanctum,tsai2017graphenesgx},
characterized performance impacts of speculative execution
mitigations~\cite{koruyeh2022retbleed,yan2018invisispec}, documented
area and power costs of hardware memory
tagging~\cite{arm2021mte}, and traced the
verification demands of capability
architectures~\cite{woodruff2014cheri}. Hardware security
researchers know, with considerable precision, what their mechanisms
cost in engineering terms.

What this framework does not track is \emph{who bears those costs}.
The overhead figure exists. The question of which organizational actor
absorbs it---the chip vendor, the OEM integrating the chip into a
platform, the cloud operator running workloads on that platform, or
the enterprise customer whose service degrades---is typically outside
the scope of the analysis. Distribution is not a variable in the
design space exploration. It is treated as someone else's problem,
to be worked out downstream by contracts, markets, and negotiations
that the engineer neither controls nor models.

This is not a criticism of the engineering tradition. Distribution
really is outside the scope of mechanism design, in the same way that
the aerodynamics of a wing are outside the scope of air traffic
control. The problem arises not from what the engineering tradition
does, but from what gets lost when its outputs are used to make
decisions that are implicitly distributional---which, as we will
argue, is most of them.

%-------------------------------------------------------------
\subsection{The Security Economics Lens: Distribution is the Point}
\label{sec:economics}
%-------------------------------------------------------------

A parallel tradition has been analyzing security decisions through
an entirely different lens. Security economics starts not from the
mechanism but from the actors---their incentives, their information,
and the ways their individually rational decisions interact to produce
collective outcomes that may be suboptimal for everyone.

The core insight of this tradition is that security is an economic
problem before it is a technical one~\cite{anderson2001why}. When the
actor who decides whether to invest in a security control is not the
actor who bears the cost of a breach, underinvestment is the
predictable equilibrium---not a mistake, but a rational response to
misaligned incentives. When buyers cannot verify the security
properties of products before purchase, markets for security tend
toward the low end, because vendors who invest in genuine security
cannot credibly signal that investment to customers who cannot
evaluate it~\cite{akerlof1970lemons}. When security failures impose
costs on parties who had no role in the original security
decision---users whose data is exposed, businesses whose supply chains
are disrupted, critical infrastructure that depends on compromised
components---those externalities create a systematic gap between
private and social returns to security investment~\cite{anderson2006economics}.

These concepts have considerable explanatory power. They explain why
software vendors historically underinvested in security relative to
the social optimum: the costs of a breach fell primarily on users,
not vendors, so vendors faced weak incentives to invest beyond what
was necessary for reputational or regulatory compliance. They explain
why security certifications have mixed effectiveness: certification
can signal a minimum baseline but cannot easily signal the
\emph{marginal} security value of investment above that baseline, so
the incentive is to meet the certification bar rather than exceed it.
They explain why incident disclosure norms matter: when vendors can
conceal breaches, the reputational cost of poor security investment
is suppressed, further weakening investment incentives.

In hardware security specifically, the economics lens illuminates
patterns that pure technical analysis cannot. Why did DRAM vendors
initially resist deploying rowhammer mitigations that system vendors
and cloud operators were demanding? Because the costs of the
vulnerability fell primarily on system integrators and operators, not
on DRAM vendors whose products were already
sold~\cite{kim2014flipping,frigo2020trrespass}. Why do OEMs sometimes
ship platforms with security features disabled by default? Because
the integration and support costs of enabled features fall on the OEM,
while the security benefits accrue to end users who may not notice or
value them. Why is firmware security investment chronically
underfunded relative to the risk it addresses? Because firmware
compromise is hard to attribute, liability is diffuse, and the
reputational cost of a firmware breach often falls on the device brand
rather than the firmware vendor who introduced the vulnerability.

What the economics lens does not provide is a way to trace these
distributional outcomes back to specific technical properties of
the mechanisms involved. The analysis operates at the level of
actors, incentives, and market structures. The mechanism is a
parameter---present or absent, adopted or not---rather than an object
of analysis in its own right. The question of \emph{why} rowhammer
mitigations concentrate costs on system integrators rather than DRAM
vendors, and whether alternative mitigation designs would have implied
a different distribution, is not one the economics toolkit is
equipped to answer. That answer is embedded in the technical
properties of the mechanisms themselves: their activation profiles,
their scope, their interaction with the supply chain boundary between
DRAM vendor and system integrator. To see it, you need the
engineering lens. But to know why it matters, you need the economics
lens.

%-------------------------------------------------------------
\subsection{The Gap: A Single Mechanism, Two Descriptions}
\label{sec:gap}
%-------------------------------------------------------------

To make the gap concrete, consider hardware AES acceleration---a
dedicated on-chip engine that performs AES encryption and decryption
in hardware rather than software, present in most modern processors
and system-on-chip designs.

A hardware architect describing this mechanism would reach for a
specific set of concepts. The accelerator occupies die area---on the
order of a few percent of total SoC area in a typical implementation,
area that could alternatively have been used for additional compute
capacity or larger caches. It consumes static power when idle and
dynamic power proportional to utilization. It delivers substantial
throughput improvements for cryptographic workloads---factors of ten
to forty times faster than software AES on the same
platform~\cite{intel2012aesni,xie2024opentitan}---but this benefit
is workload-conditional: applications that perform little encryption
see no benefit and pay only the area and static power cost. The
accelerator has an activation profile: it engages only when the
cryptographic engine is invoked, rather than imposing overhead
continuously. Its opportunity cost is specific and nameable: the die
area it occupies is area not available for, say, a larger last-level
cache or an additional CPU core, both of which would benefit all
workloads rather than only cryptographic ones. A chip vendor choosing
to include a hardware AES accelerator is making a commitment about
which workload class deserves dedicated silicon resources---a
commitment with consequences for every customer who buys the chip,
whether or not they run cryptographic workloads.

A security economist describing the same mechanism would reach for a
completely different set of concepts. The accelerator changes the cost
structure of encryption for downstream actors: by reducing the
performance penalty of strong encryption, it shifts the economic
calculus toward adoption for system integrators and application
developers who previously might have avoided encryption on
performance grounds. This is a positive externality of the hardware
investment---the chip vendor bears the area and engineering cost, but
the security benefit accrues to every party whose data is protected
by applications that now encrypt because the cost of doing so has
dropped. At the same time, the accelerator creates a subtle
principal-agent dynamic: the chip vendor who decides to include it
does not bear the consequence if applications fail to use it. The
silicon is present; whether it is invoked is a software decision made
by application developers whose incentives to use strong encryption
are shaped by factors entirely outside the chip vendor's control.
There is also a transfer effect in the supply chain: by providing
hardware acceleration, the chip vendor implicitly shifts
responsibility for cryptographic correctness toward software---the
hardware is fast, but the security property depends on correct key
management, protocol implementation, and algorithm selection, all of
which are now the application developer's problem rather than a
constraint the hardware enforces.

These are two precise, detailed, and sophisticated descriptions of the
same object. They are not in conflict. And yet they do not connect.

The architect's description has no place for ``positive externality,''
``principal-agent dynamic,'' or ``responsibility transfer.'' The
economist's description has no place for ``activation profile,''
``opportunity cost of die area,'' or ``workload-conditional benefit.''
Each tradition has developed exactly the vocabulary it needs for its
own analytical purposes, and that vocabulary stops at the boundary of
the other tradition's domain.

The consequence is that the most important questions about hardware
AES acceleration cannot be asked within either framework alone.
\emph{Why} does the vendor bear the area cost while the security
benefit accrues to end users? Because the activation profile is
conditional and the scope is module-specific---properties that are
invisible to the economics lens. \emph{What} would a different design
imply for the distribution of benefits? An always-on encryption
engine with mandatory invocation would shift benefit realization
toward the chip vendor's customers and away from application
developers' discretion---but evaluating that tradeoff requires
simultaneously reasoning about technical properties and distributional
consequences in a way neither tradition currently supports.

This is the gap the present paper addresses. It is not a gap in
either tradition's sophistication. It is a gap between them---a
missing translation layer that would allow the precise technical
characterization of mechanism properties to be connected to the
precise economic characterization of their distributional consequences.
The following sections develop that translation layer inductively,
through case studies chosen to make the gap---and its
resolution---as concrete as possible.
%=============================================================
\section{Case Study I: Speculative Execution Controls}
\label{sec:spectre}
%=============================================================

Speculative execution attacks are among the most extensively benchmarked
vulnerabilities in hardware security history. From the initial Spectre and Meltdown
disclosures in January 2018~\cite{kocher2019spectre} through the subsequent
wave of variants---Retbleed~\cite{koruyeh2022retbleed}, MDS attacks, and
others---the research community has produced hundreds of overhead measurements
with considerable precision. We know, to within a few percentage points, what
each mitigation costs on which workloads on which microarchitectures.

What the benchmarking literature does not tell us is who paid those costs. That
question requires a different vocabulary---one that does not yet exist in standard
form. This case study develops that vocabulary inductively, by first presenting what
the engineering analysis knows with precision, then what the economics analysis
observes about distribution, and finally asking what it would take to connect the two.

%-------------------------------------------------------------
\subsection{Engineering Analysis: What the Benchmarks Know}
\label{sec:spectre-engineering}
%-------------------------------------------------------------

Speculative execution is a family of performance-enhancing techniques in which a
processor executes instructions before it is certain they will be needed, discarding
the results if the speculation proves wrong. The performance benefit is substantial:
modern out-of-order processors would be dramatically slower without speculation,
as the alternative is stalling pipelines waiting for branch outcomes, memory
latencies, and data dependencies to resolve. The vulnerability, demonstrated by
Kocher et al.~\cite{kocher2019spectre} and the independent teams behind
Meltdown~\cite{lipp2018meltdown}, is that the microarchitectural side effects of
speculative execution---cache state changes, timing variations---are observable even
when the speculated instructions are ultimately discarded. An attacker can use these
side effects to read memory they should not have access to.

The mitigations that followed span a wide design space, and their costs have been
measured extensively.

\paragraph{Software mitigations: Retpoline and IBRS.}
The first deployed mitigations were software-based. Retpoline replaces indirect
branches with a return-based trampoline that defeats branch-target injection;
IBRS (Indirect Branch Restricted Speculation) restricts speculation across privilege
boundaries. Both impose recurring runtime overhead on every affected system.
Retbleed, a 2022 variant that bypasses retpoline on AMD and older Intel
microarchitectures, required additional mitigations whose overhead Wikner and
Razavi measured at 14--39\% on Linux kernel benchmarks, with a mean of
approximately 28\% on affected AMD systems~\cite{koruyeh2022retbleed}. Earlier
IBRS-based mitigations imposed overheads of 10--30\% on system-call-intensive
workloads, dropping to 2--8\% on compute-bound tasks. These are not edge-case
numbers: system-call-intensive workloads are the norm in cloud and datacenter
environments.

\paragraph{Hardware mitigations: speculation barriers and partitioning.}
Academic proposals explored hardware-level mitigations that prevent speculative
state from being observable. InvisiSpec~\cite{yan2018invisispec} proposed making
speculative loads invisible in the cache hierarchy by buffering them until they
commit; its measured overhead averaged 21\% across SPEC CPU2006 benchmarks,
compared to 74\% for STT (Speculative Taint Tracking), a more comprehensive
approach. SafeSpec proposed a similar speculation buffer and reported overheads
in the 5--20\% range depending on workload memory intensity. These hardware
proposals traded area and design complexity for better performance than pure
software mitigations, but they required new microarchitectural structures---
speculation buffers, commit-time validation logic---that consume die area and add
verification burden.

\paragraph{Microcode and firmware patches.}
Intel and AMD both released microcode updates that implemented IBRS, STIBP
(Single Thread Indirect Branch Predictors), and SSBD (Speculative Store Bypass
Disable) at the hardware level. These patches required firmware updates propagated
through OEM and platform vendor supply chains---a deployment mechanism with its
own latency, cost, and coordination burden that the performance benchmarks do not
capture at all.

\paragraph{What the benchmarks establish.}
The picture that emerges from this literature is precise along one dimension and
silent along another. We know that speculative execution mitigations impose
\emph{always-on, system-wide, recurring} overhead. The activation profile is
not conditional or workload-triggered: every system running mitigated software
pays the cost on every workload, regardless of whether it faces a realistic
threat from speculative execution attacks. The scope is system-wide: the overhead
is not isolated to security-sensitive code paths but distributed across the entire
workload. The overhead is not a one-time engineering cost but a permanent tax on
every system running the mitigation.

These properties---always-on activation, system-wide scope, recurring timing---are
not incidental. They follow directly from the nature of the vulnerability: because
speculation happens everywhere in the pipeline, and because the attacker can observe
microarchitectural state from unprivileged code, any mitigation that closes the
channel must either restrict speculation globally or instrument every speculative
operation. There is no cheap, targeted fix. The engineering analysis tells us this
with precision. What it does not tell us is who absorbs a permanent 5--28\%
performance penalty, why that particular actor bears it, or whether a different
mitigation design could have implied a different distribution.

%-------------------------------------------------------------
\subsection{Economics Analysis: Where the Costs Landed}
\label{sec:spectre-economics}
%-------------------------------------------------------------

The Spectre and Meltdown disclosure produced one of the most publicly documented
cost-distribution events in hardware security history. Because the vulnerabilities
affected virtually every major cloud provider simultaneously, and because cloud
providers must report performance characteristics to enterprise customers, the
distributional consequences were unusually visible.

\paragraph{Cloud operators absorbed the recurring overhead.}
The performance penalty from Spectre mitigations fell almost entirely on cloud
operators---Amazon, Google, Microsoft, and their counterparts---who run the
workloads affected by always-on mitigations. Google reported performance
degradation of up to 10\% on production workloads following the initial patches.
Amazon reported similar impacts, particularly on system-call-intensive workloads
common in containerized deployments. These operators had no role in the original
microarchitectural design decisions that created the vulnerability, and no seat at
the table when Intel, AMD, and ARM decided how to implement mitigations. They
were notified during a coordinated disclosure period too short to evaluate
alternatives, and then absorbed a permanent recurring cost that was not of their
choosing.

\paragraph{CPU vendors bore the engineering and reputational cost.}
Intel, as the vendor most severely affected (Meltdown was largely an Intel-specific
vulnerability~\cite{hastings2020wac}), bore substantial engineering costs:
microcode development, validation, coordinated disclosure management, and the
longer-term cost of redesigning affected microarchitectural structures in subsequent
silicon generations. Intel's stock declined approximately 3.5\% on the day of
disclosure and faced sustained reputational pressure. The in-silicon fixes that
appeared in later microarchitectures (Whiskey Lake and beyond) represent
significant design investment---area and verification costs paid upfront by the
vendor to reduce the recurring overhead borne by operators downstream.

\paragraph{OEMs and platform vendors bore the deployment cost.}
Propagating microcode and firmware patches through the supply chain required
coordination across CPU vendors, OEM platform vendors, operating system
maintainers, and cloud operators. This coordination burden---engineering time,
testing, certification, deployment orchestration---is entirely absent from the
benchmark literature. It fell on OEMs and platform vendors who were implementing
fixes for a vulnerability in a component they did not design, under a timeline
they did not control, with potential liability for systems that remained unpatched.

\paragraph{End users and enterprises bore residual and indirect costs.}
Enterprises running on-premises hardware faced a choice: apply patches and absorb
the performance penalty, or defer patching and accept continued exposure.
Organizations with performance-sensitive workloads---high-frequency trading,
scientific computing, database-intensive applications---faced real tradeoffs with
no good options. End users of cloud services experienced service degradation
or price increases as operators passed through some portion of their overhead
costs. The users who were ultimately protected by the mitigations were not,
in general, the users who paid for them.

\paragraph{What the economics analysis establishes---and does not.}
The distributional picture is clear in broad strokes: the vulnerability originated
in CPU vendor design decisions; the engineering fix cost was borne partly by CPU
vendors and partly by OEMs; the recurring operational cost landed on cloud
operators; residual risk and indirect costs fell on enterprises and end users. This
is a textbook case of cost externalization: the actors with the greatest ability
to have prevented the vulnerability---the CPU vendors---bore only a fraction of
its total lifecycle cost.

But the economics analysis, on its own, cannot explain \emph{why} this distribution
emerged from these particular mitigations rather than others. It observes that
cloud operators paid. It does not connect that observation to the always-on
activation profile and system-wide scope that made operator-borne recurring overhead
structurally inevitable given the mitigation approach chosen. It cannot ask: if
Intel had invested earlier in hardware-level speculation buffers of the InvisiSpec
variety, would the distribution have looked different? Would the upfront vendor
cost have been higher but the downstream operator cost lower? That question
requires simultaneously reasoning about technical mechanism properties and
distributional consequences---which is precisely what neither tradition alone
supports.

%-------------------------------------------------------------
\subsection{What the Synthesis Reveals}
\label{sec:spectre-synthesis}
%-------------------------------------------------------------

Placing the two analyses side by side, the gap becomes visible as something
specific and bridgeable rather than merely a vague interdisciplinary aspiration.

\paragraph{Mechanism properties determine burden pathway signatures.}
The always-on activation profile of software Spectre mitigations is not an
accident of implementation---it is a direct consequence of the threat model.
Because speculative execution is a universal processor behavior, and because
the attack channel is observable from unprivileged code at any time, there
is no cheap way to activate mitigations only when under attack. The activation
profile is always-on by necessity, which means the overhead is recurring by
necessity, which means whoever runs mitigated workloads continuously absorbs
the cost. In a cloud environment, that is the operator.

This chain of reasoning---from threat model to activation profile to scope to
bearer---is invisible in both the benchmark literature and the economics literature
taken separately. The benchmarks document the overhead without asking who runs
the workloads. The economics analysis identifies the bearer without asking why
the mechanism's properties made that bearer structurally inevitable.

\paragraph{Alternative designs imply alternative distributions.}
InvisiSpec's hardware-level approach~\cite{yan2018invisispec} would have changed
the burden pathway signature in a specific, traceable way. Its overhead (21\%
mean) is similar in magnitude to software mitigations but its bearer would have
been different: the die area and verification cost falls on the CPU vendor at
design time, as an upfront engineering cost, rather than on the operator as a
recurring runtime cost. A vendor who invested in InvisiSpec-style hardware would
have internalized more of the burden at design time and externalized less of it
to downstream operators.

This is not merely an observation about past decisions. It is a template for
prospective analysis: given a proposed mitigation, what are its activation profile,
scope, and workload interaction? Who runs affected workloads continuously? What
is the upfront vendor cost versus the recurring operator cost? These questions
can be asked---and partially answered---before deployment, if the vocabulary exists
to ask them. Currently, it does not.

\paragraph{The concepts the combined analysis requires.}
Working through this case study, we find that connecting the engineering and
economics analyses requires a small set of concepts that belong fully to neither
tradition:

\begin{itemize}
  \item \textbf{Activation profile}: whether a mechanism's overhead is always-on,
    conditional on specific operations, or event-triggered. This determines whether
    the recurring cost is universal or workload-selective.
  \item \textbf{Scope}: whether overhead is system-wide or isolated to specific
    modules or workload classes. System-wide scope means operators of any workload
    pay; module-specific scope means only operators of relevant workloads pay.
  \item \textbf{Bearer}: the organizational actor who absorbs a given cost instance.
    Distinguished from the decision-maker (who chose the mechanism) and the
    beneficiary (who receives the security property).
  \item \textbf{Timing}: whether a cost is upfront (design time), recurring
    (every operational period), or contingent (realized only upon an incident).
    The same total cost has very different distributional implications depending
    on when it is realized and by whom.
  \item \textbf{Transfer}: when the actor who bears a cost is different from
    the actor who made the design decision that incurred it, the cost has been
    transferred. Transfer can be explicit (contractual) or implicit (structural,
    as when always-on overhead is transferred to operators by the nature of
    the mechanism).
\end{itemize}

These concepts are not exotic. Engineers use some of them informally; economists
use others under different names. What is missing is their formalization as a
shared vocabulary that can be applied consistently across mechanisms and
stakeholders. The benchmark literature could, in principle, report not just overhead
magnitudes but activation profiles and scopes. Procurement processes could, in
principle, require disclosure of bearer and transfer structure alongside performance
figures. Neither currently happens, because the vocabulary to require it does not
exist in standard form.

The next case study stress-tests these concepts across a family of mechanisms
with deliberately varied burden pathway signatures, before Section~\ref{sec:vocabulary}
crystallizes them into a reusable schema.
%=============================================================
\section{Case Study II: Memory Safety as Comparative Stress Test}
\label{sec:memsafety}
%=============================================================

Memory safety is one of the oldest and most persistent problems in systems
security. Roughly 70\% of critical vulnerabilities in large codebases such as
Chrome and the Windows kernel have been attributed to memory safety
failures~\cite{miller2019msrc,chromesecurity2020}. The technical community has
responded with a wide range of mechanisms---spanning hardware tagging, capability
architectures, compiler instrumentation, and control flow enforcement---that
address the same underlying threat but produce dramatically different burden
pathway signatures.

This is what makes memory safety an ideal stress test for the vocabulary developed
in Section~\ref{sec:spectre}. Unlike the Spectre case, where a single vulnerability
class produced a relatively constrained set of mitigation approaches, memory safety
offers a natural comparative experiment: four technically distinct mechanisms, each
addressing the same objective, each with a different answer to the question of who
pays, when, and how much. If the vocabulary can account for all four and reveal
the distributional logic behind their differences, it is doing real analytical work.

We analyze each mechanism in turn, then draw the comparative conclusions that
the individual analyses make possible.

%-------------------------------------------------------------
\subsection{Arm Memory Tagging Extension (MTE)}
\label{sec:mte}
%-------------------------------------------------------------

\paragraph{What it is.}
MTE, introduced in the Armv8.5-A architecture, adds a 4-bit color tag to each
16-byte granule of memory and a corresponding tag to each pointer. On every
memory access, hardware checks that the pointer tag matches the allocation tag;
a mismatch raises a fault. This provides deterministic detection of heap and stack
buffer overflows and use-after-free errors with no false positives, at the cost of
additional tag storage and per-access check overhead.

\paragraph{Engineering analysis.}
The hardware cost of MTE falls on the chip vendor. Tag storage requires
approximately 3\% additional DRAM capacity (one tag byte per 16 data bytes) and
modest on-die tag cache structures. The memory bus must carry tag bits alongside
data, increasing interconnect complexity. Arm's own guidance reports runtime
overhead of approximately 1--5\% for tag-checking in typical workloads, with
memory overhead of roughly 3\%~\cite{arm2021mte}. The activation profile is
conditional: MTE can be enabled or disabled per memory region, and checking
overhead is incurred only on tagged allocations. Scope is therefore
workload-selective rather than system-wide---applications that opt into MTE pay
the overhead; those that do not are unaffected.

The verification burden at the chip vendor is non-trivial: MTE requires correctness
guarantees about tag propagation across the memory hierarchy, interaction with
cache coherence protocols, and behavior under speculative execution. These are
novel design constraints that add engineering labor and validation cost at tape-out.

\paragraph{Burden pathway.}
The vendor bears the upfront hardware cost: die area, interconnect complexity,
verification effort. The integrator---OEM or platform vendor---bears the firmware
and software stack cost of enabling and configuring MTE. The operator or developer
bears a conditional recurring overhead only if they opt into MTE for their
allocations. End users bear no direct cost but benefit from reduced vulnerability
exposure in MTE-enabled applications.

Crucially, the \emph{decision} to use MTE is made at multiple points in the
stack. The chip vendor decides to include the hardware. The OS vendor decides
whether to support it in the allocator. The application developer decides whether
to enable it for their heap. The security benefit accrues only when all three
decisions align---a coordination problem that the hardware overhead figures do
not capture.

%-------------------------------------------------------------
\subsection{CHERI Capability Architecture}
\label{sec:cheri}
%-------------------------------------------------------------

\paragraph{What it is.}
CHERI~\cite{woodruff2014cheri} replaces conventional pointers with
hardware-enforced capabilities: fat pointers that carry bounds, permissions, and
a validity tag checked on every dereference. A CHERI processor cannot be made
to dereference an out-of-bounds or revoked pointer; the hardware enforces the
constraint unconditionally. This provides substantially stronger memory safety
guarantees than MTE---spatial and temporal safety for all pointer operations,
not just allocation boundaries---but at a fundamentally different cost structure.

\paragraph{Engineering analysis.}
CHERI requires a complete redesign of the ISA: every pointer becomes a 128-bit
capability (double the width of a 64-bit address), every pointer operation requires
capability manipulation instructions, and the memory system must track and check
validity tags pervasively. The hardware overhead is significant: approximately
2$\times$ memory bandwidth for capability-wide pointer operations in naive
implementations, with optimized implementations achieving closer to 10--40\%
overhead on memory-intensive workloads. But the hardware cost is, in a meaningful
sense, the smaller part of the burden.

The dominant cost of CHERI is \emph{ecosystem migration}. Every piece of software
that manipulates pointers---which is essentially all systems software---must be
ported to the CHERI ABI. Existing code that plays tricks with pointer representations,
stores metadata in unused pointer bits, or relies on pointer arithmetic that
exceeds allocation bounds will not run correctly on CHERI without modification.
The CheriBSD project, which ports FreeBSD to CHERI, has invested years of
engineering effort and remains incomplete for the full software ecosystem. CHERI's
own evaluation literature acknowledges migration costs measured in engineer-years
for non-trivial codebases~\cite{woodruff2014cheri}.

\paragraph{Burden pathway.}
The hardware vendor bears substantial upfront design cost: ISA redesign,
microarchitectural changes throughout the pipeline, verification of capability
semantics. But this upfront cost is dwarfed, in aggregate, by the integration
burden it imposes on the software ecosystem. Every operating system vendor,
every systems library maintainer, every application developer who wants to run
on CHERI must invest in migration. The burden falls on integrators and developers
in proportion to their software's complexity and its reliance on low-level pointer
manipulation---precisely the code that is most security-critical and most
expensive to modify.

The timing structure is unusual: the upfront vendor cost is large but bounded;
the ecosystem migration cost is diffuse, long-tail, and potentially unbounded,
accumulating across thousands of independent software projects over years or
decades. A benchmark measuring CHERI's hardware overhead captures perhaps
10\% of the total burden the mechanism imposes on the ecosystem.

The beneficiary structure is also unusual: the strongest safety guarantees accrue
only when the \emph{entire} software stack is ported to CHERI. Partial adoption
produces partial guarantees. This means the security benefit is non-linear in
adoption, while the migration cost is roughly linear---a distributional structure
that creates strong incentives for free-riding and late adoption.

%-------------------------------------------------------------
\subsection{Compiler-Based Mitigations: AddressSanitizer and SafeStack}
\label{sec:asan}
%-------------------------------------------------------------

\paragraph{What they are.}
AddressSanitizer (ASan) instruments memory operations at compile time to detect
out-of-bounds accesses and use-after-free errors via shadow memory. SafeStack
separates the stack into safe (return addresses and function pointers) and unsafe
(everything else) regions to prevent stack-based control flow hijacking. Both
are compiler passes that require no hardware changes; their overhead is paid
entirely in software.

\paragraph{Engineering analysis.}
ASan imposes approximately 2$\times$ memory overhead (shadow memory) and
50--100\% runtime overhead in instrumented builds~\cite{serebany2012asan}---costs
that make it suitable for testing and debugging but not production deployment at
scale. SafeStack's overhead is much lower, typically 1--3\%, because it only
instruments stack operations. Neither mechanism requires any hardware investment;
the entire cost is borne at compile time (toolchain integration) and runtime
(instrumented execution).

The activation profile is entirely developer-controlled: instrumentation is a
compiler flag. This means the mechanism is not deployed unless a developer or
operator explicitly enables it, and its protection is absent in any build that
does not include the flag. There is no system-wide enforcement; protection is
opt-in and therefore uneven across the software ecosystem.

\paragraph{Burden pathway.}
The hardware vendor bears zero cost. The chip vendor who ships a processor without
MTE or CHERI can point to ASan as evidence that memory safety tooling exists,
while bearing none of its cost. The toolchain maintainer (LLVM, GCC) bears the
engineering cost of implementing and maintaining the instrumentation pass. The
developer bears the integration cost of enabling and maintaining instrumented
builds, and the operator bears the runtime overhead of any production deployment.

The distributional consequence of opt-in deployment deserves emphasis. Because
ASan is not mandatory, its benefits are realized only in software whose developers
chose to enable it, maintained the instrumented build pipeline, and accepted the
overhead. Security-critical infrastructure maintained by under-resourced teams---
precisely the software most likely to harbor exploitable memory errors---is the
least likely to have ASan enabled in production. The mechanism design implicitly
transfers the cost of memory safety to those most willing to pay it, which is not
the same as those most in need of it.

%-------------------------------------------------------------
\subsection{Hardware Control Flow Integrity: Arm BTI and Intel CET}
\label{sec:cfi}
%-------------------------------------------------------------

\paragraph{What they are.}
Arm Branch Target Identification (BTI) and Intel Control-flow Enforcement
Technology (CET) are hardware mechanisms that restrict indirect branch targets
to explicitly marked locations, preventing attackers from redirecting control flow
to arbitrary code gadgets (return-oriented programming and jump-oriented
programming attacks). Both are ISA extensions that require hardware support,
OS integration, and compiler toolchain updates to be effective.

\paragraph{Engineering analysis.}
The hardware cost of BTI and CET is modest relative to CHERI: both require
small additions to the instruction decode and branch prediction units to check
and enforce landing pad constraints, plus shadow stack hardware in the case of
CET's return address protection. Intel reports CET overhead of 1--4\% on
typical workloads; Arm reports similar figures for BTI. The activation profile
is conditional: protection applies only to code compiled with BTI/CET support
and running on an OS that enforces the constraints. Legacy binaries without
BTI/CET annotations run without protection, creating a mixed-trust environment
during the transition period.

\paragraph{Burden pathway.}
The hardware vendor bears a modest upfront cost---ISA extension design,
microarchitectural changes, verification. The OS vendor bears integration cost:
kernel changes to enable enforcement, shadow stack management, compatibility
handling for legacy binaries. The compiler toolchain maintainer bears the cost
of annotating generated code with landing pad markers. The developer bears the
cost of rebuilding with updated compilers and testing for compatibility regressions.

The transition burden deserves particular attention. During the period between
hardware availability and ecosystem-wide adoption---which, for CET, has stretched
across multiple years since its introduction in Tiger Lake processors---the
protection is partial. An attacker who can load a legacy binary into a
CET-enabled process can still use that binary's gadgets for return-oriented
programming. The overhead of the transition period falls on operators who must
manage mixed-trust environments, and on security teams who must reason about
partial protection guarantees that are harder to analyze than either full
protection or no protection.

%-------------------------------------------------------------
\subsection{Comparative Analysis: Same Objective, Different Burden Signatures}
\label{sec:memsafety-comparison}
%-------------------------------------------------------------

Table~\ref{tab:memsafety} summarizes the burden pathway signatures of the four
mechanisms using the vocabulary developed in Section~\ref{sec:spectre}.

\begin{table*}[t]
\centering
\caption{Memory safety mechanisms compared under burden-allocation vocabulary.
Bearer abbreviations: V = chip Vendor, I = Integrator/OEM, D = Developer,
O = Operator, U = end User.}
\label{tab:memsafety}
\small
\begin{tabular}{lllllll}
\toprule
Mechanism & Activation & Scope & Primary bearer & Upfront cost & Recurring cost & Transfer pattern \\
\midrule
Arm MTE & Conditional & Workload-selective & V (hardware), D (adoption) & Medium & Low--Medium & Vendor $\to$ developer (opt-in) \\
CHERI & Always-on (if adopted) & System-wide & V + ecosystem & Very high & Medium & Vendor $\to$ ecosystem (migration) \\
ASan / SafeStack & Opt-in & Build-selective & D, O & Low & High (ASan) / Low (SafeStack) & Toolchain $\to$ developer \\
Arm BTI / Intel CET & Conditional & Binary-selective & V, I, D & Medium & Very low & Vendor $\to$ integrator (transition) \\
\bottomrule
\end{tabular}
\end{table*}

The comparison reveals several patterns that neither the benchmark literature
nor the economics literature surfaces on its own.

\paragraph{Activation profile determines who pays the recurring cost.}
MTE and BTI/CET are conditional: overhead is incurred only on tagged or annotated
code paths. ASan is opt-in: overhead is incurred only in instrumented builds.
CHERI, once adopted, is always-on: the capability-wide pointer representation
is universal. This means CHERI's recurring cost is borne by every operator of
every workload on a CHERI system, while ASan's recurring cost is borne only by
developers and operators who choose instrumented builds. Same security objective;
completely different recurring cost distribution.

\paragraph{Scope determines the breadth of the burden.}
ASan's protection is build-selective: a process either runs with ASan or without
it. BTI/CET's protection is binary-selective: individual binaries either have
landing pad annotations or do not. MTE's protection is allocation-selective:
individual memory regions either have tags or do not. CHERI's protection is
system-wide: all pointer operations on a CHERI system are capability-checked.
Broader scope means stronger guarantees but also broader cost distribution ---
CHERI's stronger protection comes with costs distributed across the entire
software ecosystem, not just security-conscious developers.

\paragraph{The upfront/recurring split reflects who made the design decision.}
CHERI concentrates enormous upfront cost on the hardware vendor and the
migration cost on the ecosystem. The hardware vendor's decision to pursue CHERI
creates a liability for every software maintainer whose code must be ported.
This is a large, implicit, unilateral transfer: the vendor chooses the architecture,
and the ecosystem bears the migration burden. MTE concentrates upfront cost
on the vendor but transfers the recurring adoption decision to developers, who
can opt in or not. ASan transfers the upfront cost away from hardware entirely,
placing it on toolchain maintainers and developers. BTI/CET distributes upfront
cost across vendor, OS, toolchain, and developer in a staged transition.

\paragraph{Opt-in mechanisms transfer burden to those willing to pay, not
those most in need.}
This is perhaps the sharpest distributional finding of the comparison. Both
ASan and MTE (in its opt-in deployment model) transfer the decision to adopt
memory safety to developers and operators. This means the burden of memory
safety is borne by those who choose to bear it---which correlates with
security-consciousness, resources, and sophistication, not with the actual
distribution of vulnerability or risk. The most security-critical but
under-resourced software in the ecosystem is least likely to run with ASan
enabled or MTE opted in. The mechanism design inadvertently produces an
inequitable distribution of protection.

\paragraph{What the benchmarks miss.}
The overhead figures for these four mechanisms---1--5\% for MTE, 10--40\% for
CHERI in memory-intensive workloads, 50--100\% for ASan, 1--4\% for BTI/CET---
are real and important. But they are silent on everything that matters for
understanding how security burden is actually distributed: who bears each
component of the cost, what the opt-in and coordination dynamics are, how the
transition period creates partial-protection regimes that are harder to reason
about than either full protection or none, and how the upfront/recurring split
interacts with the organizational boundary between chip vendor, OS vendor,
toolchain maintainer, developer, and operator.

A procurement decision based solely on benchmark overhead would treat MTE
and BTI/CET as roughly equivalent (both 1--5\% overhead) and both as vastly
preferable to CHERI (10--40\%) or ASan (50--100\%). But under burden-allocation
analysis, the four mechanisms have radically different total cost structures,
different distributions of who bears each component, and different implications
for ecosystem-wide security coverage. The benchmark-only view does not just
undercount the cost---it systematically misattributes it.

\paragraph{Vocabulary confirmed and extended.}
The five concepts identified in the Spectre case study---activation profile,
scope, bearer, timing, and transfer---account for the observed variation across
all four memory safety mechanisms. The comparative analysis adds one refinement:
the distinction between \emph{bilateral transfer} (where a specific downstream
actor is identifiably burdened by an upstream design decision, as in CHERI's
ecosystem migration) and \emph{diffuse transfer} (where the burden falls on
whoever chooses to adopt, as in ASan's opt-in deployment). This distinction
matters for accountability: bilateral transfer creates a traceable liability
pathway, while diffuse transfer disperses accountability across the ecosystem
in a way that is harder to contest or correct.

The next section crystallizes these concepts into a reusable schema.

%=============================================================
\section{A Burden-Allocation Vocabulary}
\label{sec:vocabulary}
%=============================================================

The two case studies were not chosen to illustrate a pre-built framework.
They were chosen to reveal, inductively, what concepts are necessary to connect
the engineering and economics analyses of the same mechanisms. By the end of
Section~\ref{sec:spectre}, five concepts had emerged as load-bearing: activation
profile, scope, bearer, timing, and transfer. By the end of
Section~\ref{sec:memsafety}, those five had proven sufficient to account for
four mechanisms with sharply different burden pathway signatures, with one
refinement: the bilateral/diffuse distinction within transfer.

This section crystallizes those concepts into a reusable schema. The vocabulary
is not a translation between two languages that happen to use different words for
the same things. It fills gaps that neither tradition currently has words for at
all. The engineering tradition has precise terms for mechanism costs but no terms
for where those costs go. The economics tradition has precise terms for who bears
costs but no terms for why specific technical properties produce specific
distributions. The vocabulary provides the missing concepts in both directions
simultaneously --- which is what makes it a synthesis rather than a glossary.

We present it as a structured set of entity types, typed relations, and
constraints: precise enough to support consistent analysis across mechanisms and
stakeholders, light enough that practitioners in either tradition can apply it
without specialist ontology tooling. Those who wish to encode it formally in
OWL 2 DL will find the mapping straightforward; that encoding is a companion
artifact rather than a burden on the main text.

%-------------------------------------------------------------
\subsection{The Missing Slots}
\label{sec:vocab-motivation}
%-------------------------------------------------------------

The fog of war around hardware security costs persists not primarily because
the two traditions use different words, but because they report different
\emph{fields}. A benchmark paper fills in: mechanism name, workload, overhead
magnitude, hardware platform. It does not fill in: who bears the overhead,
under what organizational conditions, for how long, and as a consequence of
whose decision. An economics analysis fills in: which actor ended up paying,
what incentive misalignment explains this. It does not fill in: which specific
technical properties of the mechanism made this distribution structurally
likely, and whether an alternative design would have implied a different one.

Neither set of missing fields is exotic. Both are answerable given the right
vocabulary. The reason they go unfilled is that no shared schema defines them
as required --- so they are omitted not because they are unknown but because
there is no standard slot to put them in.

A vocabulary, in this sense, is a commitment about what information is
relevant and what omissions are meaningful. A schema that requires every cost
entry to name a bearer, a decision-maker, a lifecycle phase, and a transfer
structure makes it impossible to report overhead without answering the
distributional question. The fog of war is cleared not by more benchmarks but
by changing what any analysis is required to say.

%-------------------------------------------------------------
\subsection{Entity Types}
\label{sec:entities}
%-------------------------------------------------------------

The vocabulary requires four entity types.

\paragraph{Security Mechanism.}
A technical control implemented in hardware, firmware, or system software that
reduces the exploitability of a threat or the impact of a successful attack.
Mechanisms are the primary unit of analysis. Each mechanism carries two
engineering properties that the case studies showed to be analytically
essential: \textbf{activation profile} (always-on, conditional, or opt-in)
and \textbf{scope} (system-wide, module-specific, or workload-selective).
These are not incidental implementation details --- they are the engineering
inputs that determine burden pathway signatures, as the Spectre and memory
safety case studies demonstrated. A mechanism description that omits them
cannot support burden-allocation analysis.

\paragraph{Cost Instance.}
A specific, attributable reduction in objective value caused by adopting,
operating, or omitting a security mechanism. Cost instances are the atomic
units of burden-allocation analysis. A mechanism may have many cost
instances --- one for each combination of cost type, bearer, and lifecycle
phase. The central design commitment of this entity type is that a cost
instance without a named bearer is \emph{incomplete}: it reports an overhead
without answering who absorbs it. This is the field the benchmark literature
systematically omits.

\paragraph{Stakeholder.}
An organizational actor who bears, decides, or benefits from a cost instance.
The vocabulary separates four roles that are frequently conflated but analytically
distinct:
\begin{itemize}
  \item \textbf{Decision-maker}: the actor whose design or deployment choice
    incurred the cost.
  \item \textbf{Bearer}: the actor who absorbs the cost in practice.
  \item \textbf{Beneficiary}: the actor who receives the security property the
    mechanism provides.
  \item \textbf{Residual risk holder}: the actor who bears the consequences if
    the mechanism fails or is bypassed.
\end{itemize}

Separating these roles is where the vocabulary does its most important work.
In the Spectre case, the decision-maker (CPU vendor), the bearer (cloud
operator), the beneficiary (end user), and the residual risk holder (enterprise
customer) are four different parties. The economics tradition observes the
bearer; the engineering tradition models the decision-maker's constraints.
Neither tradition has a standard way to represent all four roles simultaneously
and ask what happens when they diverge. The vocabulary does.

Concrete stakeholder classes in hardware security include: chip vendor, OEM
or platform integrator, cloud or datacenter operator, enterprise customer,
end user or public, standards body or regulator, and software ecosystem
(toolchain maintainers, OS vendors, application developers considered
collectively).

\paragraph{Lifecycle Phase.}
The point at which a cost is realized: \textbf{design} (during hardware or
software development); \textbf{integration} (when incorporating the mechanism
into a platform); \textbf{operation} (recurring costs during normal use); and
\textbf{incident} (contingent costs upon a security failure). Lifecycle phase
interacts with bearer in ways the benchmark literature cannot capture: the same
mechanism may have its design costs borne by the vendor, its integration costs
by the OEM, its operational costs by the operator, and its incident costs by
the enterprise customer --- four different actors across four phases, none of
whom appears in a standard overhead figure.

%-------------------------------------------------------------
\subsection{Cost Taxonomy}
\label{sec:cost-taxonomy}
%-------------------------------------------------------------

Cost instances are typed across five categories. The taxonomy is motivated
by the case studies: each category corresponds to a class of costs that
appeared in the analysis and that existing reporting frameworks handle poorly
or not at all.

\textbf{Physical resource costs}: die area, power, memory bandwidth, latency.
These are what the benchmark literature measures most precisely and what the
economics literature treats as a black box.

\textbf{Engineering process costs}: labor and time in design, verification,
toolchain development, or ecosystem migration. CHERI's dominant cost is in
this category; it is entirely absent from hardware performance benchmarks.

\textbf{Operational costs}: recurring costs during system use --- runtime
performance overhead, patch deployment and maintenance, key management,
monitoring infrastructure. Spectre mitigation costs fall here; the benchmark
literature measures their magnitude but not their bearer.

\textbf{Compliance and governance costs}: audit effort, certification cycles,
documentation obligations, contractual liability. Largely invisible in the
technical literature; significant in real procurement decisions.

\textbf{Strategic and externality costs}: opportunity costs, reputational
exposure, market effects, and costs imposed on parties outside the direct
decision chain. The opportunity cost of die area committed to MTE rather than
cache capacity falls here, as does Intel's reputational exposure from Meltdown.
These costs are the most consequential for long-run ecosystem dynamics and the
least represented in either tradition's standard analysis.

%-------------------------------------------------------------
\subsection{Core Relations}
\label{sec:relations}
%-------------------------------------------------------------

Relations are where the vocabulary does its analytical work --- they are the
typed links that make burden pathways traceable rather than merely nameable.
Six relations are required.

\medskip
\noindent\textbf{incurs}(\textit{mechanism}, \textit{cost instance}): a
mechanism gives rise to a cost. Every cost must be traceable to the mechanism
that incurred it.

\medskip
\noindent\textbf{borne\_by}(\textit{cost instance}, \textit{stakeholder}):
a cost is absorbed by a stakeholder in the bearer role. This is the relation
most consistently absent from existing technical evaluations. A cost instance
without a \textbf{borne\_by} relation is, by the vocabulary's constraints,
an incomplete analysis.

\medskip
\noindent\textbf{decided\_by}(\textit{cost instance}, \textit{stakeholder}):
the decision that incurred this cost was made by this stakeholder. When
\textbf{decided\_by} $\neq$ \textbf{borne\_by}, a transfer has occurred.
This relation is what the economics tradition tracks under the heading of
principal-agent misalignment; the vocabulary makes it explicit and
mechanism-specific rather than a general market observation.

\medskip
\noindent\textbf{benefits}(\textit{cost instance}, \textit{stakeholder}):
the security property purchased by this cost accrues to this stakeholder.
When \textbf{benefits} $\neq$ \textbf{borne\_by}, the mechanism produces a
cross-subsidy: one actor pays for another's protection. This is the relation
that makes the opt-in deployment problem visible: ASan's diffuse transfer
means the actors who pay for memory safety are not the actors most in need
of it.

\medskip
\noindent\textbf{realized\_at}(\textit{cost instance}, \textit{lifecycle phase}):
the cost is realized at this phase. Combined with \textbf{borne\_by}, this
identifies not just who pays but when --- essential for comparing mechanisms
whose costs have the same magnitude but different timing structures.

\medskip
\noindent\textbf{transferred\_to}(\textit{cost instance}, \textit{stakeholder}):
used when \textbf{decided\_by} $\neq$ \textbf{borne\_by} to name the transfer
target explicitly and annotate its type:
\begin{itemize}
  \item \textbf{Bilateral transfer}: the transfer target is a specific,
    identifiable actor whose burden is traceable to the originating decision.
    CHERI's ecosystem migration cost is bilateral: the hardware vendor's
    architectural decision created an identifiable liability for software
    maintainers.
  \item \textbf{Diffuse transfer}: the burden falls on whoever opts into the
    mechanism, dispersing accountability across the ecosystem with no traceable
    path back to the originating decision. ASan's opt-in deployment is diffuse:
    no specific actor is responsible for whether any given software project runs
    with memory safety instrumentation.
\end{itemize}

This distinction matters beyond taxonomy. Bilateral transfer creates a
contestable liability pathway --- an actor who bears a cost can identify who
created it and potentially seek redress or renegotiation. Diffuse transfer
dissolves accountability: the burden is real but unattributable, which is
precisely the condition under which it tends to fall on actors with the least
capacity to resist it.

%-------------------------------------------------------------
\subsection{Constraints}
\label{sec:constraints}
%-------------------------------------------------------------

A vocabulary without constraints is a glossary. The constraints define what
a \emph{complete} burden-allocation analysis looks like --- and therefore
make omissions visible as omissions rather than as absences.

\begin{enumerate}
  \item Every cost instance must have at least one \textbf{borne\_by} relation.
    A cost without a named bearer is an incomplete analysis, not a cost-free
    mechanism.
  \item Every cost instance must have a \textbf{realized\_at} lifecycle phase.
    Undated costs cannot be compared across mechanisms with different timing
    structures.
  \item Every cost instance must have a \textbf{decided\_by} relation. This
    forces identification of who made the choice that incurred the cost ---
    the precondition for asking whether decision and burden are aligned.
  \item When \textbf{decided\_by} $\neq$ \textbf{borne\_by}, a
    \textbf{transferred\_to} relation must be present with transfer type
    annotated. Transfer is not an exceptional observation; it is a structural
    feature to be reported when present.
  \item Every mechanism must declare an activation profile and a scope. These
    are the engineering inputs that determine burden pathway signatures and
    must be reported alongside overhead magnitudes.
\end{enumerate}

The constraints are intentionally minimal. They do not require precise cost
quantification --- many hardware security costs are genuinely hard to measure,
and demanding quantification would make the schema unusable. What they require
is \emph{attribution}: every cost connected to a bearer, a decision-maker, a
timeline, and a transfer structure. Attribution without magnitude is vastly
more useful than magnitude without attribution, because attribution is what
makes the distributional question answerable at all.

%-------------------------------------------------------------
\subsection{What the Vocabulary Exposes That Neither Tradition Could}
\label{sec:vocab-exposes}
%-------------------------------------------------------------

It is worth pausing to be explicit about what the vocabulary adds beyond
what either tradition provides on its own --- since the case for a synthesis
stands or falls on this question.

The engineering tradition, applied alone to Spectre mitigations, tells you
that the overhead is 5--28\%, always-on, system-wide. It cannot tell you who
bears that overhead in a multi-tenant cloud environment, why the decision-maker
(CPU vendor) and the bearer (cloud operator) are different parties, or whether
an alternative mitigation design would have produced a different distribution.
Those questions are outside its scope by design.

The economics tradition, applied alone to Spectre, tells you that cloud
operators bore costs created by CPU vendor decisions, that this reflects
principal-agent misalignment, and that information asymmetry delayed the
market response. It cannot tell you why the always-on activation profile
and system-wide scope made operator-borne recurring overhead structurally
inevitable given the chosen mitigation approach, or what specific engineering
alternatives would have changed the distribution. Those answers require
opening the black box.

The vocabulary, applied to Spectre, connects these: the always-on activation
profile (engineering concept, now a required schema field) is the causal
mechanism that produces the recurring externality (economics concept) borne
by operators (schema field: bearer). The chain from design choice to
distributional outcome is now explicit and traversable. And crucially, the
chain runs in both directions: prospectively, a designer can ask what activation
profile a proposed mitigation implies and therefore who will bear its recurring
cost; retrospectively, an analyst can ask why the costs landed where they did
and trace the answer to specific technical decisions.

This bidirectionality --- prospective as well as retrospective analysis --- is
what neither tradition alone supports and what the vocabulary makes possible.
It is also what makes the vocabulary a practical tool rather than a post-hoc
description: the burden pathway can be analyzed before deployment, not only
after.

The following section demonstrates this in a worked example, then draws out
the broader implications.
%=============================================================
\section{What the Synthesis Reveals}
\label{sec:synthesis}
%=============================================================

The vocabulary developed in Section~\ref{sec:vocabulary} is only useful if it
reveals things that existing analyses miss. This section demonstrates that it
does, in three ways. First, two worked examples --- secure boot and hardware
random number generation --- show the vocabulary applied to mechanisms not
previously analyzed in this paper, producing burden-pathway analyses that
neither the engineering nor the economics tradition would generate alone.
Second, we draw out the general claim that the case studies and examples
support: mechanism properties determine burden distributions prospectively,
not merely retrospectively. Third, we argue that allocation opacity is not
only a consequence of disciplinary fragmentation --- it is sometimes actively
maintained, because it is convenient for certain actors, and the vocabulary
makes that convenience visible.

%-------------------------------------------------------------
\subsection{Worked Example 1: Secure Boot and Firmware Integrity}
\label{sec:secureboot}
%-------------------------------------------------------------

\paragraph{What it is.}
Secure boot is a chain-of-trust mechanism that verifies the cryptographic
signature of each stage of the boot process before executing it: firmware
verifies the bootloader, the bootloader verifies the OS kernel, and so on.
NIST SP 800-193~\cite{cooper2018sp800193} defines a related set of platform
firmware resiliency requirements that extend this to detection and recovery.
Secure boot is mandated or strongly recommended in a growing number of
regulatory and procurement contexts, including US federal procurement
requirements and major cloud provider security baselines.

\paragraph{Engineering analysis.}
The hardware cost of secure boot falls on the chip vendor: a root of trust
requires a hardware security module or equivalent trusted execution anchor,
cryptographic acceleration for signature verification, and protected storage
for root keys. These are non-trivial design investments but well-characterized
in the literature. Boot time increases by tens to hundreds of milliseconds
depending on the depth of the verification chain and the size of the firmware
images being verified. The activation profile is always-on in a meaningful
sense: verification occurs at every boot, though not during normal operation.
Scope is system-wide: a failure in the chain of trust affects the entire
platform, not a specific workload.

What the engineering analysis does not capture is the key provisioning and
lifecycle management infrastructure that secure boot requires to function. A
hardware root of trust is useless without a certificate hierarchy, a key
management process, and a revocation mechanism. These are not hardware costs ---
they are organizational and operational costs that the vendor's design decision
implicitly creates for downstream actors.

\paragraph{Burden pathway analysis.}
Applying the vocabulary:

\begin{itemize}
  \item \textbf{Chip vendor} (decision-maker, bearer of design cost):
    incurs physical resource costs (silicon area for the hardware security
    module, cryptographic engine) and engineering process costs (verification
    of the root of trust, certification). These are upfront costs realized
    at the design phase.

  \item \textbf{OEM or platform integrator} (bearer of integration cost):
    must provision keys into the hardware root of trust at manufacturing time,
    maintain the certificate hierarchy across product generations, implement
    firmware update mechanisms that preserve chain-of-trust integrity, and
    manage key revocation when vulnerabilities are discovered. These are
    engineering process and operational costs realized at the integration
    and operation phases. Critically, these costs are \emph{bilaterally
    transferred} from the chip vendor's design decision to the OEM: the
    vendor's choice to include a hardware root of trust creates a mandatory
    key management obligation for every OEM who integrates the chip.

  \item \textbf{Enterprise customer} (bearer of operational and compliance cost):
    must manage device enrollment, certificate lifecycle, and boot policy
    configuration across a potentially large fleet. In regulated industries,
    must demonstrate compliance with firmware integrity requirements to auditors.
    These operational and compliance costs are realized at the operation phase
    and are transferred bilaterally from the regulatory mandate (Authority as
    decision-maker) to the enterprise as bearer.

  \item \textbf{End user} (beneficiary, residual risk holder): receives the
    security property --- protection against persistent firmware compromise ---
    but has no visibility into whether secure boot is correctly implemented
    or whether the key management infrastructure is sound. The beneficiary
    and the residual risk holder are the same party, but neither the decision-
    maker nor the bearer.
\end{itemize}

\paragraph{What the synthesis reveals.}
The secure boot burden pathway is a cascade of bilateral transfers: the chip
vendor's design decision transfers key management obligations to OEMs; the
regulatory mandate transfers compliance obligations to enterprises; and the
security benefit is received by end users who bear none of the cost and have
no mechanism to verify the quality of what they are receiving.

This cascade is invisible to both traditions taken separately. The engineering
literature characterizes the hardware root of trust and boot-time verification
overhead. The economics literature notes that regulatory mandates can shift
costs downstream. But neither connects the specific technical architecture of
secure boot --- the fact that a hardware root of trust requires an external
key provisioning infrastructure that the chip vendor cannot supply unilaterally
--- to the specific organizational actors who end up bearing the provisioning
and lifecycle management burden. The vocabulary makes that connection explicit
and traceable.

A procurement officer evaluating two secure boot implementations with similar
hardware overhead figures would, under local-overhead analysis, treat them as
equivalent. Under burden-allocation analysis, the question becomes: what key
management obligations does each implementation transfer to the buyer, and what
does lifecycle management cost over the platform's operational life? Those
questions can be asked prospectively, before purchase, if the schema fields
exist to require them.

%-------------------------------------------------------------
\subsection{Worked Example 2: Hardware Random Number Generation}
\label{sec:hrng}
%-------------------------------------------------------------

\paragraph{What it is.}
A hardware random number generator (HRNG) uses physical entropy sources ---
thermal noise, shot noise, or other quantum phenomena --- to produce
cryptographically unpredictable random bits. HRNGs are present in most modern
processors (Intel's RdRand/RdSeed, AMD's equivalent, Arm's TRNG) and are
used as entropy sources for cryptographic key generation, nonce production,
and randomness-dependent protocols throughout the software stack.

\paragraph{Engineering analysis.}
The hardware cost of an HRNG falls entirely on the chip vendor: entropy source
design, conditioning circuitry, health testing logic, and the interface to the
instruction set. These are modest in area terms but non-trivial in verification
terms: the statistical properties of the entropy source must be validated, and
the conditioning pipeline must be shown not to introduce bias or predictability.
The activation profile is conditional: the HRNG is invoked only when software
requests random bits, imposing no always-on overhead. Scope is narrow: the
HRNG affects only workloads that request random numbers.

\paragraph{Burden pathway analysis.}
Applying the vocabulary:

\begin{itemize}
  \item \textbf{Chip vendor} (decision-maker, bearer of design cost): incurs
    physical resource and engineering process costs at the design phase.
    The vendor also bears a significant but often unacknowledged
    \emph{strategic cost}: the security of every cryptographic operation
    performed by every piece of software on the platform depends on the
    quality of the entropy source. If the entropy source is compromised ---
    whether by design flaw, implementation error, or deliberate backdoor ---
    the damage propagates silently through the entire cryptographic ecosystem.
    The vendor internalizes the design cost but externalizes this systemic
    risk to every downstream actor.

  \item \textbf{Developer} (bearer of integration cost, decision-maker for
    usage): must correctly invoke the HRNG, verify that it is available on
    the target platform, implement fallback paths for platforms where it is
    absent or fails health testing, and avoid common misuse patterns (such
    as seeding a software PRNG from a single RdRand call without appropriate
    conditioning). These are engineering process costs realized at the
    integration phase. The developer is both the bearer of the integration
    cost and a secondary decision-maker: the chip vendor provides the
    hardware, but whether it is used correctly is entirely the developer's
    decision.

  \item \textbf{End user} (beneficiary, residual risk holder): the security
    property --- cryptographic unpredictability --- accrues entirely to the
    end user, who has no ability to verify the quality of the entropy source,
    no visibility into whether the developer used it correctly, and no
    recourse if the randomness is weak or compromised. The beneficiary
    and residual risk holder are the same party, as in the secure boot case,
    but the distance between them and any decision-maker is even greater.
\end{itemize}

\paragraph{What the synthesis reveals.}
The HRNG burden pathway exposes an accountability gap that is structurally
different from the Spectre or memory safety cases. In those cases, the burden
pathway was visible --- cloud operators knew they were absorbing Spectre
mitigation overhead, and CHERI adopters knew they were paying migration costs.
In the HRNG case, the most consequential cost is \emph{invisible to its bearer}.
End users who depend on cryptographic operations for the security of their data,
communications, and transactions have no mechanism to verify the quality of
the entropy that underlies those operations. The residual risk holder cannot
observe the risk they hold.

This is not merely an information asymmetry in the Anderson sense --- a
situation where one party knows more than another. It is a structural feature
of the mechanism's design: the security property is realized deep in hardware,
the developer interface abstracts away the entropy source, and the user
interface abstracts away the developer's choices. Each layer of abstraction
is individually sensible; together they produce a situation where the party
with the most to lose has the least ability to evaluate what they are receiving.

The Intel RdRand controversy of 2019 --- in which a microcode bug caused the
instruction to return a constant value on certain processors, undermining any
cryptographic operation that relied on it exclusively --- illustrates precisely
this failure mode~\cite{amd_rng_errata}. Systems running affected processors
generated predictable keys for an unknown period. End users whose security
depended on those keys had no way of knowing their risk had changed.

The vocabulary makes this accountability gap nameable and therefore arguable:
the \textbf{benefits} relation points to end users; the \textbf{borne\_by}
relation for the strategic/systemic risk points to the same end users; but
the \textbf{decided\_by} relation points entirely to the chip vendor and the
developer, neither of whom bears the consequence of a failure. This is a
specific, contestable claim about the mechanism's distributional structure ---
one that neither the benchmark literature (which reports throughput and latency
of RdRand calls) nor the economics literature (which operates at market rather
than mechanism level) currently makes.

%-------------------------------------------------------------
\subsection{Mechanism Properties Determine Distributions Prospectively}
\label{sec:prospective}
%-------------------------------------------------------------

The four case studies and two worked examples support a general claim that
is the paper's central analytical contribution: \emph{mechanism properties
determine burden distributions prospectively, not merely retrospectively}.

This claim has two parts. The first is descriptive: the burden pathway of a
security mechanism is not an accident of market conditions or organizational
politics that could only be observed after the fact. It is a structural
consequence of specific technical choices --- activation profile, scope,
the location of key management obligations, the abstraction layers between
entropy source and user --- that could in principle be analyzed before
deployment. The Spectre mitigation costs that fell on cloud operators were
predictable from the always-on, system-wide activation profile of the chosen
mitigations, given the organizational structure of cloud computing. The CHERI
migration costs that fall on software ecosystems are predictable from the
ISA-wide scope of capability enforcement and the distribution of pointer-
manipulating code across the software stack.

The second part is normative: because distributions are prospectively
determinable, mechanism evaluation that reports only local overhead is not
merely incomplete --- it is evaluating the wrong thing. A mechanism that
imposes modest local overhead but transfers large operational costs to downstream
actors is not a cheap mechanism; it is a mechanism whose costs have been
externalized from the measurement context. Reporting only what is measured
is not neutral; it systematically favors mechanisms that externalize costs
over mechanisms that internalize them, because externalized costs are invisible
in standard benchmarks.

This has a concrete implication for how the research community reports results.
A benchmark paper that reports the overhead of a security mechanism without
reporting its activation profile, scope, and bearer structure is not providing
incomplete information --- it is providing information that actively misleads
procurement and adoption decisions by making externalized-cost mechanisms look
cheaper than they are.

%-------------------------------------------------------------
\subsection{Allocation Opacity is Sometimes Convenient}
\label{sec:opacity}
%-------------------------------------------------------------

The fog of war around security costs is partly a consequence of disciplinary
fragmentation --- the two traditions simply do not share vocabulary. But it is
also, in part, maintained. Opacity has beneficiaries, and those beneficiaries
have incentives to preserve it.

Consider what allocation transparency would require of vendors. A chip vendor
who discloses the full burden pathway of a security mechanism --- including the
key management obligations transferred to OEMs, the operational costs
transferred to cloud operators, and the systemic risks externalized to end
users --- is providing information that procurement processes could use to
negotiate, demand changes, or choose alternatives. Opacity forecloses those
negotiations before they begin. A vendor who can present a benchmark figure
and a certification status, without disclosing transfer structure, has
structurally advantaged themselves in procurement conversations.

The same dynamic operates at the regulatory level. Compliance regimes that
reward the presence of a control without measuring its burden distribution
create incentives to adopt controls that impose high downstream costs while
satisfying the compliance requirement. A regulation that mandates secure boot
without specifying the key management obligations it transfers to integrators
and enterprises is a regulation that can be satisfied on paper while
externalizing its real cost to actors who had no say in the requirement.

Hastings and Sethumadhavan~\cite{hastings2020wac} observed that hardware
security failures persist because market forces allow those best positioned
to fix problems to avoid paying for them. The vocabulary makes a more specific
version of this claim: the mechanism by which cost avoidance operates is
allocation opacity, and allocation opacity is produced and maintained by the
absence of a shared schema that makes bearer, transfer, and timing fields
mandatory. The fog of war is not an accident; it is, in part, a product.

This does not require assuming bad faith on the part of vendors or regulators.
Opacity can be maintained by convention, by the absence of standards, and by
the path dependence of reporting norms that were established before
burden-allocation analysis existed as a framework. But recognizing that opacity
has beneficiaries changes the political economy of adopting the vocabulary:
requiring burden-allocation disclosure in benchmarks, procurement, and
regulatory requirements is not a neutral technical improvement. It is a
redistribution of informational power that will be resisted by actors who
currently benefit from the fog.

%-------------------------------------------------------------
\subsection{Implications}
\label{sec:implications}
%-------------------------------------------------------------

The synthesis has three concrete implications for practice.

\paragraph{For mechanism evaluation and benchmarking.}
Benchmark papers should report activation profile and scope alongside overhead
magnitude, and should identify the organizational bearer of the measured
overhead. A benchmark measured on a single-tenant research cluster does not
transfer to a multi-tenant cloud environment where the bearer is an operator
rather than a developer. Reporting context as well as magnitude is the minimum
change required to make benchmark results burden-allocation-aware.

\paragraph{For procurement.}
Hardware security procurement should require burden-allocation disclosure
alongside performance specifications: who absorbs recurring operational cost,
what integration obligations are transferred to the buyer, what contingent
costs are retained by the vendor versus externalized, and what the key
management or ecosystem obligations are over the platform's operational life.
The worked examples show that two mechanisms with similar benchmark overhead
can have radically different total burden structures when transfer and timing
are accounted for.

\paragraph{For standards and regulation.}
Regulatory requirements should specify not only which security mechanisms are
mandated but what their bearer structure must be. A mandate that requires a
control without constraining who bears its cost creates an incentive to adopt
the cheapest-to-vendor implementation, which is typically the one that
externalizes the most cost downstream. Burden-allocation-aware regulation
would require that controls not impose unreasonable obligations on actors
who had no role in the regulatory process --- a principle analogous to the
regulatory concept of proportionality, applied to distributional rather than
aggregate cost.
%=============================================================
\section{Limitations and Future Work}
\label{sec:limitations}
%=============================================================

This paper establishes a conceptual framework and demonstrates its analytical
productivity across six mechanisms. It does not claim to have finished the job.
Several limitations are worth stating explicitly, both to bound the contribution
honestly and to identify the most productive directions for follow-on work.

\paragraph{Inductive scope.}
The vocabulary was developed inductively from a small number of case studies ---
speculative execution controls, four memory safety mechanisms, secure boot, and
hardware random number generation. The concepts that emerged as load-bearing
(activation profile, scope, bearer, timing, transfer, and the bilateral/diffuse
distinction) may not be sufficient to account for the full breadth of hardware
security mechanisms. Mechanisms with unusual deployment models, novel supply
chain structures, or emerging threat contexts may require additional entity
types or relations. The vocabulary should be treated as a well-motivated
starting point, not a finished taxonomy.

\paragraph{Data availability.}
Many of the most consequential burden terms are proprietary or unobserved.
Vendor engineering costs, OEM integration costs, enterprise operational costs,
and incident response expenditures are rarely disclosed in forms that support
rigorous analysis. The schema can be partially populated from public sources ---
benchmark papers, incident reports, regulatory filings --- but a complete
burden-allocation analysis for any real mechanism would require primary data
from vendors and integrators who have strong incentives not to provide it.
This is not a flaw in the vocabulary; it is a reflection of the same allocation
opacity the vocabulary is designed to expose. But it does mean that the
worked examples in this paper are illustrative rather than exhaustive, and that
empirical validation of the framework will require either cooperative disclosure
or natural experiments where costs become visible involuntarily.

\paragraph{Attacker modeling.}
The vocabulary models burden distribution among defenders --- vendors, integrators,
operators, users, and regulators --- but does not model attacker costs or the
strategic interaction between defenders and attackers. Hastings and
Sethumadhavan~\cite{hastings2020wac} include attackers as a fourth player whose
cost-bearing is a legitimate policy lever; our framework sets this aside
deliberately to keep the scope tractable. A fuller treatment would incorporate
attacker economics --- the cost of exploitation, the value of attack targets,
and the asymmetries between offense and defense --- into the burden-allocation
model. This is a meaningful omission for mechanisms whose design explicitly
aims to raise attacker cost, such as hardware-enforced control flow integrity.

\paragraph{Temporal dynamics.}
The vocabulary represents burden pathways as relatively static structures,
but real burden distributions shift over time. The Spectre mitigation burden
on cloud operators will diminish as in-silicon fixes propagate through the
installed base; CHERI's ecosystem migration cost will shift as software
toolchains mature; secure boot's key management burden will evolve as
automation improves. The current schema captures lifecycle phases but does
not model how burden distributions change as mechanisms age, as bypasses are
discovered, or as the ecosystem adapts. A dynamic extension of the vocabulary
would support longitudinal analysis and strengthen the prospective claim.

\paragraph{Formal encoding.}
The vocabulary is presented here in structured prose and semi-formal notation.
A full OWL 2 DL encoding would enable automated consistency checking,
entailment-based inference, and interoperability with existing cybersecurity
ontologies such as UCO and D3FEND. This encoding is a natural next artifact
but is beyond the scope of the present paper. The conceptual commitments
necessary to produce it are fully specified here; the encoding itself is
straightforward for practitioners familiar with description logic tooling.

\paragraph{Future work.}
The most productive near-term directions are three. First, empirical validation:
applying the schema to a broader corpus of mechanisms using primary data from
vendors, integrators, and operators, ideally in a context where disclosure
incentives are aligned --- such as a procurement process that requires
burden-allocation transparency as a condition of evaluation. Second, procurement
tooling: translating the schema into a practical disclosure template that
procurement officers can require vendors to complete, analogous to a
nutritional label for security mechanism costs. Third, regulatory application:
engaging with standards bodies such as NIST, JEDEC, and relevant IEEE working
groups to incorporate burden-allocation fields into existing evaluation and
certification frameworks. The last of these is the highest-leverage intervention
--- if burden-allocation disclosure becomes a standard reporting requirement,
the fog of war begins to clear not by persuasion but by mandate.


% \section{Conceptual Framework and Testable Claims}
\subsection{Core Claim}
The paper's paradigm claim is that hardware-security mechanisms are not only
technical controls; they are also burden-allocation instruments. Mechanism
evaluation is therefore incomplete when it reports only local performance/power/
area impacts and omits who bears lifecycle burden and when
\cite{anderson2001economics,anderson2006economics}.

\subsection{Assumptions}
The claim relies on three assumptions. First, burden pathways are part of the
evaluation target, not post-hoc commentary. Second, transfer and externality
terms are common enough in real deployments to matter for choice. Third,
uncertainty should be explicit and auditable rather than hidden inside point
estimates.

\subsection{Falsifiability}
The claim is rejected if repeated studies show all three of the following:
allocation-aware analysis does not change practical choices, transfer/externality
paths are empirically negligible, and uncertainty-aware reporting does not
improve decision robustness. These criteria are used in this paper as explicit
failure conditions, not rhetorical caveats.

% \section{Background and Related Work}
\subsection{Ontology Primer for Hardware Security Readers}
Because ontology language can be unfamiliar outside knowledge-representation
communities, we fix terminology up front. A \textbf{class} denotes a category
(for example, \textit{SecurityMechanism}, \textit{Stakeholder},
\textit{Cost}); an \textbf{individual} denotes a concrete instance (for
example, a specific TEE deployment cost entry); a \textbf{property} denotes a
typed relation between instances (for example, mechanism \emph{incurs} cost and
cost \emph{borne by} stakeholder); an \textbf{axiom} denotes a constraint (for
example, every cost entry specifies at least one bearer and one time horizon);
and a \textbf{reasoner} denotes an automated checker that validates constraints
and derives implied facts.
Operationally, this is similar to maintaining a strongly typed schema plus
integrity constraints, then running consistency checks and canned analytical
queries. We use ontology tooling because we need explicit semantics for
cross-stakeholder and cross-time reasoning, not only tabular storage.

\subsection{Cybersecurity Ontologies}
Prior ontology efforts in cybersecurity have focused primarily on representing
threat knowledge, incident semantics, and investigative workflows. These
projects commonly model entities such as vulnerabilities, attacks, defensive
techniques, indicators, and forensic artifacts, with the goal of improving
interoperability and machine reasoning across tools and organizations
\cite{syed2016uco,case2024,d3fend2026,nist2023unified}.

This tradition provides useful methodological foundations for our work. First,
it demonstrates how ontology design can bridge fragmented operational datasets.
Second, it shows the value of explicit relation typing (for example,
``mitigates,'' ``depends on,'' and ``caused by'') for downstream query and
analysis. Third, it highlights recurring modeling tensions relevant to our
setting: how to separate events from dispositions, how to encode uncertain
evidence, and how to maintain stability as standards and attacker behavior
evolve.

However, most existing cybersecurity ontologies are not designed to represent
economic burden in hardware development. Cost-bearing pathways, organizational
constraints, and stakeholder transfer effects are usually peripheral rather
than first-class concepts.

\subsection{Hardware Security and Architecture Tradeoffs}
Hardware-security research has produced extensive evidence that protections can
impose measurable overhead in power, performance, die area, design complexity,
or validation effort. Examples include trusted execution mechanisms
\cite{costan2016intelsgx,graphenesgx2017,sanctum2016}, side-channel mitigations
motivated by transient execution attacks \cite{kocher2019spectre,retbleed2022,invisispec2018},
and hardware cryptographic acceleration in secure SoC platforms
\cite{opentitan2024,intel_aesni2012,intel_xeon_crypto2017}, in addition to
broader hardware assurance practices \cite{tehranipoor2011hardware}.

Yet this body of work tends to evaluate local technical impact rather than
system-wide cost distribution. A published mechanism may report PPA overhead
without explicitly modeling who bears integration cost, who receives risk
reduction benefits, and which externalities remain. As a result, two mechanisms
with similar benchmark overhead can have very different lifecycle economics once
deployment, patching, compliance, and ecosystem effects are considered.
This is precisely the missing burden-allocation instrument perspective.

\subsection{Security Economics, Governance, and Incentives}
Security-economics research contributes concepts that are central for hardware:
externalities, principal-agent misalignment, information asymmetry, and
risk-shifting \cite{anderson2001economics,anderson2006economics,akerlof1970lemons}.
These concepts explain why individually rational decisions by vendors,
integrators, or platform operators may underinvest in controls with high social
value, or overinvest in controls that serve compliance signaling more than
actual risk reduction.

Regulatory and assurance regimes add another layer. Certification, audit, and
reporting requirements can improve baseline security but also introduce
engineering overhead, schedule pressure, and documentation debt. These costs are
not uniformly distributed, and they can alter product strategy and market
structure, especially for smaller actors with limited compliance capacity.

\subsection{Gap and Positioning of This Work}
Existing ontology efforts in cybersecurity are mature in representing threats,
incidents, and controls; existing hardware-security efforts are mature in
quantifying mechanism-level technical overhead. What remains underdeveloped is a
formal ontology that links these two traditions through explicit cost
semantics: who pays, who benefits, when costs occur, and how residual risk is
transferred.

This paper positions the cost ontology as connective infrastructure across
technical architecture analysis, organizational decision-making, and policy
evaluation. Instead of treating overhead as a single scalar, we model security
cost as a structured, stakeholder- and time-indexed object that can be queried,
compared, and empirically validated. This makes the
``mechanism-as-burden-allocation-instrument'' framing operational rather than
rhetorical.

\subsection{Novelty Boundary Against Adjacent Ontologies}
Table~\ref{tab:novelty-boundary} summarizes how this work relates to common
security ontology and risk-modeling baselines. The intent is not to replace
these models, but to add a missing cost-semantics layer that can interoperate
with them.

\begin{table*}[t]
  \caption{Positioning of the proposed ontology against adjacent frameworks.}
  \label{tab:novelty-boundary}
  \centering
  \small
  \begin{tabular}{p{0.16\textwidth}p{0.22\textwidth}p{0.22\textwidth}p{0.30\textwidth}}
    \toprule
    Framework family & Primary focus & Typical omission for this problem & Contribution of this work \\
    \midrule
    Threat/control ontologies (for example UCO, CASE) & Cyber events, artifacts, incidents, and investigation semantics & Cost-bearing actor, timing of burden realization, and opportunity cost are not first-class & Adds explicit mechanism--cost--stakeholder--time representation for hardware design and operation \\
    Defensive technique knowledge graphs (for example D3FEND) & Defensive technique catalog and relation to adversary behavior & Does not model full lifecycle burden allocation or externalities across organizations & Adds burden semantics that can map onto technique nodes for comparative evaluation \\
    Cyber risk quantification models (for example FAIR-style) & Financial loss estimation and risk factors & Usually weak on hardware/microarchitectural mechanism representation and implementation cost decomposition & Adds engineering and architectural overhead classes tied to threat mitigation and deployment context \\
    Hardware security evaluation papers & Mechanism-local PPA/performance/security measurements & Usually omit cross-stakeholder transfer, compliance burden, and post-incident cost pathways & Adds a shared schema to aggregate local measurements into system-level cost allocation \\
    \bottomrule
  \end{tabular}
\end{table*}

Taken together, this boundary analysis motivates the next section's explicit
class and relation design: the contribution is not a replacement for threat or
risk ontologies, but a cost-semantics layer that can interoperate with them.

% \section{Ontology of Security Costs}
\subsection{Core Entities}
\begin{itemize}
  \item \textbf{Asset/Objectives}: performance, power, area, schedule, trust, safety, and revenue.
  \item \textbf{Security Mechanism/Requirement}: technical controls and external mandates.
  \item \textbf{Threat/Failure Mode}: exploit class, vulnerability type, attacker capability, and expected impact.
  \item \textbf{Cost Type}: structured classes of burden with explicit definitions.
  \item \textbf{Stakeholder}: who decides, who pays, who benefits, and who bears residual risk.
  \item \textbf{Time Horizon}: design-time, validation-time, deployment-time, incident-time, and post-incident recovery.
  \item \textbf{Evidence}: benchmark data, field incidents, audit artifacts, and expert elicitation.
\end{itemize}

\subsection{Core Relations}
\begin{itemize}
  \item Mechanism \emph{incurs} cost.
  \item Cost is \emph{borne by} stakeholder.
  \item Cost may be \emph{shifted to} a different stakeholder.
  \item Cost is \emph{realized at} a time horizon.
  \item Mechanism \emph{mitigates} threat.
  \item Requirement \emph{constrains} design space.
  \item Constraint \emph{induces} opportunity cost.
  \item Incident \emph{causes} contingent and reputational losses.
  \item Incentive misalignment \emph{shifts} costs across stakeholders.
\end{itemize}

\subsection{Cost Taxonomy}
We define cost as any measurable reduction in technical, organizational,
economic, or social objective value caused by adopting, operating, or omitting
security controls. The ontology distinguishes the following cost classes:
\begin{enumerate}
  \item \textbf{Physical Resource Cost}: die area, static/dynamic power, latency, throughput, and memory overhead.
  \item \textbf{Microarchitectural Performance Cost}: CPI increases, frequency reductions, and disabled speculative behaviors.
  \item \textbf{Engineering Labor Cost}: architecture/design effort, RTL rework, firmware changes, and security review labor.
  \item \textbf{Verification and Validation Cost}: formal proofs, simulation campaigns, emulation, test generation, and bug triage.
  \item \textbf{Toolchain and Infrastructure Cost}: EDA licenses, CI hardening, fuzzing infrastructure, and secure build pipelines.
  \item \textbf{Lifecycle Operations Cost}: key management, patch deployment, field support, secure update operations, and incident response.
  \item \textbf{Compliance and Assurance Cost}: documentation, audits, certification, and recurring attestation obligations.
  \item \textbf{Opportunity Cost}: schedule slip, delayed feature roadmap, lower SKU flexibility, and foregone market windows.
  \item \textbf{Market and Contractual Cost}: warranty exposure, insurance premiums, indemnification terms, and procurement penalties.
  \item \textbf{Reputation and Trust Cost}: customer churn, reduced adoption, investor confidence loss, and long-tail brand damage.
  \item \textbf{Liability and Redress Cost}: legal defense, settlements, recalls, regulatory fines, and remediation programs.
  \item \textbf{Externality Cost}: harms imposed on non-deciding parties, including downstream operators, end users, and public infrastructure.
\end{enumerate}

\subsection{Cost-Bearing Dimensions}
Each instantiated cost in the ontology is annotated along four dimensions:
\begin{itemize}
  \item \textbf{Bearer}: vendor, integrator/OEM, cloud operator, enterprise customer, end user, regulator/public.
  \item \textbf{Timing}: upfront, recurring, contingent, or deferred.
  \item \textbf{Bearing Mode}: internalized, transferred by contract, insured, or externalized.
  \item \textbf{Evidence Type}: measured, estimated, or expert-elicited.
\end{itemize}

This representation allows the same mechanism to carry different burden profiles
across deployment contexts even when technical overhead appears similar.

\subsection{Cost Manifestation Modes}
In addition to ``who pays,'' the ontology tracks \emph{how} costs manifest in
operation:
\begin{itemize}
  \item \textbf{Activation Profile}: always-on, conditional, or event-triggered cost realization.
  \item \textbf{Scope}: system-wide versus module/workload-specific manifestation.
  \item \textbf{Workload Interaction}: whether the mechanism improves, degrades, or redistributes performance across workload classes.
  \item \textbf{Counterfactual Opportunity Cost}: what alternative capability could have been implemented with the same area/power/budget.
\end{itemize}

This distinction is important for mechanisms with mixed effects. For example, a
protection with near-constant overhead (for example, broad randomization
features) is modeled as always-on and wide-scope, while a dedicated crypto block
may accelerate cryptographic workloads but still induce system-level opportunity
cost by consuming resources that could otherwise support general-purpose
performance.

\subsection{Security Posture: Proactive vs Reactive}
The ontology also classifies mechanisms by security posture:
\begin{itemize}
  \item \textbf{Proactive}: mechanisms that reduce exploitability before compromise, typically via prevention or hardening.
  \item \textbf{Reactive}: mechanisms that detect, contain, or recover from compromise after or during attack execution.
  \item \textbf{Hybrid}: mechanisms with both proactive and reactive components.
\end{itemize}

Posture affects cost incidence. Proactive controls often concentrate burden in
upfront design and recurring overhead, while reactive controls can shift burden
toward monitoring operations, incident-time response, and post-incident recovery.
Modeling posture alongside activation profile and scope helps separate
``preventive cost'' from ``response cost'' in cross-mechanism comparisons.

\subsection{Initial Cost-Bearing Matrix}
Table~\ref{tab:cost-bearing-matrix} summarizes representative burden patterns
that the ontology is designed to capture and query.

\begin{table*}[t]
  \caption{Illustrative cost-bearing matrix by stakeholder and time horizon.}
  \label{tab:cost-bearing-matrix}
  \centering
  \small
  \begin{tabular}{p{0.17\textwidth}p{0.11\textwidth}p{0.11\textwidth}p{0.11\textwidth}p{0.40\textwidth}}
    \toprule
    Stakeholder & Upfront & Recurring & Contingent & Typical burden classes \\
    \midrule
    Chip vendor & High & Medium & Medium & Physical resource, engineering labor, verification and validation \\
    OEM/integrator & Medium & Medium & Medium & Integration overhead, compliance burden, lifecycle operations \\
    Cloud operator & Medium & High & High & Performance overhead, recurring operations, incident response \\
    Enterprise customer & Low & Medium & High & Migration effort, downtime risk, contractual and liability exposure \\
    End user/public & Low & Low & High & Service degradation, privacy harms, and externalized impacts \\
    \bottomrule
  \end{tabular}
\end{table*}

\subsection{Machine-Readable Representation}
The formal class/property encoding of this taxonomy is maintained in the
companion artifact released with the paper materials, which provides canonical
definitions for cost classes, stakeholder roles, and burden relations.

% \section{Analytical Framework}
\subsection{Multi-Stakeholder Cost Accounting}
Define total expected cost as a function of direct implementation costs,
probability-weighted incident losses, and cost-shifting terms:
\begin{equation}
\begin{aligned}
C_{\text{total}} =\;& C_{\text{direct}} + \mathbb{E}[L_{\text{incident}}] + C_{\text{compliance}} \\
& + C_{\text{opportunity}} + C_{\text{externalized}}.
\end{aligned}
\end{equation}

\subsection{Risk Transfer and Incentive Misalignment}
A key claim of this draft is that many hardware security decisions optimize
local objective functions while externalizing residual risk, which mirrors
established observations from security economics \cite{anderson2001economics,anderson2006economics}.
The ontology captures these transfers explicitly.

% \section{Evidence-Grounded Case Studies}
\subsection{Why Deep Case Mapping Matters}
Case studies are the bridge between ontology semantics and hardware-security
practice. They test whether classes and relations are expressive enough for
real mechanisms, and they expose burden pathways that mechanism-local reporting
usually hides (transfer, externality, and delayed loss effects)
\cite{anderson2001economics,anderson2006economics}.
In other words, they test whether the burden-allocation instrument framing
matches how mechanisms are engineered and deployed.

\subsection{Case Design and Corpus}
The seed corpus now covers six families (20 tuples each): speculation controls,
TEEs, cryptographic accelerators, rowhammer mitigations, memory-safety
mitigations, and reactive runtime detection/response controls. We additionally
encode 12 incident-linkage tuples to connect weak/missing control families to
loss observations with confidence annotations. Each family is treated as a
distinct burden-allocation instrument profile.

\subsection{Cross-Family Snapshot}
Table~\ref{tab:worked-case} summarizes representative anchors and ontology
dimensions across families.

\begin{table*}[t]
  \caption{Cross-family measurements mapped to ontology dimensions.}
  \label{tab:worked-case}
  \centering
  \scriptsize
  \setlength{\tabcolsep}{3pt}
  \begin{tabular}{p{0.14\textwidth}p{0.07\textwidth}p{0.10\textwidth}p{0.13\textwidth}p{0.21\textwidth}p{0.17\textwidth}p{0.10\textwidth}}
    \toprule
    Family & Posture & Activation & Scope & Representative anchors & Primary bearer(s) & Timing \\
    \midrule
    Speculation controls & Proactive & Often always-on & System-wide & Retbleed slowdown 5.78--27.90\% \cite{retbleed2022}; InvisiSpec 21\% mean overhead vs.\ 74\% STT \cite{invisispec2018} & Operator, platform vendor & Recurring \\
    TEE isolation & Proactive & Mixed & Module specific & Sanctum: +0.78\% gates, +1.9\% flip-flops \cite{sanctum2016}; Graphene-SGX: near-native to below 2x \cite{graphenesgx2017} & Vendor, integrator, operator & Upfront + recurring \\
    Crypto acceleration & Proactive & Conditional & Module specific & OpenTitan AES up to 44x \cite{opentitan2024}; AES-NI 2--3x to 10x \cite{intel_aesni2012} & Vendor, integrator & Upfront + recurring \\
    Rowhammer mitigations & Proactive & Mostly always-on & System-wide memory path & Disturbance characterization \cite{kim2014rowhammer}; TRRespass/Blacksmith bypass pressure \cite{trrespass2020,blacksmith2022} & Vendor, operator & Upfront + recurring \\
    Memory safety mitigations & Proactive & Conditional checks & Module/workload-specific & Arm MTE deployment overhead guidance \cite{arm_mte2021}; CHERI migration tradeoffs \cite{cheri2015} & Vendor, integrator, operator & Upfront + recurring \\
    Runtime detection/response & Reactive & Event-triggered & System-wide telemetry/recovery path & Firmware resiliency/recovery workflow requirements \cite{nist800193}; runtime platform instrumentation context \cite{opentitan2024} & Operator, integrator, customer & Recurring + contingent \\
    \bottomrule
  \end{tabular}
\end{table*}

\subsection{Structured Worked Outputs}
To avoid purely narrative treatment, each family is summarized as
(\textit{tuple slice}, \textit{measured anchors}, \textit{inferred/synthetic
allocations}, \textit{decision implication}) in
Table~\ref{tab:worked-structured}.

\begin{table*}[t]
  \caption{Worked case outputs with explicit data slices and decision implications.}
  \label{tab:worked-structured}
  \centering
  \scriptsize
  \setlength{\tabcolsep}{3pt}
  \begin{tabular}{p{0.13\textwidth}p{0.13\textwidth}p{0.20\textwidth}p{0.23\textwidth}p{0.23\textwidth}}
    \toprule
    Family & Tuple slice & Measured anchors (E1/Measured) & Inferred or synthetic allocations (E2/E3) & Decision implication \\
    \midrule
    Speculation controls & \texttt{S01--S20} & Runtime overhead and ops burden anchors from Retbleed \cite{retbleed2022} & Transfer/externalized burdens under alternative mitigation assumptions from InvisiSpec-context rows \cite{invisispec2018} & ``Best local runtime'' and ``lowest transferred burden'' can diverge. \\
    TEE isolation & \texttt{T21--T40} & Hardware increments and selected runtime behavior anchors from Sanctum \cite{sanctum2016} & Integration and downstream burden variation from Graphene-SGX style adaptation contexts \cite{graphenesgx2017} & Upfront provisioning and recurring adaptation must be jointly optimized. \\
    Crypto acceleration & \texttt{C41--C60} & Crypto-path speedups and area footprints from OpenTitan/AES-NI \cite{opentitan2024,intel_aesni2012} & Opportunity and transfer effects for non-crypto workloads and operations \cite{intel_xeon_crypto2017} & Acceleration benefit does not remove system-level opportunity cost. \\
    Rowhammer mitigations & \texttt{R61--R80} & Disturbance and defense-pressure anchors from ISCA/SP evidence \cite{kim2014rowhammer,trrespass2020,blacksmith2022} & Validation refresh and externalized reliability burden in E2/E3 rows & Durability costs can dominate one-time mitigation overhead. \\
    Memory safety & \texttt{M81--M100} & Runtime and deployment anchors from MTE/CHERI sources \cite{arm_mte2021,cheri2015} & Toolchain migration and transfer/externality terms in E2/E3 rows & Adoption economics materially affect mechanism ranking. \\
    Runtime detection/response & \texttt{D101--D120} & None in current seed (deliberately no E1 reactive rows) & Recovery-time burden, contractual spillover, and contingent loss terms from inferred/synthetic rows \cite{nist800193,anderson2006economics} & Reactive controls expose contingent burden that proactive-only views miss. \\
    \bottomrule
  \end{tabular}
\end{table*}

\subsection{Worked Family Notes}
\textbf{Speculation controls (\texttt{S01--S20}).}
Always-on behavior and system-wide scope produce recurring operator burden. The
seed shows decision-delta behavior where minimizing local runtime and
minimizing transferred burden can select different tuples under different
objective weights.

\textbf{TEE isolation (\texttt{T21--T40}).}
TEE rows show mixed manifestation: upfront hardware/verification and recurring
runtime adaptation. This family prevents collapse into a single ``TEE overhead''
scalar by forcing separate burden channels.

\textbf{Crypto acceleration (\texttt{C41--C60}).}
This family encodes both acceleration gains and opportunity costs.
Allocation-aware analysis keeps ``faster crypto path'' and ``foregone
general-purpose capacity'' in the same decision view.

\textbf{Rowhammer mitigations (\texttt{R61--R80}).}
These rows model mitigation durability pressure: bypass research implies
recurring validation and refresh burden even when first-generation controls are
deployed.

\textbf{Memory safety mitigations (\texttt{M81--M100}).}
Rows mix hardware support, toolchain integration, and migration costs. The
family is useful because burden incidence can move between vendor, integrator,
and operator depending on rollout strategy.

\textbf{Runtime detection/response (\texttt{D101--D120}).}
This reactive family captures event-triggered and contingent burden channels
that are absent from proactive-only corpora: incident handling, recovery
operations, contractual penalties, and reputation effects \cite{nist800193}.
It highlights that reactive mechanisms are burden-allocation instruments with
different timing signatures than proactive mechanisms.

\paragraph{Incident linkage for CQ4.}
Incident tuples (\texttt{I01--I12}) encode observed/seeded losses linked to
families with attribution confidence and provenance. This enables executable
loss-linkage queries instead of treating attribution as future work.

\paragraph{Community value.}
The structured format above is reusable: each family can be extended with
additional rows without changing schema, and review can target concrete tuple
claims rather than prose-only argument. That improves comparability across
architecture, systems-security, and policy analyses.

\paragraph{Seed artifact status.}
The current release is still a seed dataset, not a final benchmark corpus:
120 cost tuples, 12 incident tuples, explicit data-origin labels, and full CQ
coverage under executable checks. This is enough for methodological validation,
while still requiring broader source-complete expansion for benchmark-grade
conclusions.

% \section{Formal Methods}
\subsection{Methodological Goals}
The methods section serves four goals: (1) make ontology construction
reproducible, (2) justify modeling decisions as answers to explicit research
questions, (3) define objective validation criteria, and (4) connect the
resulting ontology to empirical analysis in hardware-security settings.

\subsection{Ontology Engineering Process}
We follow a competency-question-driven ontology workflow aligned with established
ontology-engineering practice \cite{noy2001ontology,fernandez1997methontology}.
The process has six stages:
\begin{enumerate}
  \item \textbf{Specification}: define scope, stakeholders, intended users, and non-goals.
  \item \textbf{Knowledge acquisition}: collect terms, relations, and measurement concepts from hardware-security and security-economics literature.
  \item \textbf{Conceptualization}: define core classes and relation patterns (for example, mechanism--cost, stakeholder--burden, and threat--impact).
  \item \textbf{Formalization}: encode the model in OWL 2 with explicit object properties, data properties, and constraints.
  \item \textbf{Implementation}: build a versioned machine-readable artifact (Turtle/RDF or OWL/XML) with example instances.
  \item \textbf{Evaluation and iteration}: test queries, refine class boundaries, and resolve inconsistencies.
\end{enumerate}

\subsection{Competency Questions}
Competency questions (CQs) define what the ontology must answer
\cite{gruninger1995methodology}. Initial CQs for this paper are:
\begin{itemize}
  \item \textbf{CQ1}: Which stakeholders bear the direct and indirect costs of a given security mechanism?
  \item \textbf{CQ2}: Which cost categories occur at design time versus deployment time versus incident time?
  \item \textbf{CQ3}: For a given requirement, which costs are internalized versus externalized?
  \item \textbf{CQ4}: Which mechanisms mitigate the same threat class but induce different cost distributions?
  \item \textbf{CQ5}: Which observed incident losses can be mapped to missing or weakly adopted controls?
\end{itemize}

\subsection{CQ-to-Model Mapping}
To make evaluation reproducible, each CQ is mapped to explicit conceptual
elements in the ontology:
\begin{itemize}
  \item \textbf{CQ1} (who bears cost) maps to mechanism--cost--stakeholder relations.
  \item \textbf{CQ2} (when costs occur) maps to cost-class and time-horizon modeling.
  \item \textbf{CQ3} (internalized versus externalized burden) maps to requirement constraints, bearing-mode attributes, and cost-allocation relations.
  \item \textbf{CQ4} (cross-mechanism comparison) maps to shared threat-mitigation links combined with stakeholder-time cost distributions and security posture (proactive/reactive/hybrid).
  \item \textbf{CQ5} (incident loss linkage) is partially covered by threat-mitigation and failure-cost concepts; full coverage requires explicit incident-event classes in the next revision.
\end{itemize}

\subsection{Formalization and Semantics}
The ontology will be represented in OWL 2 DL to support automated consistency
checking while preserving sufficient expressiveness for class restrictions and
property assertions. We separate:
\begin{itemize}
  \item \textbf{Structural concepts}: mechanism, stakeholder, threat, asset, requirement.
  \item \textbf{Cost concepts}: cost type, magnitude, timing, uncertainty, and payer/beneficiary roles.
  \item \textbf{Evidence concepts}: source type, confidence level, and measurement provenance.
\end{itemize}
To reduce taxonomic ambiguity, class hierarchies will be reviewed using
OntoClean-style meta-properties (rigidity, identity, dependence) where
applicable \cite{guarino2009ontoclean}.

\subsection{Representative Axioms}
To make formal commitments explicit, the current draft uses the following
representative constraints:
\begin{itemize}
  \item \textbf{Disjointness}: \textit{ProactivePosture}, \textit{ReactivePosture}, and \textit{HybridPosture} are pairwise disjoint subclasses of \textit{SecurityPosture}.
  \item \textbf{Participation}: every \textit{CostInstance} must have at least one bearer and one realization time.
  \item \textbf{Typing}: values of \textit{hasActivationProfile} are restricted to \textit{AlwaysOn}, \textit{Conditional}, or \textit{EventTriggered}.
  \item \textbf{Transfer semantics}: if a cost is marked \textit{externalized}, there exists a stakeholder distinct from the decision-maker that bears part of the burden.
\end{itemize}

In description-logic style, a minimal fragment is:
\begin{equation}
\begin{aligned}
\textit{CostInstance} &\sqsubseteq \exists \textit{borneBy}.\textit{Stakeholder} \\
\textit{CostInstance} &\sqsubseteq \exists \textit{realizedAt}.\textit{TimeHorizon} \\
\textit{SecurityMechanism} &\sqsubseteq \exists \textit{hasSecurityPosture}.\textit{SecurityPosture}.
\end{aligned}
\end{equation}
These axioms are intentionally lightweight in the draft and can be tightened in
camera-ready versions after additional data mapping.

\subsection{Validation Strategy}
We evaluate the ontology against three criteria:
\begin{enumerate}
  \item \textbf{Logical validity}: ontology consistency and satisfiability under a DL reasoner.
  \item \textbf{Question coverage}: whether CQs can be answered by executable queries.
  \item \textbf{Analytical utility}: whether outputs support comparative assessments of hardware-security mechanisms across stakeholder-time matrices.
\end{enumerate}
For empirical grounding, we map case-study data into normalized tuples
(\textit{mechanism}, \textit{cost type}, \textit{stakeholder}, \textit{time},
\textit{evidence}) and perform sensitivity analysis when probabilities or loss
magnitudes are uncertain.

\subsection{Measurement Protocol}
Each mapped cost instance includes a unit, uncertainty annotation, and evidence
grade to improve reproducibility:
\begin{itemize}
  \item \textbf{Physical/resource metrics}: percent area, watts, performance delta, memory overhead.
  \item \textbf{Labor/process metrics}: engineer-months, verification campaign size, patch turnaround time.
  \item \textbf{Governance metrics}: audit effort hours, certification cycle length, compliance frequency.
  \item \textbf{Loss metrics}: incident frequency, recovery cost, liability outlay, churn proxy.
\end{itemize}

Evidence quality is tracked with a three-level rubric:
\begin{itemize}
  \item \textbf{E1 (measured)}: benchmarked or operationally observed values.
  \item \textbf{E2 (estimated)}: model-based interpolation/extrapolation.
  \item \textbf{E3 (elicited)}: expert-judgment assumptions.
\end{itemize}
All reported comparisons state the proportion of E1/E2/E3 values and include a
sensitivity range when E2 or E3 dominates.

\subsection{Executable CQ Templates}
To demonstrate question coverage, we use query templates over the
machine-readable artifact:

\begin{verbatim}
# CQ1: Who bears costs for a given mechanism?
SELECT ?stakeholder ?costType ?time WHERE {
  ?m a ex:SecurityMechanism ; ex:label "AES accelerator" ;
     ex:incurs ?c .
  ?c ex:borneBy ?stakeholder ;
     ex:hasCostType ?costType ;
     ex:realizedAt ?time .
}
\end{verbatim}

\begin{verbatim}
# CQ4: Compare mechanisms mitigating the same threat
SELECT ?m ?posture ?stakeholder ?costType WHERE {
  ?m ex:mitigates ex:TransientExecutionThreat ;
     ex:hasSecurityPosture ?posture ;
     ex:incurs ?c .
  ?c ex:borneBy ?stakeholder ;
     ex:hasCostType ?costType .
}
\end{verbatim}

% \section{Evaluation Results}
\subsection{Executed Workflow}
The current release was evaluated using seven artifacts:
\path{ontology.ttl}, \path{artifacts/cost_tuples.csv},
\path{artifacts/incident_tuples.csv}, \path{artifacts/cq_queries.sparql},
\path{artifacts/cq_results.csv}, \path{artifacts/voi_priorities.csv}, and
\path{artifacts/sensitivity_rankings.csv}. We regenerate CQ outputs from data
with \texttt{scripts/generate\_cq\_results.py} and then run integrity and
distribution checks via \texttt{make repro}.

\subsection{Artifact Integrity Results}
The seed cost dataset contains 120 tuples across six mechanism families (20 per
family) with no missing values in key analysis fields (family, cost type, time
horizon, bearing mode, evidence grade, data origin, source key, source
locator). Distribution is: 59 E1, 39 E2, and 22 E3 rows; 59 Measured, 39
Inferred, and 22 Synthetic rows; 54 upfront, 60 recurring, 4 contingent, and 2
deferred rows; 66 internalized, 34 transferred, and 20 externalized rows.

Incident linkage contains 12 tuples with complete family-linked loss,
confidence, and provenance fields (8 E2/Inferred and 4 E3/Synthetic).

\subsection{Competency-Question Coverage Results}
All six CQs are executable in the current release through generated artifacts.
CQ4 and CQ6 moved from partial to pass because incident-linkage tuples and VOI
ranking are now part of the artifact pipeline.

\begin{table}[t]
  \caption{Observed CQ coverage results from executable artifacts.}
  \label{tab:cq-coverage}
  \centering
  \footnotesize
  \setlength{\tabcolsep}{3pt}
  \begin{tabular}{p{0.18\columnwidth}p{0.10\columnwidth}p{0.60\columnwidth}}
    \toprule
    CQ & Status & Coverage metric \\
    \midrule
    CQ1 (visibility) & Pass & 120/120 tuples include bearer, time, unit, evidence, data origin, and source provenance. \\
    CQ2 (transfer) & Pass & 120/120 tuples include bearing mode (Externalized=20). \\
    CQ3 (comparison) & Pass & 6/6 families satisfy shared comparability criteria. \\
    CQ4 (attribution) & Pass & 12/12 incident tuples include family-linked loss, confidence, and provenance. \\
    CQ5 (sensitivity) & Pass & Family ranking stable under $\pm20\%$ E2/E3 perturbation. \\
    CQ6 (information gap) & Pass & VOI priorities computed for 40 family-cost cells. \\
    \bottomrule
  \end{tabular}
\end{table}

\subsection{RQ1--RQ4 Answer Summary}
Table~\ref{tab:rq-summary} reports direct answers to RQ1--RQ4 with explicit
seed-corpus scope.

\begin{table*}[t]
  \caption{Direct answers to RQ1--RQ4 with evidence anchors in the current seed release.}
  \label{tab:rq-summary}
  \centering
  \small
  \begin{tabular}{p{0.07\textwidth}p{0.44\textwidth}p{0.41\textwidth}}
    \toprule
    RQ & Answer from current results & Evidence anchor \\
    \midrule
    RQ1 & Security mechanisms induce heterogeneous burden classes across lifecycle phases, not a single scalar overhead. & 120 tuples spanning physical, performance, labor, operations, compliance, strategic, and externality costs across upfront/recurring/contingent/deferred horizons. \\
    RQ2 & A usable representation requires bearer, time horizon, bearing mode, evidence grade, data origin, and source provenance per row. & Complete field coverage in \texttt{cost\_tuples.csv} with executable CQ1--CQ2 checks. \\
    RQ3 & Transfer and externality semantics are necessary to model incentive misalignment in hardware-security choice. & 34 transferred and 20 externalized tuples; decision-delta examples where local-performance and allocation-aware choices diverge. \\
    RQ4 & Cross-family comparison is feasible when families share posture, activation, scope, and burden schema. & CQ3 pass across six families, including one reactive family, on a shared burden representation. \\
    \bottomrule
  \end{tabular}
\end{table*}

\subsection{Decision-Delta and Reactive Coverage}
Compared with prior proactive-only coverage, the reactive
runtime-detection/response family adds contingent and deferred burden channels.
This enables explicit proactive-versus-reactive portfolio discussion in the same
schema, instead of treating posture as a descriptive label only.

\subsection{Sensitivity Results (CQ5)}
Microarchitectural performance ranking was stable across baseline and
$\pm20\%$ perturbations of E2/E3 values:
CryptoAccelerators, RuntimeDetectionResponse, MemorySafetyMitigations,
RowhammerMitigations, TrustedExecutionEnvironments, SpeculationControls.
This result is generated and recorded in
\texttt{artifacts/sensitivity\_rankings.csv}.

\subsection{VOI Results (CQ6)}
The VOI pipeline ranks high-value measurement gaps by uncertainty, transfer, and
incident linkage. Top-ranked cells in the current seed release include
RuntimeDetectionResponse/ReputationTrustCost and transfer-heavy crypto,
rowhammer, and memory-safety cells. Full ranking is emitted to
\texttt{artifacts/voi\_priorities.csv}.

\subsection{Calibration of Claims}
These results establish methodological execution, not final empirical
ground truth. The corpus remains a seed dataset with mixed measured and
inferred/synthetic rows, so conclusions should be read as demonstrating
decision visibility and queryability rather than definitive market-level
effect sizes.

\subsection{Reproducibility Outputs}
Running \texttt{make repro} regenerates CQ/VOI/sensitivity artifacts and writes
\texttt{artifacts/repro\_report.txt} with tuple counts, distribution checks, and
CQ status summaries used in this section.

% \section{Discussion}
\subsection{Policy and Regulation}
Regulatory requirements can reduce social risk while increasing localized
industry burden; the ontology helps identify when these interventions are
productive versus distortionary.

\subsection{Design Implications}
Architects should evaluate security features by system-level welfare impact,
not only local PPA or benchmark metrics.

\subsection{Limitations}
This draft still has three major limitations: limited measured hardware data,
incomplete incident-loss mapping, and dependence on mixed evidence quality.

\subsection{Threats to Validity}
\begin{itemize}
  \item \textbf{Selection bias}: published mechanisms over-represent successful or benchmark-friendly designs.
  \item \textbf{Measurement heterogeneity}: area, power, and performance numbers are often reported under incompatible assumptions.
  \item \textbf{Attribution ambiguity}: incident losses may be multi-causal and difficult to map to one missing control.
  \item \textbf{Private-cost opacity}: legal, contract, and response costs are frequently proprietary.
\end{itemize}
These risks are mitigated by explicit evidence grading, sensitivity analysis,
and by reporting uncertainty intervals alongside point estimates.

\subsection{Bibliographic and Dataset Maturity}
The current draft cites representative prior work, but a camera-ready version
requires a more systematic corpus review and a source-complete benchmark table
for all numeric claims.

%=============================================================
\section{Conclusion}
\label{sec:conclusion}
%=============================================================

Hardware security mechanisms are not neutral technical controls. They are
instruments that distribute costs, risks, and residual liabilities across
stakeholders, organizations, and timescales --- and they do so as a direct
consequence of specific design choices that could, in principle, have been
made differently. The performance tax from Spectre mitigations did not
accidentally fall on cloud operators; it fell there because the always-on,
system-wide activation profile of the chosen mitigations made operator-borne
recurring overhead structurally inevitable given the organizational structure
of cloud computing. The ecosystem migration burden of CHERI did not
accidentally fall on software maintainers; it fell there because capability-
wide pointer enforcement requires every piece of pointer-manipulating code
to be ported, and that code lives in the ecosystem, not in the vendor's
design team.

These distributions are not mysteries. They are readable from the technical
properties of the mechanisms themselves, once you have the vocabulary to
read them. That vocabulary is what this paper provides.

The synthesis we have developed brings together two traditions that have been
analyzing the same phenomenon from incompatible angles. The systems engineering
tradition characterizes mechanism costs with precision but treats distribution
as outside its scope. The security economics tradition characterizes
distribution clearly but treats the mechanism as a black box. Neither can
trace burden pathways through the mechanism --- from specific technical design
choices to specific distributional outcomes. The burden-allocation vocabulary
fills the gaps in both traditions simultaneously: it gives the engineering
analysis the distributional concepts it lacks, and gives the economics analysis
the mechanism-level causal structure it lacks.

The result is not merely a richer description of what has already happened.
It is a framework for prospective analysis: given a proposed mechanism, what
activation profile does it imply, what scope, and therefore who will bear its
recurring cost? What integration obligations does it transfer to downstream
actors, and are those actors the ones who made the decision? What systemic
risks does it externalize to parties who have no visibility into the quality
of what they are receiving? These questions can be asked before deployment,
and they should be --- by designers, by procurement officers, by regulators,
and by the research community that produces the benchmark literature on which
all of them rely.

The fog of war around hardware security costs is real, consequential, and
partly maintained. It persists because the two traditions that could together
dispel it have not been in conversation, and because the actors who benefit
from opacity have little incentive to supply the missing vocabulary voluntarily.
Clearing it requires not only the conceptual framework this paper provides but
the institutional will to require burden-allocation disclosure as a standard
condition of evaluation, procurement, and certification.

Hardware security is not just an engineering problem and not just an economics
problem. It is both, simultaneously, and the failure to treat it as both is
why the costs keep ending up somewhere unexpected.

\bibliographystyle{ACM-Reference-Format}
\bibliography{refs}

\end{document}
