% Anonymous submission toggle:
% - \anonsubmissiontrue  => anonymous submission build
% - \anonsubmissionfalse => named draft / camera-ready build
\newif\ifanonsubmission
\anonsubmissiontrue
\ifanonsubmission
  \PassOptionsToClass{anonymous}{acmart}
\fi
\documentclass[sigconf]{acmart}

\usepackage{amsmath}
\usepackage{booktabs}

\setcopyright{none}
\settopmatter{printfolios=true}
\acmDOI{}
\acmISBN{}
\acmConference[NSPW '26]{New Security Paradigms Workshop}{2026}{TBD}
\acmBooktitle{Proceedings of the New Security Paradigms Workshop (NSPW '26)}

\title{An Ontology of Costs in Hardware and Systems Security}

\author{Author Name}
\affiliation{%
  \institution{Affiliation}
  \city{City}
  \country{Country}
}
\email{email@example.com}

\begin{document}
\begin{abstract}
Security in hardware and computer architecture is often treated as a feature,
but in practice it manifests as a portfolio of costs: power, performance, die
area, engineering effort, compliance overhead, schedule delay, and downstream
reputational or market losses when failures occur. This paper proposes an
ontology for these costs and advances a central framing: hardware security
mechanisms should be treated as \emph{burden-allocation instruments}, not only
as local technical controls. The paper sketches a framework for comparing
security investments across stakeholders and timescales under this framing.
\end{abstract}

\maketitle

\begin{CCSXML}
<ccs2012>
<concept>
<concept_id>10002978.10003022.10003023</concept_id>
<concept_desc>Security and privacy~Embedded systems security</concept_desc>
<concept_significance>500</concept_significance>
</concept>
<concept>
<concept_id>10002978.10003022.10003036</concept_id>
<concept_desc>Security and privacy~Systems security</concept_desc>
<concept_significance>300</concept_significance>
</concept>
</ccs2012>
\end{CCSXML}
\ccsdesc[500]{Security and privacy~Embedded systems security}
\ccsdesc[300]{Security and privacy~Systems security}

\keywords{hardware security, ontology, security economics, computer architecture, burden-allocation instruments, risk transfer}

\section{Introduction}
Hardware security decisions are frequently framed as technical tradeoffs (for
example, performance versus protection), but this framing is incomplete.
Security costs are distributed unevenly across actors (chip vendors, OEMs,
cloud operators, users, and regulators) and across time (upfront design effort
versus contingent incident losses). This paper develops a structured ontology to
unify these costs and make tradeoffs explicit.

\subsection{Research Questions}
This paper is guided by four research questions:
\begin{itemize}
  \item \textbf{RQ1}: What cost types are induced by hardware-security mechanisms across the design, deployment, and incident lifecycle?
  \item \textbf{RQ2}: How should these costs be represented so that stakeholder, time horizon, and uncertainty are explicit?
  \item \textbf{RQ3}: Which relations are required to reason about risk transfer, externalities, and incentive misalignment?
  \item \textbf{RQ4}: How can the ontology support comparable analysis across mechanisms with different technical overhead and governance context?
\end{itemize}

\subsection{Contributions}
\begin{itemize}
  \item A lifecycle-oriented taxonomy of security costs in hardware, spanning physical resource, engineering process, compliance, strategic, and externality costs.
  \item A formal ontology design that links mechanisms, threats, stakeholders, and cost-bearing pathways with explicit temporal semantics.
  \item A multi-stakeholder analytical framing that combines direct implementation costs, expected incident loss, and cost externalization.
  \item A set of competency-question-driven evaluation goals for testing whether the ontology supports policy and architecture decision use cases.
\end{itemize}

\section{Conceptual Framework and Testable Claims}
\subsection{Core Claim}
The paper's paradigm claim is that hardware-security mechanisms are not only
technical controls; they are also burden-allocation instruments. Mechanism
evaluation is therefore incomplete when it reports only local performance/power/
area impacts and omits who bears lifecycle burden and when
\cite{anderson2001economics,anderson2006economics}.

\subsection{Assumptions}
The claim relies on three assumptions. First, burden pathways are part of the
evaluation target, not post-hoc commentary. Second, transfer and externality
terms are common enough in real deployments to matter for choice. Third,
uncertainty should be explicit and auditable rather than hidden inside point
estimates.

\subsection{Falsifiability}
The claim is rejected if repeated studies show all three of the following:
allocation-aware analysis does not change practical choices, transfer/externality
paths are empirically negligible, and uncertainty-aware reporting does not
improve decision robustness. These criteria are used in this paper as explicit
failure conditions, not rhetorical caveats.

\section{Motivating Problem and Scope}
\subsection{What Counts as a Security Cost?}
Security costs include direct design overhead (area and energy), verification
and validation effort, software/hardware co-design complexity, certification and
compliance burdens, opportunity costs from constrained design space, and
post-incident costs (recall, liability, and reputational damage)
\cite{anderson2001economics,anderson2006economics}.

\subsection{System Boundary}
This draft focuses on security mechanisms and requirements that affect:
\begin{itemize}
  \item Microarchitecture and SoC design,
  \item Platform firmware and trusted boot chains,
  \item Supply-chain assurance and lifecycle support,
  \item Product and organizational decision-making.
\end{itemize}

\subsection{Non-Goals}
To keep the ontology tractable in the first version, this paper does not
attempt to fully model:
\begin{itemize}
  \item purely software-only control economics with no hardware or firmware coupling,
  \item geopolitical or national-security externalities beyond product and operator contexts,
  \item attacker utility models and adversary adaptation dynamics in full game-theoretic detail,
  \item jurisdiction-specific legal doctrine for liability allocation.
\end{itemize}

\section{Background and Related Work}
\subsection{Ontology Primer for Hardware Security Readers}
Because ontology language can be unfamiliar outside knowledge-representation
communities, we fix terminology up front. A \textbf{class} denotes a category
(for example, \textit{SecurityMechanism}, \textit{Stakeholder},
\textit{Cost}); an \textbf{individual} denotes a concrete instance (for
example, a specific TEE deployment cost entry); a \textbf{property} denotes a
typed relation between instances (for example, mechanism \emph{incurs} cost and
cost \emph{borne by} stakeholder); an \textbf{axiom} denotes a constraint (for
example, every cost entry specifies at least one bearer and one time horizon);
and a \textbf{reasoner} denotes an automated checker that validates constraints
and derives implied facts.
Operationally, this is similar to maintaining a strongly typed schema plus
integrity constraints, then running consistency checks and canned analytical
queries. We use ontology tooling because we need explicit semantics for
cross-stakeholder and cross-time reasoning, not only tabular storage.

\subsection{Cybersecurity Ontologies}
Prior ontology efforts in cybersecurity have focused primarily on representing
threat knowledge, incident semantics, and investigative workflows. These
projects commonly model entities such as vulnerabilities, attacks, defensive
techniques, indicators, and forensic artifacts, with the goal of improving
interoperability and machine reasoning across tools and organizations
\cite{syed2016uco,case2024,d3fend2026,nist2023unified}.

This tradition provides useful methodological foundations for our work. First,
it demonstrates how ontology design can bridge fragmented operational datasets.
Second, it shows the value of explicit relation typing (for example,
``mitigates,'' ``depends on,'' and ``caused by'') for downstream query and
analysis. Third, it highlights recurring modeling tensions relevant to our
setting: how to separate events from dispositions, how to encode uncertain
evidence, and how to maintain stability as standards and attacker behavior
evolve.

However, most existing cybersecurity ontologies are not designed to represent
economic burden in hardware development. Cost-bearing pathways, organizational
constraints, and stakeholder transfer effects are usually peripheral rather
than first-class concepts.

\subsection{Hardware Security and Architecture Tradeoffs}
Hardware-security research has produced extensive evidence that protections can
impose measurable overhead in power, performance, die area, design complexity,
or validation effort. Examples include trusted execution mechanisms
\cite{costan2016intelsgx,graphenesgx2017,sanctum2016}, side-channel mitigations
motivated by transient execution attacks \cite{kocher2019spectre,retbleed2022,invisispec2018},
and hardware cryptographic acceleration in secure SoC platforms
\cite{opentitan2024,intel_aesni2012,intel_xeon_crypto2017}, in addition to
broader hardware assurance practices \cite{tehranipoor2011hardware}.

Yet this body of work tends to evaluate local technical impact rather than
system-wide cost distribution. A published mechanism may report PPA overhead
without explicitly modeling who bears integration cost, who receives risk
reduction benefits, and which externalities remain. As a result, two mechanisms
with similar benchmark overhead can have very different lifecycle economics once
deployment, patching, compliance, and ecosystem effects are considered.
This is precisely the missing burden-allocation instrument perspective.

\subsection{Security Economics, Governance, and Incentives}
Security-economics research contributes concepts that are central for hardware:
externalities, principal-agent misalignment, information asymmetry, and
risk-shifting \cite{anderson2001economics,anderson2006economics,akerlof1970lemons}.
These concepts explain why individually rational decisions by vendors,
integrators, or platform operators may underinvest in controls with high social
value, or overinvest in controls that serve compliance signaling more than
actual risk reduction.

Regulatory and assurance regimes add another layer. Certification, audit, and
reporting requirements can improve baseline security but also introduce
engineering overhead, schedule pressure, and documentation debt. These costs are
not uniformly distributed, and they can alter product strategy and market
structure, especially for smaller actors with limited compliance capacity.

\subsection{Gap and Positioning of This Work}
Existing ontology efforts in cybersecurity are mature in representing threats,
incidents, and controls; existing hardware-security efforts are mature in
quantifying mechanism-level technical overhead. What remains underdeveloped is a
formal ontology that links these two traditions through explicit cost
semantics: who pays, who benefits, when costs occur, and how residual risk is
transferred.

This paper positions the cost ontology as connective infrastructure across
technical architecture analysis, organizational decision-making, and policy
evaluation. Instead of treating overhead as a single scalar, we model security
cost as a structured, stakeholder- and time-indexed object that can be queried,
compared, and empirically validated. This makes the
``mechanism-as-burden-allocation-instrument'' framing operational rather than
rhetorical.

\subsection{Novelty Boundary Against Adjacent Ontologies}
Table~\ref{tab:novelty-boundary} summarizes how this work relates to common
security ontology and risk-modeling baselines. The intent is not to replace
these models, but to add a missing cost-semantics layer that can interoperate
with them.

\begin{table*}[t]
  \caption{Positioning of the proposed ontology against adjacent frameworks.}
  \label{tab:novelty-boundary}
  \centering
  \small
  \begin{tabular}{p{0.16\textwidth}p{0.22\textwidth}p{0.22\textwidth}p{0.30\textwidth}}
    \toprule
    Framework family & Primary focus & Typical omission for this problem & Contribution of this work \\
    \midrule
    Threat/control ontologies (for example UCO, CASE) & Cyber events, artifacts, incidents, and investigation semantics & Cost-bearing actor, timing of burden realization, and opportunity cost are not first-class & Adds explicit mechanism--cost--stakeholder--time representation for hardware design and operation \\
    Defensive technique knowledge graphs (for example D3FEND) & Defensive technique catalog and relation to adversary behavior & Does not model full lifecycle burden allocation or externalities across organizations & Adds burden semantics that can map onto technique nodes for comparative evaluation \\
    Cyber risk quantification models (for example FAIR-style) & Financial loss estimation and risk factors & Usually weak on hardware/microarchitectural mechanism representation and implementation cost decomposition & Adds engineering and architectural overhead classes tied to threat mitigation and deployment context \\
    Hardware security evaluation papers & Mechanism-local PPA/performance/security measurements & Usually omit cross-stakeholder transfer, compliance burden, and post-incident cost pathways & Adds a shared schema to aggregate local measurements into system-level cost allocation \\
    \bottomrule
  \end{tabular}
\end{table*}

Taken together, this boundary analysis motivates the next section's explicit
class and relation design: the contribution is not a replacement for threat or
risk ontologies, but a cost-semantics layer that can interoperate with them.

\section{Ontology of Security Costs}
\subsection{Core Entities}
\begin{itemize}
  \item \textbf{Asset/Objectives}: performance, power, area, schedule, trust, safety, and revenue.
  \item \textbf{Security Mechanism/Requirement}: technical controls and external mandates.
  \item \textbf{Threat/Failure Mode}: exploit class, vulnerability type, attacker capability, and expected impact.
  \item \textbf{Cost Type}: structured classes of burden with explicit definitions.
  \item \textbf{Stakeholder}: who decides, who pays, who benefits, and who bears residual risk.
  \item \textbf{Time Horizon}: design-time, validation-time, deployment-time, incident-time, and post-incident recovery.
  \item \textbf{Evidence}: benchmark data, field incidents, audit artifacts, and expert elicitation.
\end{itemize}

\subsection{Core Relations}
\begin{itemize}
  \item Mechanism \emph{incurs} cost.
  \item Cost is \emph{borne by} stakeholder.
  \item Cost may be \emph{shifted to} a different stakeholder.
  \item Cost is \emph{realized at} a time horizon.
  \item Mechanism \emph{mitigates} threat.
  \item Requirement \emph{constrains} design space.
  \item Constraint \emph{induces} opportunity cost.
  \item Incident \emph{causes} contingent and reputational losses.
  \item Incentive misalignment \emph{shifts} costs across stakeholders.
\end{itemize}

\subsection{Cost Taxonomy}
We define cost as any measurable reduction in technical, organizational,
economic, or social objective value caused by adopting, operating, or omitting
security controls. The ontology distinguishes the following cost classes:
\begin{enumerate}
  \item \textbf{Physical Resource Cost}: die area, static/dynamic power, latency, throughput, and memory overhead.
  \item \textbf{Microarchitectural Performance Cost}: CPI increases, frequency reductions, and disabled speculative behaviors.
  \item \textbf{Engineering Labor Cost}: architecture/design effort, RTL rework, firmware changes, and security review labor.
  \item \textbf{Verification and Validation Cost}: formal proofs, simulation campaigns, emulation, test generation, and bug triage.
  \item \textbf{Toolchain and Infrastructure Cost}: EDA licenses, CI hardening, fuzzing infrastructure, and secure build pipelines.
  \item \textbf{Lifecycle Operations Cost}: key management, patch deployment, field support, secure update operations, and incident response.
  \item \textbf{Compliance and Assurance Cost}: documentation, audits, certification, and recurring attestation obligations.
  \item \textbf{Opportunity Cost}: schedule slip, delayed feature roadmap, lower SKU flexibility, and foregone market windows.
  \item \textbf{Market and Contractual Cost}: warranty exposure, insurance premiums, indemnification terms, and procurement penalties.
  \item \textbf{Reputation and Trust Cost}: customer churn, reduced adoption, investor confidence loss, and long-tail brand damage.
  \item \textbf{Liability and Redress Cost}: legal defense, settlements, recalls, regulatory fines, and remediation programs.
  \item \textbf{Externality Cost}: harms imposed on non-deciding parties, including downstream operators, end users, and public infrastructure.
\end{enumerate}

\subsection{Cost-Bearing Dimensions}
Each instantiated cost in the ontology is annotated along four dimensions:
\begin{itemize}
  \item \textbf{Bearer}: vendor, integrator/OEM, cloud operator, enterprise customer, end user, regulator/public.
  \item \textbf{Timing}: upfront, recurring, contingent, or deferred.
  \item \textbf{Bearing Mode}: internalized, transferred by contract, insured, or externalized.
  \item \textbf{Evidence Type}: measured, estimated, or expert-elicited.
\end{itemize}

This representation allows the same mechanism to carry different burden profiles
across deployment contexts even when technical overhead appears similar.

\subsection{Cost Manifestation Modes}
In addition to ``who pays,'' the ontology tracks \emph{how} costs manifest in
operation:
\begin{itemize}
  \item \textbf{Activation Profile}: always-on, conditional, or event-triggered cost realization.
  \item \textbf{Scope}: system-wide versus module/workload-specific manifestation.
  \item \textbf{Workload Interaction}: whether the mechanism improves, degrades, or redistributes performance across workload classes.
  \item \textbf{Counterfactual Opportunity Cost}: what alternative capability could have been implemented with the same area/power/budget.
\end{itemize}

This distinction is important for mechanisms with mixed effects. For example, a
protection with near-constant overhead (for example, broad randomization
features) is modeled as always-on and wide-scope, while a dedicated crypto block
may accelerate cryptographic workloads but still induce system-level opportunity
cost by consuming resources that could otherwise support general-purpose
performance.

\subsection{Security Posture: Proactive vs Reactive}
The ontology also classifies mechanisms by security posture:
\begin{itemize}
  \item \textbf{Proactive}: mechanisms that reduce exploitability before compromise, typically via prevention or hardening.
  \item \textbf{Reactive}: mechanisms that detect, contain, or recover from compromise after or during attack execution.
  \item \textbf{Hybrid}: mechanisms with both proactive and reactive components.
\end{itemize}

Posture affects cost incidence. Proactive controls often concentrate burden in
upfront design and recurring overhead, while reactive controls can shift burden
toward monitoring operations, incident-time response, and post-incident recovery.
Modeling posture alongside activation profile and scope helps separate
``preventive cost'' from ``response cost'' in cross-mechanism comparisons.

\subsection{Initial Cost-Bearing Matrix}
Table~\ref{tab:cost-bearing-matrix} summarizes representative burden patterns
that the ontology is designed to capture and query.

\begin{table*}[t]
  \caption{Illustrative cost-bearing matrix by stakeholder and time horizon.}
  \label{tab:cost-bearing-matrix}
  \centering
  \small
  \begin{tabular}{p{0.17\textwidth}p{0.11\textwidth}p{0.11\textwidth}p{0.11\textwidth}p{0.40\textwidth}}
    \toprule
    Stakeholder & Upfront & Recurring & Contingent & Typical burden classes \\
    \midrule
    Chip vendor & High & Medium & Medium & Physical resource, engineering labor, verification and validation \\
    OEM/integrator & Medium & Medium & Medium & Integration overhead, compliance burden, lifecycle operations \\
    Cloud operator & Medium & High & High & Performance overhead, recurring operations, incident response \\
    Enterprise customer & Low & Medium & High & Migration effort, downtime risk, contractual and liability exposure \\
    End user/public & Low & Low & High & Service degradation, privacy harms, and externalized impacts \\
    \bottomrule
  \end{tabular}
\end{table*}

\subsection{Machine-Readable Representation}
The formal class/property encoding of this taxonomy is maintained in the
companion artifact released with the paper materials, which provides canonical
definitions for cost classes, stakeholder roles, and burden relations.

\section{Analytical Framework}
\subsection{Multi-Stakeholder Cost Accounting}
Define total expected cost as a function of direct implementation costs,
probability-weighted incident losses, and cost-shifting terms:
\begin{equation}
\begin{aligned}
C_{\text{total}} =\;& C_{\text{direct}} + \mathbb{E}[L_{\text{incident}}] + C_{\text{compliance}} \\
& + C_{\text{opportunity}} + C_{\text{externalized}}.
\end{aligned}
\end{equation}

\subsection{Risk Transfer and Incentive Misalignment}
A key claim of this draft is that many hardware security decisions optimize
local objective functions while externalizing residual risk, which mirrors
established observations from security economics \cite{anderson2001economics,anderson2006economics}.
The ontology captures these transfers explicitly.

\section{Evidence-Grounded Case Studies}
\subsection{Why Deep Case Mapping Matters}
Case studies are the bridge between ontology semantics and hardware-security
practice. They test whether classes and relations are expressive enough for
real mechanisms, and they expose burden pathways that mechanism-local reporting
usually hides (transfer, externality, and delayed loss effects)
\cite{anderson2001economics,anderson2006economics}.
In other words, they test whether the burden-allocation instrument framing
matches how mechanisms are engineered and deployed.

\subsection{Case Design and Corpus}
The seed corpus now covers six families (20 tuples each): speculation controls,
TEEs, cryptographic accelerators, rowhammer mitigations, memory-safety
mitigations, and reactive runtime detection/response controls. We additionally
encode 12 incident-linkage tuples to connect weak/missing control families to
loss observations with confidence annotations. Each family is treated as a
distinct burden-allocation instrument profile.

\subsection{Cross-Family Snapshot}
Table~\ref{tab:worked-case} summarizes representative anchors and ontology
dimensions across families.

\begin{table*}[t]
  \caption{Cross-family measurements mapped to ontology dimensions.}
  \label{tab:worked-case}
  \centering
  \scriptsize
  \setlength{\tabcolsep}{3pt}
  \begin{tabular}{p{0.14\textwidth}p{0.07\textwidth}p{0.10\textwidth}p{0.13\textwidth}p{0.21\textwidth}p{0.17\textwidth}p{0.10\textwidth}}
    \toprule
    Family & Posture & Activation & Scope & Representative anchors & Primary bearer(s) & Timing \\
    \midrule
    Speculation controls & Proactive & Often always-on & System-wide & Retbleed slowdown 5.78--27.90\% \cite{retbleed2022}; InvisiSpec 21\% mean overhead vs.\ 74\% STT \cite{invisispec2018} & Operator, platform vendor & Recurring \\
    TEE isolation & Proactive & Mixed & Module specific & Sanctum: +0.78\% gates, +1.9\% flip-flops \cite{sanctum2016}; Graphene-SGX: near-native to below 2x \cite{graphenesgx2017} & Vendor, integrator, operator & Upfront + recurring \\
    Crypto acceleration & Proactive & Conditional & Module specific & OpenTitan AES up to 44x \cite{opentitan2024}; AES-NI 2--3x to 10x \cite{intel_aesni2012} & Vendor, integrator & Upfront + recurring \\
    Rowhammer mitigations & Proactive & Mostly always-on & System-wide memory path & Disturbance characterization \cite{kim2014rowhammer}; TRRespass/Blacksmith bypass pressure \cite{trrespass2020,blacksmith2022} & Vendor, operator & Upfront + recurring \\
    Memory safety mitigations & Proactive & Conditional checks & Module/workload-specific & Arm MTE deployment overhead guidance \cite{arm_mte2021}; CHERI migration tradeoffs \cite{cheri2015} & Vendor, integrator, operator & Upfront + recurring \\
    Runtime detection/response & Reactive & Event-triggered & System-wide telemetry/recovery path & Firmware resiliency/recovery workflow requirements \cite{nist800193}; runtime platform instrumentation context \cite{opentitan2024} & Operator, integrator, customer & Recurring + contingent \\
    \bottomrule
  \end{tabular}
\end{table*}

\subsection{Structured Worked Outputs}
To avoid purely narrative treatment, each family is summarized as
(\textit{tuple slice}, \textit{measured anchors}, \textit{inferred/synthetic
allocations}, \textit{decision implication}) in
Table~\ref{tab:worked-structured}.

\begin{table*}[t]
  \caption{Worked case outputs with explicit data slices and decision implications.}
  \label{tab:worked-structured}
  \centering
  \scriptsize
  \setlength{\tabcolsep}{3pt}
  \begin{tabular}{p{0.13\textwidth}p{0.13\textwidth}p{0.20\textwidth}p{0.23\textwidth}p{0.23\textwidth}}
    \toprule
    Family & Tuple slice & Measured anchors (E1/Measured) & Inferred or synthetic allocations (E2/E3) & Decision implication \\
    \midrule
    Speculation controls & \texttt{S01--S20} & Runtime overhead and ops burden anchors from Retbleed \cite{retbleed2022} & Transfer/externalized burdens under alternative mitigation assumptions from InvisiSpec-context rows \cite{invisispec2018} & ``Best local runtime'' and ``lowest transferred burden'' can diverge. \\
    TEE isolation & \texttt{T21--T40} & Hardware increments and selected runtime behavior anchors from Sanctum \cite{sanctum2016} & Integration and downstream burden variation from Graphene-SGX style adaptation contexts \cite{graphenesgx2017} & Upfront provisioning and recurring adaptation must be jointly optimized. \\
    Crypto acceleration & \texttt{C41--C60} & Crypto-path speedups and area footprints from OpenTitan/AES-NI \cite{opentitan2024,intel_aesni2012} & Opportunity and transfer effects for non-crypto workloads and operations \cite{intel_xeon_crypto2017} & Acceleration benefit does not remove system-level opportunity cost. \\
    Rowhammer mitigations & \texttt{R61--R80} & Disturbance and defense-pressure anchors from ISCA/SP evidence \cite{kim2014rowhammer,trrespass2020,blacksmith2022} & Validation refresh and externalized reliability burden in E2/E3 rows & Durability costs can dominate one-time mitigation overhead. \\
    Memory safety & \texttt{M81--M100} & Runtime and deployment anchors from MTE/CHERI sources \cite{arm_mte2021,cheri2015} & Toolchain migration and transfer/externality terms in E2/E3 rows & Adoption economics materially affect mechanism ranking. \\
    Runtime detection/response & \texttt{D101--D120} & None in current seed (deliberately no E1 reactive rows) & Recovery-time burden, contractual spillover, and contingent loss terms from inferred/synthetic rows \cite{nist800193,anderson2006economics} & Reactive controls expose contingent burden that proactive-only views miss. \\
    \bottomrule
  \end{tabular}
\end{table*}

\subsection{Worked Family Notes}
\textbf{Speculation controls (\texttt{S01--S20}).}
Always-on behavior and system-wide scope produce recurring operator burden. The
seed shows decision-delta behavior where minimizing local runtime and
minimizing transferred burden can select different tuples under different
objective weights.

\textbf{TEE isolation (\texttt{T21--T40}).}
TEE rows show mixed manifestation: upfront hardware/verification and recurring
runtime adaptation. This family prevents collapse into a single ``TEE overhead''
scalar by forcing separate burden channels.

\textbf{Crypto acceleration (\texttt{C41--C60}).}
This family encodes both acceleration gains and opportunity costs.
Allocation-aware analysis keeps ``faster crypto path'' and ``foregone
general-purpose capacity'' in the same decision view.

\textbf{Rowhammer mitigations (\texttt{R61--R80}).}
These rows model mitigation durability pressure: bypass research implies
recurring validation and refresh burden even when first-generation controls are
deployed.

\textbf{Memory safety mitigations (\texttt{M81--M100}).}
Rows mix hardware support, toolchain integration, and migration costs. The
family is useful because burden incidence can move between vendor, integrator,
and operator depending on rollout strategy.

\textbf{Runtime detection/response (\texttt{D101--D120}).}
This reactive family captures event-triggered and contingent burden channels
that are absent from proactive-only corpora: incident handling, recovery
operations, contractual penalties, and reputation effects \cite{nist800193}.
It highlights that reactive mechanisms are burden-allocation instruments with
different timing signatures than proactive mechanisms.

\paragraph{Incident linkage for CQ4.}
Incident tuples (\texttt{I01--I12}) encode observed/seeded losses linked to
families with attribution confidence and provenance. This enables executable
loss-linkage queries instead of treating attribution as future work.

\paragraph{Community value.}
The structured format above is reusable: each family can be extended with
additional rows without changing schema, and review can target concrete tuple
claims rather than prose-only argument. That improves comparability across
architecture, systems-security, and policy analyses.

\paragraph{Seed artifact status.}
The current release is still a seed dataset, not a final benchmark corpus:
120 cost tuples, 12 incident tuples, explicit data-origin labels, and full CQ
coverage under executable checks. This is enough for methodological validation,
while still requiring broader source-complete expansion for benchmark-grade
conclusions.

\section{Formal Methods}
\subsection{Methodological Goals}
The methods section serves four goals: (1) make ontology construction
reproducible, (2) justify modeling decisions as answers to explicit research
questions, (3) define objective validation criteria, and (4) connect the
resulting ontology to empirical analysis in hardware-security settings.

\subsection{Ontology Engineering Process}
We follow a competency-question-driven ontology workflow aligned with established
ontology-engineering practice \cite{noy2001ontology,fernandez1997methontology}.
The process has six stages:
\begin{enumerate}
  \item \textbf{Specification}: define scope, stakeholders, intended users, and non-goals.
  \item \textbf{Knowledge acquisition}: collect terms, relations, and measurement concepts from hardware-security and security-economics literature.
  \item \textbf{Conceptualization}: define core classes and relation patterns (for example, mechanism--cost, stakeholder--burden, and threat--impact).
  \item \textbf{Formalization}: encode the model in OWL 2 with explicit object properties, data properties, and constraints.
  \item \textbf{Implementation}: build a versioned machine-readable artifact (Turtle/RDF or OWL/XML) with example instances.
  \item \textbf{Evaluation and iteration}: test queries, refine class boundaries, and resolve inconsistencies.
\end{enumerate}

\subsection{Competency Questions}
Competency questions (CQs) define what the ontology must answer
\cite{gruninger1995methodology}. Initial CQs for this paper are:
\begin{itemize}
  \item \textbf{CQ1}: Which stakeholders bear the direct and indirect costs of a given security mechanism?
  \item \textbf{CQ2}: Which cost categories occur at design time versus deployment time versus incident time?
  \item \textbf{CQ3}: For a given requirement, which costs are internalized versus externalized?
  \item \textbf{CQ4}: Which mechanisms mitigate the same threat class but induce different cost distributions?
  \item \textbf{CQ5}: Which observed incident losses can be mapped to missing or weakly adopted controls?
\end{itemize}

\subsection{CQ-to-Model Mapping}
To make evaluation reproducible, each CQ is mapped to explicit conceptual
elements in the ontology:
\begin{itemize}
  \item \textbf{CQ1} (who bears cost) maps to mechanism--cost--stakeholder relations.
  \item \textbf{CQ2} (when costs occur) maps to cost-class and time-horizon modeling.
  \item \textbf{CQ3} (internalized versus externalized burden) maps to requirement constraints, bearing-mode attributes, and cost-allocation relations.
  \item \textbf{CQ4} (cross-mechanism comparison) maps to shared threat-mitigation links combined with stakeholder-time cost distributions and security posture (proactive/reactive/hybrid).
  \item \textbf{CQ5} (incident loss linkage) is partially covered by threat-mitigation and failure-cost concepts; full coverage requires explicit incident-event classes in the next revision.
\end{itemize}

\subsection{Formalization and Semantics}
The ontology will be represented in OWL 2 DL to support automated consistency
checking while preserving sufficient expressiveness for class restrictions and
property assertions. We separate:
\begin{itemize}
  \item \textbf{Structural concepts}: mechanism, stakeholder, threat, asset, requirement.
  \item \textbf{Cost concepts}: cost type, magnitude, timing, uncertainty, and payer/beneficiary roles.
  \item \textbf{Evidence concepts}: source type, confidence level, and measurement provenance.
\end{itemize}
To reduce taxonomic ambiguity, class hierarchies will be reviewed using
OntoClean-style meta-properties (rigidity, identity, dependence) where
applicable \cite{guarino2009ontoclean}.

\subsection{Representative Axioms}
To make formal commitments explicit, the current draft uses the following
representative constraints:
\begin{itemize}
  \item \textbf{Disjointness}: \textit{ProactivePosture}, \textit{ReactivePosture}, and \textit{HybridPosture} are pairwise disjoint subclasses of \textit{SecurityPosture}.
  \item \textbf{Participation}: every \textit{CostInstance} must have at least one bearer and one realization time.
  \item \textbf{Typing}: values of \textit{hasActivationProfile} are restricted to \textit{AlwaysOn}, \textit{Conditional}, or \textit{EventTriggered}.
  \item \textbf{Transfer semantics}: if a cost is marked \textit{externalized}, there exists a stakeholder distinct from the decision-maker that bears part of the burden.
\end{itemize}

In description-logic style, a minimal fragment is:
\begin{equation}
\begin{aligned}
\textit{CostInstance} &\sqsubseteq \exists \textit{borneBy}.\textit{Stakeholder} \\
\textit{CostInstance} &\sqsubseteq \exists \textit{realizedAt}.\textit{TimeHorizon} \\
\textit{SecurityMechanism} &\sqsubseteq \exists \textit{hasSecurityPosture}.\textit{SecurityPosture}.
\end{aligned}
\end{equation}
These axioms are intentionally lightweight in the draft and can be tightened in
camera-ready versions after additional data mapping.

\subsection{Validation Strategy}
We evaluate the ontology against three criteria:
\begin{enumerate}
  \item \textbf{Logical validity}: ontology consistency and satisfiability under a DL reasoner.
  \item \textbf{Question coverage}: whether CQs can be answered by executable queries.
  \item \textbf{Analytical utility}: whether outputs support comparative assessments of hardware-security mechanisms across stakeholder-time matrices.
\end{enumerate}
For empirical grounding, we map case-study data into normalized tuples
(\textit{mechanism}, \textit{cost type}, \textit{stakeholder}, \textit{time},
\textit{evidence}) and perform sensitivity analysis when probabilities or loss
magnitudes are uncertain.

\subsection{Measurement Protocol}
Each mapped cost instance includes a unit, uncertainty annotation, and evidence
grade to improve reproducibility:
\begin{itemize}
  \item \textbf{Physical/resource metrics}: percent area, watts, performance delta, memory overhead.
  \item \textbf{Labor/process metrics}: engineer-months, verification campaign size, patch turnaround time.
  \item \textbf{Governance metrics}: audit effort hours, certification cycle length, compliance frequency.
  \item \textbf{Loss metrics}: incident frequency, recovery cost, liability outlay, churn proxy.
\end{itemize}

Evidence quality is tracked with a three-level rubric:
\begin{itemize}
  \item \textbf{E1 (measured)}: benchmarked or operationally observed values.
  \item \textbf{E2 (estimated)}: model-based interpolation/extrapolation.
  \item \textbf{E3 (elicited)}: expert-judgment assumptions.
\end{itemize}
All reported comparisons state the proportion of E1/E2/E3 values and include a
sensitivity range when E2 or E3 dominates.

\subsection{Executable CQ Templates}
To demonstrate question coverage, we use query templates over the
machine-readable artifact:

\begin{verbatim}
# CQ1: Who bears costs for a given mechanism?
SELECT ?stakeholder ?costType ?time WHERE {
  ?m a ex:SecurityMechanism ; ex:label "AES accelerator" ;
     ex:incurs ?c .
  ?c ex:borneBy ?stakeholder ;
     ex:hasCostType ?costType ;
     ex:realizedAt ?time .
}
\end{verbatim}

\begin{verbatim}
# CQ4: Compare mechanisms mitigating the same threat
SELECT ?m ?posture ?stakeholder ?costType WHERE {
  ?m ex:mitigates ex:TransientExecutionThreat ;
     ex:hasSecurityPosture ?posture ;
     ex:incurs ?c .
  ?c ex:borneBy ?stakeholder ;
     ex:hasCostType ?costType .
}
\end{verbatim}

\section{Evaluation Results}
\subsection{Executed Workflow}
The current release was evaluated using seven artifacts:
\path{ontology.ttl}, \path{artifacts/cost_tuples.csv},
\path{artifacts/incident_tuples.csv}, \path{artifacts/cq_queries.sparql},
\path{artifacts/cq_results.csv}, \path{artifacts/voi_priorities.csv}, and
\path{artifacts/sensitivity_rankings.csv}. We regenerate CQ outputs from data
with \texttt{scripts/generate\_cq\_results.py} and then run integrity and
distribution checks via \texttt{make repro}.

\subsection{Artifact Integrity Results}
The seed cost dataset contains 120 tuples across six mechanism families (20 per
family) with no missing values in key analysis fields (family, cost type, time
horizon, bearing mode, evidence grade, data origin, source key, source
locator). Distribution is: 59 E1, 39 E2, and 22 E3 rows; 59 Measured, 39
Inferred, and 22 Synthetic rows; 54 upfront, 60 recurring, 4 contingent, and 2
deferred rows; 66 internalized, 34 transferred, and 20 externalized rows.

Incident linkage contains 12 tuples with complete family-linked loss,
confidence, and provenance fields (8 E2/Inferred and 4 E3/Synthetic).

\subsection{Competency-Question Coverage Results}
All six CQs are executable in the current release through generated artifacts.
CQ4 and CQ6 moved from partial to pass because incident-linkage tuples and VOI
ranking are now part of the artifact pipeline.

\begin{table}[t]
  \caption{Observed CQ coverage results from executable artifacts.}
  \label{tab:cq-coverage}
  \centering
  \footnotesize
  \setlength{\tabcolsep}{3pt}
  \begin{tabular}{p{0.18\columnwidth}p{0.10\columnwidth}p{0.60\columnwidth}}
    \toprule
    CQ & Status & Coverage metric \\
    \midrule
    CQ1 (visibility) & Pass & 120/120 tuples include bearer, time, unit, evidence, data origin, and source provenance. \\
    CQ2 (transfer) & Pass & 120/120 tuples include bearing mode (Externalized=20). \\
    CQ3 (comparison) & Pass & 6/6 families satisfy shared comparability criteria. \\
    CQ4 (attribution) & Pass & 12/12 incident tuples include family-linked loss, confidence, and provenance. \\
    CQ5 (sensitivity) & Pass & Family ranking stable under $\pm20\%$ E2/E3 perturbation. \\
    CQ6 (information gap) & Pass & VOI priorities computed for 40 family-cost cells. \\
    \bottomrule
  \end{tabular}
\end{table}

\subsection{RQ1--RQ4 Answer Summary}
Table~\ref{tab:rq-summary} reports direct answers to RQ1--RQ4 with explicit
seed-corpus scope.

\begin{table*}[t]
  \caption{Direct answers to RQ1--RQ4 with evidence anchors in the current seed release.}
  \label{tab:rq-summary}
  \centering
  \small
  \begin{tabular}{p{0.07\textwidth}p{0.44\textwidth}p{0.41\textwidth}}
    \toprule
    RQ & Answer from current results & Evidence anchor \\
    \midrule
    RQ1 & Security mechanisms induce heterogeneous burden classes across lifecycle phases, not a single scalar overhead. & 120 tuples spanning physical, performance, labor, operations, compliance, strategic, and externality costs across upfront/recurring/contingent/deferred horizons. \\
    RQ2 & A usable representation requires bearer, time horizon, bearing mode, evidence grade, data origin, and source provenance per row. & Complete field coverage in \texttt{cost\_tuples.csv} with executable CQ1--CQ2 checks. \\
    RQ3 & Transfer and externality semantics are necessary to model incentive misalignment in hardware-security choice. & 34 transferred and 20 externalized tuples; decision-delta examples where local-performance and allocation-aware choices diverge. \\
    RQ4 & Cross-family comparison is feasible when families share posture, activation, scope, and burden schema. & CQ3 pass across six families, including one reactive family, on a shared burden representation. \\
    \bottomrule
  \end{tabular}
\end{table*}

\subsection{Decision-Delta and Reactive Coverage}
Compared with prior proactive-only coverage, the reactive
runtime-detection/response family adds contingent and deferred burden channels.
This enables explicit proactive-versus-reactive portfolio discussion in the same
schema, instead of treating posture as a descriptive label only.

\subsection{Sensitivity Results (CQ5)}
Microarchitectural performance ranking was stable across baseline and
$\pm20\%$ perturbations of E2/E3 values:
CryptoAccelerators, RuntimeDetectionResponse, MemorySafetyMitigations,
RowhammerMitigations, TrustedExecutionEnvironments, SpeculationControls.
This result is generated and recorded in
\texttt{artifacts/sensitivity\_rankings.csv}.

\subsection{VOI Results (CQ6)}
The VOI pipeline ranks high-value measurement gaps by uncertainty, transfer, and
incident linkage. Top-ranked cells in the current seed release include
RuntimeDetectionResponse/ReputationTrustCost and transfer-heavy crypto,
rowhammer, and memory-safety cells. Full ranking is emitted to
\texttt{artifacts/voi\_priorities.csv}.

\subsection{Calibration of Claims}
These results establish methodological execution, not final empirical
ground truth. The corpus remains a seed dataset with mixed measured and
inferred/synthetic rows, so conclusions should be read as demonstrating
decision visibility and queryability rather than definitive market-level
effect sizes.

\subsection{Reproducibility Outputs}
Running \texttt{make repro} regenerates CQ/VOI/sensitivity artifacts and writes
\texttt{artifacts/repro\_report.txt} with tuple counts, distribution checks, and
CQ status summaries used in this section.

\section{Discussion}
\subsection{Policy and Regulation}
Regulatory requirements can reduce social risk while increasing localized
industry burden; the ontology helps identify when these interventions are
productive versus distortionary.

\subsection{Design Implications}
Architects should evaluate security features by system-level welfare impact,
not only local PPA or benchmark metrics.

\subsection{Limitations}
This draft still has three major limitations: limited measured hardware data,
incomplete incident-loss mapping, and dependence on mixed evidence quality.

\subsection{Threats to Validity}
\begin{itemize}
  \item \textbf{Selection bias}: published mechanisms over-represent successful or benchmark-friendly designs.
  \item \textbf{Measurement heterogeneity}: area, power, and performance numbers are often reported under incompatible assumptions.
  \item \textbf{Attribution ambiguity}: incident losses may be multi-causal and difficult to map to one missing control.
  \item \textbf{Private-cost opacity}: legal, contract, and response costs are frequently proprietary.
\end{itemize}
These risks are mitigated by explicit evidence grading, sensitivity analysis,
and by reporting uncertainty intervals alongside point estimates.

\subsection{Bibliographic and Dataset Maturity}
The current draft cites representative prior work, but a camera-ready version
requires a more systematic corpus review and a source-complete benchmark table
for all numeric claims.

%=============================================================
\section{Conclusion}
\label{sec:conclusion}
%=============================================================

Hardware security mechanisms are not neutral technical controls. They are
instruments that distribute costs, risks, and residual liabilities across
stakeholders, organizations, and timescales --- and they do so as a direct
consequence of specific design choices that could, in principle, have been
made differently. The performance tax from Spectre mitigations did not
accidentally fall on cloud operators; it fell there because the always-on,
system-wide activation profile of the chosen mitigations made operator-borne
recurring overhead structurally inevitable given the organizational structure
of cloud computing. The ecosystem migration burden of CHERI did not
accidentally fall on software maintainers; it fell there because capability-
wide pointer enforcement requires every piece of pointer-manipulating code
to be ported, and that code lives in the ecosystem, not in the vendor's
design team.

These distributions are not mysteries. They are readable from the technical
properties of the mechanisms themselves, once you have the vocabulary to
read them. That vocabulary is what this paper provides.

The synthesis we have developed brings together two traditions that have been
analyzing the same phenomenon from incompatible angles. The systems engineering
tradition characterizes mechanism costs with precision but treats distribution
as outside its scope. The security economics tradition characterizes
distribution clearly but treats the mechanism as a black box. Neither can
trace burden pathways through the mechanism --- from specific technical design
choices to specific distributional outcomes. The burden-allocation vocabulary
fills the gaps in both traditions simultaneously: it gives the engineering
analysis the distributional concepts it lacks, and gives the economics analysis
the mechanism-level causal structure it lacks.

The result is not merely a richer description of what has already happened.
It is a framework for prospective analysis: given a proposed mechanism, what
activation profile does it imply, what scope, and therefore who will bear its
recurring cost? What integration obligations does it transfer to downstream
actors, and are those actors the ones who made the decision? What systemic
risks does it externalize to parties who have no visibility into the quality
of what they are receiving? These questions can be asked before deployment,
and they should be --- by designers, by procurement officers, by regulators,
and by the research community that produces the benchmark literature on which
all of them rely.

The fog of war around hardware security costs is real, consequential, and
partly maintained. It persists because the two traditions that could together
dispel it have not been in conversation, and because the actors who benefit
from opacity have little incentive to supply the missing vocabulary voluntarily.
Clearing it requires not only the conceptual framework this paper provides but
the institutional will to require burden-allocation disclosure as a standard
condition of evaluation, procurement, and certification.

Hardware security is not just an engineering problem and not just an economics
problem. It is both, simultaneously, and the failure to treat it as both is
why the costs keep ending up somewhere unexpected.

\bibliographystyle{ACM-Reference-Format}
\bibliography{refs}

\end{document}
