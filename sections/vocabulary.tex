%=============================================================
\section{A Burden-Allocation Vocabulary}
\label{sec:vocabulary}
%=============================================================

The two case studies were not chosen to illustrate a pre-built framework.
They were chosen to reveal, inductively, what concepts are necessary to connect
the engineering and economics analyses of the same mechanisms. By the end of
Section~\ref{sec:spectre}, five concepts had emerged as load-bearing: activation
profile, scope, bearer, timing, and transfer. By the end of
Section~\ref{sec:memsafety}, those five had proven sufficient to account for
four mechanisms with sharply different burden pathway signatures, with one
refinement: the bilateral/diffuse distinction within transfer.

This section crystallizes those concepts into a reusable schema. The vocabulary
is not a translation between two languages that happen to use different words for
the same things. It fills gaps that neither tradition currently has words for at
all. The engineering tradition has precise terms for mechanism costs but no terms
for where those costs go. The economics tradition has precise terms for who bears
costs but no terms for why specific technical properties produce specific
distributions. The vocabulary provides the missing concepts in both directions
simultaneously --- which is what makes it a synthesis rather than a glossary.

We present it as a structured set of entity types, typed relations, and
constraints: precise enough to support consistent analysis across mechanisms and
stakeholders, light enough that practitioners in either tradition can apply it
without specialist ontology tooling. Those who wish to encode it formally in
OWL 2 DL will find the mapping straightforward; that encoding is a companion
artifact rather than a burden on the main text.

%-------------------------------------------------------------
\subsection{The Missing Slots}
\label{sec:vocab-motivation}
%-------------------------------------------------------------

The fog of war around hardware security costs persists not primarily because
the two traditions use different words, but because they report different
\emph{fields}. A benchmark paper fills in: mechanism name, workload, overhead
magnitude, hardware platform. It does not fill in: who bears the overhead,
under what organizational conditions, for how long, and as a consequence of
whose decision. An economics analysis fills in: which actor ended up paying,
what incentive misalignment explains this. It does not fill in: which specific
technical properties of the mechanism made this distribution structurally
likely, and whether an alternative design would have implied a different one.

Neither set of missing fields is exotic. Both are answerable given the right
vocabulary. The reason they go unfilled is that no shared schema defines them
as required --- so they are omitted not because they are unknown but because
there is no standard slot to put them in.

A vocabulary, in this sense, is a commitment about what information is
relevant and what omissions are meaningful. A schema that requires every cost
entry to name a bearer, a decision-maker, a lifecycle phase, and a transfer
structure makes it impossible to report overhead without answering the
distributional question. The fog of war is cleared not by more benchmarks but
by changing what any analysis is required to say.

%-------------------------------------------------------------
\subsection{Entity Types}
\label{sec:entities}
%-------------------------------------------------------------

The vocabulary requires four entity types.

\paragraph{Security Mechanism.}
A technical control implemented in hardware, firmware, or system software that
reduces the exploitability of a threat or the impact of a successful attack.
Mechanisms are the primary unit of analysis. Each mechanism carries two
engineering properties that the case studies showed to be analytically
essential: \textbf{activation profile} (always-on, conditional, or opt-in)
and \textbf{scope} (system-wide, module-specific, or workload-selective).
These are not incidental implementation details --- they are the engineering
inputs that determine burden pathway signatures, as the Spectre and memory
safety case studies demonstrated. A mechanism description that omits them
cannot support burden-allocation analysis.

\paragraph{Cost Instance.}
A specific, attributable reduction in objective value caused by adopting,
operating, or omitting a security mechanism. Cost instances are the atomic
units of burden-allocation analysis. A mechanism may have many cost
instances --- one for each combination of cost type, bearer, and lifecycle
phase. The central design commitment of this entity type is that a cost
instance without a named bearer is \emph{incomplete}: it reports an overhead
without answering who absorbs it. This is the field the benchmark literature
systematically omits.

\paragraph{Stakeholder.}
An organizational actor who bears, decides, or benefits from a cost instance.
The vocabulary separates four roles that are frequently conflated but analytically
distinct:
\begin{itemize}
  \item \textbf{Decision-maker}: the actor whose design or deployment choice
    incurred the cost.
  \item \textbf{Bearer}: the actor who absorbs the cost in practice.
  \item \textbf{Beneficiary}: the actor who receives the security property the
    mechanism provides.
  \item \textbf{Residual risk holder}: the actor who bears the consequences if
    the mechanism fails or is bypassed.
\end{itemize}

Separating these roles is where the vocabulary does its most important work.
In the Spectre case, the decision-maker (CPU vendor), the bearer (cloud
operator), the beneficiary (end user), and the residual risk holder (enterprise
customer) are four different parties. The economics tradition observes the
bearer; the engineering tradition models the decision-maker's constraints.
Neither tradition has a standard way to represent all four roles simultaneously
and ask what happens when they diverge. The vocabulary does.

Concrete stakeholder classes in hardware security include: chip vendor, OEM
or platform integrator, cloud or datacenter operator, enterprise customer,
end user or public, standards body or regulator, and software ecosystem
(toolchain maintainers, OS vendors, application developers considered
collectively).

\paragraph{Lifecycle Phase.}
The point at which a cost is realized: \textbf{design} (during hardware or
software development); \textbf{integration} (when incorporating the mechanism
into a platform); \textbf{operation} (recurring costs during normal use); and
\textbf{incident} (contingent costs upon a security failure). Lifecycle phase
interacts with bearer in ways the benchmark literature cannot capture: the same
mechanism may have its design costs borne by the vendor, its integration costs
by the OEM, its operational costs by the operator, and its incident costs by
the enterprise customer --- four different actors across four phases, none of
whom appears in a standard overhead figure.

%-------------------------------------------------------------
\subsection{Cost Taxonomy}
\label{sec:cost-taxonomy}
%-------------------------------------------------------------

Cost instances are typed across five categories. The taxonomy is motivated
by the case studies: each category corresponds to a class of costs that
appeared in the analysis and that existing reporting frameworks handle poorly
or not at all.

\textbf{Physical resource costs}: die area, power, memory bandwidth, latency.
These are what the benchmark literature measures most precisely and what the
economics literature treats as a black box.

\textbf{Engineering process costs}: labor and time in design, verification,
toolchain development, or ecosystem migration. CHERI's dominant cost is in
this category; it is entirely absent from hardware performance benchmarks.

\textbf{Operational costs}: recurring costs during system use --- runtime
performance overhead, patch deployment and maintenance, key management,
monitoring infrastructure. Spectre mitigation costs fall here; the benchmark
literature measures their magnitude but not their bearer.

\textbf{Compliance and governance costs}: audit effort, certification cycles,
documentation obligations, contractual liability. Largely invisible in the
technical literature; significant in real procurement decisions.

\textbf{Strategic and externality costs}: opportunity costs, reputational
exposure, market effects, and costs imposed on parties outside the direct
decision chain. The opportunity cost of die area committed to MTE rather than
cache capacity falls here, as does Intel's reputational exposure from Meltdown.
These costs are the most consequential for long-run ecosystem dynamics and the
least represented in either tradition's standard analysis.

%-------------------------------------------------------------
\subsection{Core Relations}
\label{sec:relations}
%-------------------------------------------------------------

Relations are where the vocabulary does its analytical work --- they are the
typed links that make burden pathways traceable rather than merely nameable.
Six relations are required.

\medskip
\noindent\textbf{incurs}(\textit{mechanism}, \textit{cost instance}): a
mechanism gives rise to a cost. Every cost must be traceable to the mechanism
that incurred it.

\medskip
\noindent\textbf{borne\_by}(\textit{cost instance}, \textit{stakeholder}):
a cost is absorbed by a stakeholder in the bearer role. This is the relation
most consistently absent from existing technical evaluations. A cost instance
without a \textbf{borne\_by} relation is, by the vocabulary's constraints,
an incomplete analysis.

\medskip
\noindent\textbf{decided\_by}(\textit{cost instance}, \textit{stakeholder}):
the decision that incurred this cost was made by this stakeholder. When
\textbf{decided\_by} $\neq$ \textbf{borne\_by}, a transfer has occurred.
This relation is what the economics tradition tracks under the heading of
principal-agent misalignment; the vocabulary makes it explicit and
mechanism-specific rather than a general market observation.

\medskip
\noindent\textbf{benefits}(\textit{cost instance}, \textit{stakeholder}):
the security property purchased by this cost accrues to this stakeholder.
When \textbf{benefits} $\neq$ \textbf{borne\_by}, the mechanism produces a
cross-subsidy: one actor pays for another's protection. This is the relation
that makes the opt-in deployment problem visible: ASan's diffuse transfer
means the actors who pay for memory safety are not the actors most in need
of it.

\medskip
\noindent\textbf{realized\_at}(\textit{cost instance}, \textit{lifecycle phase}):
the cost is realized at this phase. Combined with \textbf{borne\_by}, this
identifies not just who pays but when --- essential for comparing mechanisms
whose costs have the same magnitude but different timing structures.

\medskip
\noindent\textbf{transferred\_to}(\textit{cost instance}, \textit{stakeholder}):
used when \textbf{decided\_by} $\neq$ \textbf{borne\_by} to name the transfer
target explicitly and annotate its type:
\begin{itemize}
  \item \textbf{Bilateral transfer}: the transfer target is a specific,
    identifiable actor whose burden is traceable to the originating decision.
    CHERI's ecosystem migration cost is bilateral: the hardware vendor's
    architectural decision created an identifiable liability for software
    maintainers.
  \item \textbf{Diffuse transfer}: the burden falls on whoever opts into the
    mechanism, dispersing accountability across the ecosystem with no traceable
    path back to the originating decision. ASan's opt-in deployment is diffuse:
    no specific actor is responsible for whether any given software project runs
    with memory safety instrumentation.
\end{itemize}

This distinction matters beyond taxonomy. Bilateral transfer creates a
contestable liability pathway --- an actor who bears a cost can identify who
created it and potentially seek redress or renegotiation. Diffuse transfer
dissolves accountability: the burden is real but unattributable, which is
precisely the condition under which it tends to fall on actors with the least
capacity to resist it.

%-------------------------------------------------------------
\subsection{Constraints}
\label{sec:constraints}
%-------------------------------------------------------------

A vocabulary without constraints is a glossary. The constraints define what
a \emph{complete} burden-allocation analysis looks like --- and therefore
make omissions visible as omissions rather than as absences.

\begin{enumerate}
  \item Every cost instance must have at least one \textbf{borne\_by} relation.
    A cost without a named bearer is an incomplete analysis, not a cost-free
    mechanism.
  \item Every cost instance must have a \textbf{realized\_at} lifecycle phase.
    Undated costs cannot be compared across mechanisms with different timing
    structures.
  \item Every cost instance must have a \textbf{decided\_by} relation. This
    forces identification of who made the choice that incurred the cost ---
    the precondition for asking whether decision and burden are aligned.
  \item When \textbf{decided\_by} $\neq$ \textbf{borne\_by}, a
    \textbf{transferred\_to} relation must be present with transfer type
    annotated. Transfer is not an exceptional observation; it is a structural
    feature to be reported when present.
  \item Every mechanism must declare an activation profile and a scope. These
    are the engineering inputs that determine burden pathway signatures and
    must be reported alongside overhead magnitudes.
\end{enumerate}

The constraints are intentionally minimal. They do not require precise cost
quantification --- many hardware security costs are genuinely hard to measure,
and demanding quantification would make the schema unusable. What they require
is \emph{attribution}: every cost connected to a bearer, a decision-maker, a
timeline, and a transfer structure. Attribution without magnitude is vastly
more useful than magnitude without attribution, because attribution is what
makes the distributional question answerable at all.

%-------------------------------------------------------------
\subsection{What the Vocabulary Exposes That Neither Tradition Could}
\label{sec:vocab-exposes}
%-------------------------------------------------------------

It is worth pausing to be explicit about what the vocabulary adds beyond
what either tradition provides on its own --- since the case for a synthesis
stands or falls on this question.

The engineering tradition, applied alone to Spectre mitigations, tells you
that the overhead is 5--28\%, always-on, system-wide. It cannot tell you who
bears that overhead in a multi-tenant cloud environment, why the decision-maker
(CPU vendor) and the bearer (cloud operator) are different parties, or whether
an alternative mitigation design would have produced a different distribution.
Those questions are outside its scope by design.

The economics tradition, applied alone to Spectre, tells you that cloud
operators bore costs created by CPU vendor decisions, that this reflects
principal-agent misalignment, and that information asymmetry delayed the
market response. It cannot tell you why the always-on activation profile
and system-wide scope made operator-borne recurring overhead structurally
inevitable given the chosen mitigation approach, or what specific engineering
alternatives would have changed the distribution. Those answers require
opening the black box.

The vocabulary, applied to Spectre, connects these: the always-on activation
profile (engineering concept, now a required schema field) is the causal
mechanism that produces the recurring externality (economics concept) borne
by operators (schema field: bearer). The chain from design choice to
distributional outcome is now explicit and traversable. And crucially, the
chain runs in both directions: prospectively, a designer can ask what activation
profile a proposed mitigation implies and therefore who will bear its recurring
cost; retrospectively, an analyst can ask why the costs landed where they did
and trace the answer to specific technical decisions.

This bidirectionality --- prospective as well as retrospective analysis --- is
what neither tradition alone supports and what the vocabulary makes possible.
It is also what makes the vocabulary a practical tool rather than a post-hoc
description: the burden pathway can be analyzed before deployment, not only
after.

The following section demonstrates this in a worked example, then draws out
the broader implications.