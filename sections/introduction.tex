\section{Introduction}
Hardware security decisions are frequently framed as technical tradeoffs (for
example, performance versus protection), but this framing is incomplete.
Security costs are distributed unevenly across actors (chip vendors, OEMs,
cloud operators, users, and regulators) and across time (upfront design effort
versus contingent incident losses). This paper develops a structured ontology to
unify these costs and make tradeoffs explicit.

In this paper, \emph{ontology} can be read as a rigorous shared data model:
named entity types (for example mechanism, stakeholder, cost) and typed
relationships between them. For readers with a hardware-systems background, the
closest intuition is a normalized schema plus explicit integrity constraints and
query semantics. The goal is not philosophical classification; the goal is
decision support.

\subsection{Research Questions}
This paper is guided by four research questions:
\begin{itemize}
  \item \textbf{RQ1}: What cost types are induced by hardware-security mechanisms across the design, deployment, and incident lifecycle?
  \item \textbf{RQ2}: How should these costs be represented so that stakeholder, time horizon, and uncertainty are explicit?
  \item \textbf{RQ3}: Which relations are required to reason about risk transfer, externalities, and incentive misalignment?
  \item \textbf{RQ4}: How can the ontology support comparable analysis across mechanisms with different technical overhead and governance context?
\end{itemize}

\subsection{Contributions}
\begin{itemize}
  \item A paradigm claim that hardware security mechanisms should be evaluated as both technical controls and risk-allocation instruments.
  \item A lifecycle-oriented taxonomy of security costs in hardware, spanning physical resource, engineering process, compliance, strategic, and externality costs.
  \item A formal ontology design that links mechanisms, threats, stakeholders, cost-bearing pathways, and incident-loss linkage with explicit temporal semantics.
  \item A multi-stakeholder analytical framing that combines direct implementation costs, expected incident loss, and cost externalization.
  \item An executable artifact pipeline that regenerates CQ outcomes, sensitivity checks, and VOI priorities from seed datasets with explicit data-origin labels.
\end{itemize}

The remainder of the paper introduces scope and prior work, defines the ontology
and analytical framework, presents evidence-grounded case studies, and evaluates
coverage and utility against the competency-question set.
