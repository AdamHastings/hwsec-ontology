\section{Introduction}
Hardware security decisions are frequently framed as technical tradeoffs (for
example, performance versus protection), but this framing is incomplete.
Security costs are distributed unevenly across actors (chip vendors, OEMs,
cloud operators, users, and regulators) and across time (upfront design effort
versus contingent incident losses). This paper develops a structured ontology to
unify these costs and make tradeoffs explicit.

\subsection{Research Questions}
This paper is guided by four research questions:
\begin{itemize}
  \item \textbf{RQ1}: What cost types are induced by hardware-security mechanisms across the design, deployment, and incident lifecycle?
  \item \textbf{RQ2}: How should these costs be represented so that stakeholder, time horizon, and uncertainty are explicit?
  \item \textbf{RQ3}: Which relations are required to reason about risk transfer, externalities, and incentive misalignment?
  \item \textbf{RQ4}: How can the ontology support comparable analysis across mechanisms with different technical overhead and governance context?
\end{itemize}

\subsection{Contributions}
\begin{itemize}
  \item A lifecycle-oriented taxonomy of security costs in hardware, spanning physical resource, engineering process, compliance, strategic, and externality costs.
  \item A formal ontology design that links mechanisms, threats, stakeholders, and cost-bearing pathways with explicit temporal semantics.
  \item A multi-stakeholder analytical framing that combines direct implementation costs, expected incident loss, and cost externalization.
  \item A set of competency-question-driven evaluation goals for testing whether the ontology supports policy and architecture decision use cases.
\end{itemize}
