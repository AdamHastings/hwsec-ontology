\section{Introduction}
Hardware and systems security mechanisms are often evaluated as local technical
tradeoffs (for example, performance, power, and area versus protection), but
this view is incomplete. In practice, security costs are distributed across
stakeholders and across time: a control that looks efficient at design time can
still shift recurring operations burden, compliance burden, or contingent
incident loss to other actors later in the lifecycle.

This gap creates \emph{allocation opacity}. Architecture teams, operators, and
policy actors routinely make high-impact choices under uncertainty, yet most
published evidence is reported as mechanism-local overhead rather than who
bears which burdens, when they appear, and who receives the benefits. As a
result, key roles---who decides, who pays, who benefits, and who bears residual
risk---are often missing from the decision record. In this work, we
operationalize allocation opacity as missing or untestable burden-allocation
fields and relations, then evaluate its reduction using executable CQ checks.

Opacity persists in part because it is convenient. Vendors can market
``secure-by-default'' features while downstream integrators and operators pay
integration and runtime costs. Organizations can justify controls as
``required'' without recording the counterfactual---what capability was
foregone under schedule, die-area, or power constraints. Compliance regimes can
reward documentation that a control exists more than measurement of its
marginal overhead or risk reduction. After incidents, ambiguity about whether a
control was absent, weakly adopted, or bypassed can diffuse accountability.
These incentive patterns make cost shifting common but hard to see
\cite{anderson2001economics,anderson2006economics,akerlof1970lemons,hastings2020wac}.

This paper presents an ontology for costs in hardware and systems security,
framed around the claim that security mechanisms act as
\emph{burden-allocation instruments}. Here, ``ontology'' can be read as a
rigorous shared data model: named entity types (mechanisms, stakeholders,
costs), typed relations between them (incurs, borne by, shifted to, benefits),
and constraints that prevent common omissions (for example, every cost entry
must name at least one bearer and one time horizon). The goal is not
classification for its own sake; the goal is decision support: a representation
that makes burden pathways and counterfactuals explicit and queryable.

Building on that model, the paper defines an analytical framework that compares
mechanisms under two decision logics: local-overhead optimization and
allocation-aware optimization. This makes objective functions, baseline
assumptions, and transfer/externality terms explicit, so disagreements about
mechanism choice can be traced to assumptions instead of hidden in prose.

The paper then grounds the model in an executable artifact pipeline and a seed
corpus spanning six mechanism families (speculation controls, TEEs, crypto
acceleration, rowhammer mitigations, memory-safety mitigations, and reactive
runtime detection/response), plus incident-linkage tuples. Competency questions
are used as acceptance tests for visibility, transfer semantics, comparison,
attribution structure, sensitivity, information-gap ranking, semantic
consistency checks, and opportunity-cost explicitness.

Finally, we report what this seed release establishes and what it does not. It
establishes methodological feasibility for allocation-aware, ontology-backed
security-cost analysis; it does not claim final benchmark-grade effect sizes.
The contribution is therefore a decision framework and executable methodology
for exposing burden pathways that local-overhead analyses routinely miss.

\paragraph{Scope.}
In this paper, security cost includes direct design overhead (area and energy),
verification and validation effort, software/hardware co-design complexity,
certification and compliance burden, opportunity cost from constrained design
space, and post-incident costs (recall, liability, and reputational damage)
\cite{anderson2001economics,anderson2006economics}. The system boundary covers
security mechanisms and requirements that affect microarchitecture and SoC
design, platform firmware and trusted boot chains, supply-chain assurance and
lifecycle support, reactive detection/recovery pipelines, and product-level
organizational decision-making.

\paragraph{Non-goals.}
To keep the first version tractable, we do not attempt a full treatment of
software-only control economics without hardware or firmware coupling,
geopolitical externalities beyond product and operator contexts, attacker-utility
modeling in full game-theoretic detail, or jurisdiction-specific legal doctrine
for liability allocation.

\subsection{Research Questions}
This paper is guided by four research questions:
\begin{itemize}
  \item \textbf{RQ1}: What cost types are induced by hardware-security mechanisms across the design, deployment, and incident lifecycle?
  \item \textbf{RQ2}: How should these costs be represented so that stakeholder, time horizon, and uncertainty are explicit?
  \item \textbf{RQ3}: Which relations are required to reason about risk transfer, externalities, and incentive misalignment?
  \item \textbf{RQ4}: How can the ontology support comparable instrument-level analysis across mechanisms with different technical overhead and governance context?
\end{itemize}

\subsection{Contributions}
\begin{itemize}
  \item A central framing claim that hardware security mechanisms should be evaluated as burden-allocation instruments, not only as local technical controls.
  \item A lifecycle-oriented taxonomy of security costs in hardware, spanning physical resource, engineering process, compliance, strategic, and externality costs.
  \item A formal ontology design that links mechanisms, threats, stakeholders, cost-bearing pathways, and incident-loss linkage with explicit temporal semantics.
  \item A multi-stakeholder analytical framing that combines direct implementation costs, expected incident loss, and cost externalization.
  \item An executable artifact pipeline that regenerates CQ outcomes, sensitivity checks, and VOI priorities from seed datasets with explicit data-origin labels.
\end{itemize}

The remainder of the paper reviews prior work, defines the ontology and
analytical framework, presents evidence-grounded case studies, and evaluates
coverage and utility against the competency-question set.
