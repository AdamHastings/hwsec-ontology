
%=============================================================
\section{Introduction}
%=============================================================

Every security mechanism hides a distributional question its designers rarely ask
explicitly: \emph{who pays?} The performance tax from speculative execution mitigations
fell almost entirely on cloud operators who had no role in the original microarchitectural
decisions~\cite{kocher2019spectre,koruyeh2022retbleed}. The integration burden from
capability-based memory safety falls on software ecosystems retrofitting decades of
existing code~\cite{woodruff2014cheri}. The compliance overhead from trusted boot
requirements falls on OEMs navigating certification timelines they did not
set~\cite{cooper2018sp800193}. In each case, the actor who chose the mechanism,
the actor who paid for it, and the actor who received its protection are three different
parties.

This is not an occasional anomaly. It is the normal condition of hardware and systems
security. Mechanisms are evaluated and selected within one organizational context
while their costs materialize---sometimes years later---in another. The result is a
persistent \emph{fog of war} around security costs: burden pathways are implicit,
allocation decisions are unrecorded, and accountability for downstream costs is
systematically diffuse.

Hastings and Sethumadhavan~\cite{hastings2020wac} observed that hardware security
is best understood as a burden to be distributed across Vendors, Users, Authorities,
and Attackers, and that failures persist when that distribution goes wrong. This is
a useful framing. But their doctrine does not connect to the security economics
literature, does not treat security mechanisms as the instruments through which
burden is allocated, and does not provide a vocabulary for tracing \emph{why}
specific technical choices produce specific distributions. Those are the gaps this
paper addresses.

\paragraph{The gap this paper addresses.}
Two traditions have been analyzing the same underlying problem with incompatible
vocabularies. The systems engineering tradition characterizes mechanism costs with
precision---power, performance, area overhead, verification burden---but treats
distribution as outside its scope. The security economics tradition characterizes
distribution clearly---externalities, principal-agent misalignment, cost transfer---but
treats the mechanism as a black box. Neither can trace burden pathways
\emph{through} the mechanism: from specific technical design choices to specific
distributional outcomes and back.

This gap matters practically. Without it, burden-allocation analysis can identify who
ended up paying after the fact but cannot evaluate prospectively whether a proposed
mechanism will concentrate costs on actors who had no say in the design, or whether
an alternative design would distribute burden more equitably or efficiently. The fog
of war persists not because we lack either tradition, but because we lack the
translation layer between them.

\paragraph{This paper.}
We propose a synthesis organized around a central claim: \emph{hardware security
mechanisms are burden-allocation instruments}. Their technical properties do not
merely determine performance and security outcomes---they determine who pays, when,
and how much, as a direct consequence of design choices that could in principle have
been made differently. Mechanism evaluation that reports only local overhead is
therefore incomplete; it omits the distributional consequences that are often the
dominant factor in real deployment decisions.

The synthesis has three components. First, we show that the engineering concept of
a design tradeoff and the economic concept of a cost externality are descriptions of
the same phenomenon from different angles, and that recognizing this equivalence
changes what questions we ask. Second, we develop a shared vocabulary---inductively,
from case studies---that maps engineering concepts onto economic concepts and makes
burden pathways explicit and queryable. Third, we demonstrate that the combined
framework reveals things neither tradition can see alone: specifically, that mechanism
properties \emph{determine} burden distributions prospectively, not merely
retrospectively.

We ground the argument in two case studies. The first---speculative execution
controls---provides a deep analysis of a single mechanism family whose burden
pathway is historically documented and consequential. The second---memory safety
mechanisms---is explicitly comparative: four technically distinct approaches to the
same security objective, with sharply different burden pathway signatures,
stress-testing the vocabulary and showing that distributional differences follow
directly from differences in mechanism design.

\paragraph{Contributions.}
\begin{itemize}
  \item A synthesis of systems engineering and security economics that closes the
    gap identified above: a shared vocabulary for tracing burden pathways through
    technical mechanism properties to distributional outcomes.
  \item A mechanism-level analysis that explains \emph{why} specific technical
    choices produce specific burden distributions---and what alternative designs
    would imply---extending prior burden-allocation framing~\cite{hastings2020wac}
    with both economic grounding and engineering precision.
  \item Two grounded case studies showing that allocation-aware mechanism analysis
    changes evaluative conclusions relative to local-overhead analysis alone.
  \item A semi-formal ontology crystallizing the shared vocabulary, providing a
    reusable schema for burden-allocation analysis across hardware security contexts.
\end{itemize}

\paragraph{Scope.}
We focus on hardware and systems security mechanisms: microarchitectural controls,
trusted execution environments, memory safety hardware, and related platform
features. We do not treat software-only security economics without hardware coupling,
geopolitical supply-chain externalities, or full game-theoretic attacker modeling.

The remainder proceeds as follows. Section~\ref{sec:traditions} develops the
two-traditions framing and identifies the specific gap. Sections~\ref{sec:spectre}
and~\ref{sec:memsafety} present the case studies. Section~\ref{sec:vocabulary}
crystallizes the shared vocabulary. Section~\ref{sec:synthesis} draws out what the
synthesis reveals. Section~\ref{sec:limitations} addresses limitations, and
Section~\ref{sec:conclusion} concludes.