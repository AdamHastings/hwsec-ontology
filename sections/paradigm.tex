\section{Conceptual Framework and Testable Claims}
\subsection{Core Claim}
The paper's paradigm claim is that hardware-security mechanisms are not only
technical controls; they are also burden-allocation instruments. Mechanism
evaluation is therefore incomplete when it reports only local performance/power/
area impacts and omits who bears lifecycle burden and when
\cite{anderson2001economics,anderson2006economics}.

\subsection{Assumptions}
The claim relies on three assumptions. First, burden pathways are part of the
evaluation target, not post-hoc commentary. Second, transfer and externality
terms are common enough in real deployments to matter for choice. Third,
uncertainty should be explicit and auditable rather than hidden inside point
estimates.

\subsection{Falsifiability}
The claim is rejected if repeated studies show all three of the following:
allocation-aware analysis does not change practical choices, transfer/externality
paths are empirically negligible, and uncertainty-aware reporting does not
improve decision robustness. These criteria are used in this paper as explicit
failure conditions, not rhetorical caveats.
