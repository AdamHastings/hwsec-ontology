%=============================================================
\section{Limitations and Future Work}
\label{sec:limitations}
%=============================================================

This paper establishes a conceptual framework and demonstrates its analytical
productivity across six mechanisms. It does not claim to have finished the job.
Several limitations are worth stating explicitly, both to bound the contribution
honestly and to identify the most productive directions for follow-on work.

\paragraph{Inductive scope.}
The vocabulary was developed inductively from a small number of case studies ---
speculative execution controls, four memory safety mechanisms, secure boot, and
hardware random number generation. The concepts that emerged as load-bearing
(activation profile, scope, bearer, timing, transfer, and the bilateral/diffuse
distinction) may not be sufficient to account for the full breadth of hardware
security mechanisms. Mechanisms with unusual deployment models, novel supply
chain structures, or emerging threat contexts may require additional entity
types or relations. The vocabulary should be treated as a well-motivated
starting point, not a finished taxonomy.

\paragraph{Data availability.}
Many of the most consequential burden terms are proprietary or unobserved.
Vendor engineering costs, OEM integration costs, enterprise operational costs,
and incident response expenditures are rarely disclosed in forms that support
rigorous analysis. The schema can be partially populated from public sources ---
benchmark papers, incident reports, regulatory filings --- but a complete
burden-allocation analysis for any real mechanism would require primary data
from vendors and integrators who have strong incentives not to provide it.
This is not a flaw in the vocabulary; it is a reflection of the same allocation
opacity the vocabulary is designed to expose. But it does mean that the
worked examples in this paper are illustrative rather than exhaustive, and that
empirical validation of the framework will require either cooperative disclosure
or natural experiments where costs become visible involuntarily.

\paragraph{Attacker modeling.}
The vocabulary models burden distribution among defenders --- vendors, integrators,
operators, users, and regulators --- but does not model attacker costs or the
strategic interaction between defenders and attackers. Hastings and
Sethumadhavan~\cite{hastings2020wac} include attackers as a fourth player whose
cost-bearing is a legitimate policy lever; our framework sets this aside
deliberately to keep the scope tractable. A fuller treatment would incorporate
attacker economics --- the cost of exploitation, the value of attack targets,
and the asymmetries between offense and defense --- into the burden-allocation
model. This is a meaningful omission for mechanisms whose design explicitly
aims to raise attacker cost, such as hardware-enforced control flow integrity.

\paragraph{Temporal dynamics.}
The vocabulary represents burden pathways as relatively static structures,
but real burden distributions shift over time. The Spectre mitigation burden
on cloud operators will diminish as in-silicon fixes propagate through the
installed base; CHERI's ecosystem migration cost will shift as software
toolchains mature; secure boot's key management burden will evolve as
automation improves. The current schema captures lifecycle phases but does
not model how burden distributions change as mechanisms age, as bypasses are
discovered, or as the ecosystem adapts. A dynamic extension of the vocabulary
would support longitudinal analysis and strengthen the prospective claim.

\paragraph{Formal encoding.}
The vocabulary is presented here in structured prose and semi-formal notation.
A full OWL 2 DL encoding would enable automated consistency checking,
entailment-based inference, and interoperability with existing cybersecurity
ontologies such as UCO and D3FEND. This encoding is a natural next artifact
but is beyond the scope of the present paper. The conceptual commitments
necessary to produce it are fully specified here; the encoding itself is
straightforward for practitioners familiar with description logic tooling.

\paragraph{Future work.}
The most productive near-term directions are three. First, empirical validation:
applying the schema to a broader corpus of mechanisms using primary data from
vendors, integrators, and operators, ideally in a context where disclosure
incentives are aligned --- such as a procurement process that requires
burden-allocation transparency as a condition of evaluation. Second, procurement
tooling: translating the schema into a practical disclosure template that
procurement officers can require vendors to complete, analogous to a
nutritional label for security mechanism costs. Third, regulatory application:
engaging with standards bodies such as NIST, JEDEC, and relevant IEEE working
groups to incorporate burden-allocation fields into existing evaluation and
certification frameworks. The last of these is the highest-leverage intervention
--- if burden-allocation disclosure becomes a standard reporting requirement,
the fog of war begins to clear not by persuasion but by mandate.