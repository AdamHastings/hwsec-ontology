\section{Evidence-Grounded Case Studies}
\subsection{Why Deep Case Mapping Matters}
Case studies are the bridge between ontology semantics and hardware-security
practice. They test whether classes and relations are expressive enough for
real mechanisms, and they expose burden pathways that mechanism-local reporting
usually hides (transfer, externality, and delayed loss effects)
\cite{anderson2001economics,anderson2006economics}.
In other words, they test whether the burden-allocation instrument framing
matches how mechanisms are engineered and deployed.

\subsection{Case Design and Corpus}
The seed corpus now covers six families (20 tuples each): speculation controls,
TEEs, cryptographic accelerators, rowhammer mitigations, memory-safety
mitigations, and reactive runtime detection/response controls. We additionally
encode 12 incident-linkage tuples to connect weak/missing control families to
loss observations with confidence annotations. Each family is treated as a
distinct burden-allocation instrument profile.

\subsection{Cross-Family Snapshot}
Table~\ref{tab:worked-case} summarizes representative anchors and ontology
dimensions across families.

\begin{table*}[t]
  \caption{Cross-family measurements mapped to ontology dimensions.}
  \label{tab:worked-case}
  \centering
  \scriptsize
  \setlength{\tabcolsep}{3pt}
  \begin{tabular}{p{0.14\textwidth}p{0.07\textwidth}p{0.10\textwidth}p{0.13\textwidth}p{0.21\textwidth}p{0.17\textwidth}p{0.10\textwidth}}
    \toprule
    Family & Posture & Activation & Scope & Representative anchors & Primary bearer(s) & Timing \\
    \midrule
    Speculation controls & Proactive & Often always-on & System-wide & Retbleed slowdown 5.78--27.90\% \cite{retbleed2022}; InvisiSpec 21\% mean overhead vs.\ 74\% STT \cite{invisispec2018} & Operator, platform vendor & Recurring \\
    TEE isolation & Proactive & Mixed & Module specific & Sanctum: +0.78\% gates, +1.9\% flip-flops \cite{sanctum2016}; Graphene-SGX: near-native to below 2x \cite{graphenesgx2017} & Vendor, integrator, operator & Upfront + recurring \\
    Crypto acceleration & Proactive & Conditional & Module specific & OpenTitan AES up to 44x \cite{opentitan2024}; AES-NI 2--3x to 10x \cite{intel_aesni2012} & Vendor, integrator & Upfront + recurring \\
    Rowhammer mitigations & Proactive & Mostly always-on & System-wide memory path & Disturbance characterization \cite{kim2014rowhammer}; TRRespass/Blacksmith bypass pressure \cite{trrespass2020,blacksmith2022} & Vendor, operator & Upfront + recurring \\
    Memory safety mitigations & Proactive & Conditional checks & Module/workload-specific & Arm MTE deployment overhead guidance \cite{arm_mte2021}; CHERI migration tradeoffs \cite{cheri2015} & Vendor, integrator, operator & Upfront + recurring \\
    Runtime detection/response & Reactive & Event-triggered & System-wide telemetry/recovery path & Firmware resiliency/recovery workflow requirements \cite{nist800193}; runtime platform instrumentation context \cite{opentitan2024} & Operator, integrator, customer & Recurring + contingent \\
    \bottomrule
  \end{tabular}
\end{table*}

\subsection{Structured Worked Outputs}
To avoid purely narrative treatment, each family is summarized as
(\textit{tuple slice}, \textit{measured anchors}, \textit{inferred/synthetic
allocations}, \textit{decision implication}) in
Table~\ref{tab:worked-structured}.

\begin{table*}[t]
  \caption{Worked case outputs with explicit data slices and decision implications.}
  \label{tab:worked-structured}
  \centering
  \scriptsize
  \setlength{\tabcolsep}{3pt}
  \begin{tabular}{p{0.13\textwidth}p{0.13\textwidth}p{0.20\textwidth}p{0.23\textwidth}p{0.23\textwidth}}
    \toprule
    Family & Tuple slice & Measured anchors (E1/Measured) & Inferred or synthetic allocations (E2/E3) & Decision implication \\
    \midrule
    Speculation controls & \texttt{S01--S20} & Runtime overhead and ops burden anchors from Retbleed \cite{retbleed2022} & Transfer/externalized burdens under alternative mitigation assumptions from InvisiSpec-context rows \cite{invisispec2018} & ``Best local runtime'' and ``lowest transferred burden'' can diverge. \\
    TEE isolation & \texttt{T21--T40} & Hardware increments and selected runtime behavior anchors from Sanctum \cite{sanctum2016} & Integration and downstream burden variation from Graphene-SGX style adaptation contexts \cite{graphenesgx2017} & Upfront provisioning and recurring adaptation must be jointly optimized. \\
    Crypto acceleration & \texttt{C41--C60} & Crypto-path speedups and area footprints from OpenTitan/AES-NI \cite{opentitan2024,intel_aesni2012} & Opportunity and transfer effects for non-crypto workloads and operations \cite{intel_xeon_crypto2017} & Acceleration benefit does not remove system-level opportunity cost. \\
    Rowhammer mitigations & \texttt{R61--R80} & Disturbance and defense-pressure anchors from ISCA/SP evidence \cite{kim2014rowhammer,trrespass2020,blacksmith2022} & Validation refresh and externalized reliability burden in E2/E3 rows & Durability costs can dominate one-time mitigation overhead. \\
    Memory safety & \texttt{M81--M100} & Runtime and deployment anchors from MTE/CHERI sources \cite{arm_mte2021,cheri2015} & Toolchain migration and transfer/externality terms in E2/E3 rows & Adoption economics materially affect mechanism ranking. \\
    Runtime detection/response & \texttt{D101--D120} & None in current seed (deliberately no E1 reactive rows) & Recovery-time burden, contractual spillover, and contingent loss terms from inferred/synthetic rows \cite{nist800193,anderson2006economics} & Reactive controls expose contingent burden that proactive-only views miss. \\
    \bottomrule
  \end{tabular}
\end{table*}

\subsection{Worked Family Notes}
\textbf{Speculation controls (\texttt{S01--S20}).}
Always-on behavior and system-wide scope produce recurring operator burden. The
seed shows decision-delta behavior where minimizing local runtime and
minimizing transferred burden can select different tuples under different
objective weights.

\textbf{TEE isolation (\texttt{T21--T40}).}
TEE rows show mixed manifestation: upfront hardware/verification and recurring
runtime adaptation. This family prevents collapse into a single ``TEE overhead''
scalar by forcing separate burden channels.

\textbf{Crypto acceleration (\texttt{C41--C60}).}
This family encodes both acceleration gains and opportunity costs.
Allocation-aware analysis keeps ``faster crypto path'' and ``foregone
general-purpose capacity'' in the same decision view.

\textbf{Rowhammer mitigations (\texttt{R61--R80}).}
These rows model mitigation durability pressure: bypass research implies
recurring validation and refresh burden even when first-generation controls are
deployed.

\textbf{Memory safety mitigations (\texttt{M81--M100}).}
Rows mix hardware support, toolchain integration, and migration costs. The
family is useful because burden incidence can move between vendor, integrator,
and operator depending on rollout strategy.

\textbf{Runtime detection/response (\texttt{D101--D120}).}
This reactive family captures event-triggered and contingent burden channels
that are absent from proactive-only corpora: incident handling, recovery
operations, contractual penalties, and reputation effects \cite{nist800193}.
It highlights that reactive mechanisms are burden-allocation instruments with
different timing signatures than proactive mechanisms.

\paragraph{Incident linkage for CQ4.}
Incident tuples (\texttt{I01--I12}) encode observed/seeded losses linked to
families with attribution confidence and provenance. This enables executable
loss-linkage queries instead of treating attribution as future work.

\paragraph{Community value.}
The structured format above is reusable: each family can be extended with
additional rows without changing schema, and review can target concrete tuple
claims rather than prose-only argument. That improves comparability across
architecture, systems-security, and policy analyses.

\paragraph{Seed artifact status.}
The current release is still a seed dataset, not a final benchmark corpus:
120 cost tuples, 12 incident tuples, explicit data-origin labels, and full CQ
coverage under executable checks. This is enough for methodological validation,
while still requiring broader source-complete expansion for benchmark-grade
conclusions.
