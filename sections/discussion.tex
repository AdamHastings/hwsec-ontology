\section{Discussion}
\subsection{Policy and Regulation}
Regulatory requirements can reduce social risk while increasing localized
industry burden; the ontology helps identify when these interventions are
productive versus distortionary.

\subsection{Design Implications}
Architects should evaluate security features by system-level welfare impact,
not only local PPA or benchmark metrics.

\subsection{Limitations}
This draft still has three major limitations: limited measured hardware data,
incomplete incident-loss mapping, and dependence on mixed evidence quality.

\subsection{Threats to Validity}
\begin{itemize}
  \item \textbf{Selection bias}: published mechanisms over-represent successful or benchmark-friendly designs.
  \item \textbf{Measurement heterogeneity}: area, power, and performance numbers are often reported under incompatible assumptions.
  \item \textbf{Attribution ambiguity}: incident losses may be multi-causal and difficult to map to one missing control.
  \item \textbf{Private-cost opacity}: legal, contract, and response costs are frequently proprietary.
\end{itemize}
These risks are mitigated by explicit evidence grading, sensitivity analysis,
and by reporting uncertainty intervals alongside point estimates.

\subsection{Bibliographic and Dataset Maturity}
The current draft cites representative prior work, but a camera-ready version
requires a more systematic corpus review and a source-complete benchmark table
for all numeric claims.
