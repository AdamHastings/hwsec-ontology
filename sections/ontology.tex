\section{Ontology of Security Costs}
\subsection{From Spreadsheet Intuition to Ontology Terms}
To ground the model in familiar engineering artifacts, we map ontology terms to
tabular representations. A \textbf{row} in a dataset corresponds to a
\textbf{cost instance}, a \textbf{column name} (for example bearer or time
horizon) corresponds to a \textbf{property}, an \textbf{allowed value set} (for
example internalized, transferred, externalized) corresponds to a controlled
vocabulary element, and a \textbf{schema constraint} corresponds to an
\textbf{axiom}.
For example, ``Speculation barrier mitigation imposes recurring performance cost
on cloud operators'' is represented as linked facts:
\textit{SpeculationBarrier} \emph{incurs} a \textit{CostInstance}; that instance
has type \textit{MicroarchitecturalPerformanceCost}, is \emph{borneBy}
\textit{CloudOperator}, and is \emph{realizedAt} \textit{Recurring}.
This representation is the concrete ontology form of the burden-allocation
instrument view.

\subsection{Core Entities}
The model contains seven core entity families: assets/objectives (for example
performance, power, area, schedule, trust, safety, and revenue); security
mechanisms and requirements; threats and failure modes; cost types; stakeholder
roles (who decides, pays, benefits, and bears residual risk); lifecycle time
horizons; and evidence records drawn from benchmarks, incidents, audits, and
expert elicitation. The current release also includes explicit
\textit{IncidentEvent} entities and \textit{MeasurementPriority} entities so
loss attribution and information-gap ranking are queryable artifacts rather than
future placeholders.

\subsection{Core Relations}
The principal relations encode burden pathways: a mechanism \emph{incurs} cost,
each cost is \emph{borne by} one or more stakeholders, cost can be \emph{shifted
to} another stakeholder, and cost is \emph{realized at} a lifecycle phase. The
same graph also links mechanisms to threats they \emph{mitigate}, requirements
to design constraints they \emph{impose}, and constraints to opportunity costs
they \emph{induce}. Additional stakeholder-role relations capture who
\emph{decides} (\textit{decidedBy}), who \emph{benefits}
(\textit{benefits}), and who \emph{bears residual risk}
(\textit{bearsResidualRisk}) when incidents occur. Incident entities capture
contingent and reputational losses, while incentive-misalignment relations
capture systematic cost transfer.
Together, these relations treat each mechanism instance as an instrument that
allocates burden and residual risk across an ecosystem.

\subsection{Cost Taxonomy}
We define cost as any measurable reduction in technical, organizational,
economic, or social objective value caused by adopting, operating, or omitting
security controls. The taxonomy spans physical-resource costs (die area,
static/dynamic power, latency, throughput, memory overhead) and
microarchitectural performance costs (for example CPI increases and speculation
restrictions). It also includes engineering labor, verification and validation,
and toolchain/infrastructure burdens. Lifecycle operations and compliance costs
capture key management, patching, field support, audit, and certification
obligations. Finally, the model captures strategic and societal terms: opportunity
cost, market/contractual cost, reputation/trust cost, liability/redress cost,
and externality cost imposed on non-deciding parties.

\subsection{Cost-Bearing Dimensions}
Each instantiated cost is annotated by bearer (vendor, integrator/OEM, cloud
operator, enterprise customer, end user, regulator/public), explicit
decision-maker, transfer target (for transferred/externalized rows), timing
(upfront, recurring, contingent, deferred), bearing mode (internalized,
transferred, insured, externalized), and evidence type (measured, estimated, or
expert-elicited), plus row-level provenance (\texttt{source\_key},
\texttt{source\_locator}) and data-origin maturity
(\texttt{Measured}/\texttt{Inferred}/\texttt{Synthetic}). This representation allows the same mechanism to carry
different burden profiles across deployment contexts even when technical
overhead appears similar.

\subsection{Cost Manifestation Modes}
In addition to ``who pays,'' the ontology tracks \emph{how} costs manifest in
operation through activation profile (always-on, conditional, or
event-triggered), scope (system-wide versus module/workload-specific), workload
interaction (improves, degrades, or redistributes performance by workload
class), and foregone-alternative opportunity cost. Opportunity rows now link to an
explicit \textit{AlternativeUse} node (for example, cache capacity expansion or
general-compute acceleration) rather than only a cost-to-cost edge, while
retaining backward-compatible links for legacy queries. This distinction is
essential for
mechanisms with mixed effects. A broad randomization control may impose
near-constant overhead with wide scope, whereas a dedicated crypto block may
accelerate cryptographic workloads while inducing system-level opportunity cost
through resource diversion from general-purpose performance.

\subsection{Security Posture: Proactive vs Reactive}
The ontology classifies mechanisms as proactive (reduce exploitability before
compromise), reactive (detect, contain, or recover during/after attack), or
hybrid. Posture affects cost incidence: proactive controls often concentrate
burden in upfront design and recurring overhead, while reactive controls shift
burden toward monitoring, incident response, and post-incident recovery.
Modeling posture alongside activation profile and scope helps separate
preventive from response-oriented cost in cross-mechanism comparisons. The seed
corpus now includes both proactive families and a reactive
runtime-detection/response family to make this dimension executable in
cross-family analysis.

\subsection{Initial Cost-Bearing Matrix}
Table~\ref{tab:cost-bearing-matrix} summarizes representative burden patterns
that the ontology is designed to capture and query.

\begin{table*}[t]
  \caption{Illustrative cost-bearing matrix by stakeholder and time horizon.}
  \label{tab:cost-bearing-matrix}
  \centering
  \small
  \begin{tabular}{p{0.17\textwidth}p{0.11\textwidth}p{0.11\textwidth}p{0.11\textwidth}p{0.40\textwidth}}
    \toprule
    Stakeholder & Upfront & Recurring & Contingent & Typical burden classes \\
    \midrule
    Chip vendor & High & Medium & Medium & Physical resource, engineering labor, verification and validation \\
    OEM/integrator & Medium & Medium & Medium & Integration overhead, compliance burden, lifecycle operations \\
    Cloud operator & Medium & High & High & Performance overhead, recurring operations, incident response \\
    Enterprise customer & Low & Medium & High & Migration effort, downtime risk, contractual and liability exposure \\
    End user/public & Low & Low & High & Service degradation, privacy harms, and externalized impacts \\
    \bottomrule
  \end{tabular}
\end{table*}

\subsection{Machine-Readable Representation}
The formal class/property encoding of this taxonomy is maintained in the
companion artifact released with the paper materials, which provides canonical
definitions for cost classes, stakeholder roles, and burden relations.
The companion tuple artifact mirrors these semantics with explicit provenance
fields so every quantitative claim can be traced to a citation key and source
location tag. Companion incident, objective-weight, and SHACL-rule artifacts
extend this to attribution structure, explicit baseline comparison, and semantic
consistency checking.
With these semantics specified, the next section introduces the analytical
framework used to compare mechanisms under alternative decision rules.
